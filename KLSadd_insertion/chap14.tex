\documentclass[envcountchap,graybox]{svmono}

\addtolength{\textwidth}{1mm}

\usepackage{amsmath,amssymb}

\usepackage{amsfonts}
%\usepackage{breqn}
\usepackage{DLMFmath}
\usepackage{DRMFfcns}

\usepackage{mathptmx}
\usepackage{helvet}
\usepackage{courier}

\usepackage{makeidx}
\usepackage{graphicx}

\usepackage{multicol}
\usepackage[bottom]{footmisc}

\makeindex

\def\bibname{Bibliography}
\def\refname{Bibliography}

\def\theequation{\thesection.\arabic{equation}}

\smartqed

\let\corollary=\undefined
\spnewtheorem{corollary}[theorem]{Corollary}{\bfseries}{\itshape}

\newcounter{rom}

\newcommand{\hyp}[5]{\mbox{}_{#1}{F}_{#2}
\left(\genfrac{}{}{0pt}{}{#3}{#4}\,;\,#5\right)}

\newcommand{\qhyp}[5]{\mbox{}_{#1}{\phi}_{#2}
\left(\genfrac{}{}{0pt}{}{#3}{#4}\,;\,q,\,#5\right)}

\newcommand{\qhypK}[5]{\,\mbox{}_{#1}\phi_{#2}\!\left(
  \genfrac{}{}{0pt}{}{#3}{#4};#5\right)}

\newcommand{\mathindent}{\hspace{7.5mm}}

\newcommand{\e}{\textrm{e}}

\renewcommand{\E}{\textrm{E}}

\renewcommand{\textfraction}{-1}

\renewcommand{\Gamma}{\varGamma}

\renewcommand{\leftlegendglue}{\hfil}

\settowidth{\tocchpnum}{14\enspace}
\settowidth{\tocsecnum}{14.30\enspace}
\settowidth{\tocsubsecnum}{14.12.1\enspace}

\makeatletter
\def\cleartoversopage{\clearpage\if@twoside\ifodd\c@page
         \hbox{}\newpage\if@twocolumn\hbox{}\newpage\fi
         \else\fi\fi}

\newcommand{\clearemptyversopage}{
        \clearpage{\pagestyle{empty}\cleartoversopage}}
\makeatother

\oddsidemargin -1.5cm
\topmargin -2.0cm
\textwidth 16.3cm
\textheight 25cm 

\begin{document}

\author{Roelof Koekoek\\[2.5mm]Peter A. Lesky\\[2.5mm]Ren\'e F. Swarttouw}
\title{Hypergeometric orthogonal polynomials and their\\$q$-analogues}
\subtitle{-- Monograph --}
\maketitle

\frontmatter

\large

\pagenumbering{roman}
\addtocounter{chapter}{13}

\chapter{Basic hypergeometric orthogonal polynomials}
\label{BasicHyperOrtPol}


In this chapter we deal with all families of basic hypergeometric orthogonal polynomials
appearing in the $q$-analogue of the Askey scheme on the pages~\pageref{qscheme1} and
\pageref{qscheme2}. For each family of orthogonal polynomials we state the most important
properties such as a representation as a basic hypergeometric function, orthogonality
relation(s), the three-term recurrence relation, the second-order $q$-difference equation,
the forward shift (or degree lowering) and backward shift (or degree raising) operator, a
Rodrigues-type formula and some generating functions. Throughout this chapter we assume
that $0<q<1$. In each case we use the notation which seems to be most common in the
literature. Moreover, in each case we also state the limit relations between various
families of $q$-orthogonal polynomials and the limit relations ($q\rightarrow 1$) to the
classical hypergeometric orthogonal polynomials belonging to the Askey scheme on
page~\pageref{scheme}. For notations the reader is referred to chapter~\ref{Definitions}.


\section{Askey-Wilson}\index{Askey-Wilson polynomials}

\subsection*{Basic hypergeometric representation}
\begin{eqnarray}
\label{DefAskeyWilson}
& &\frac{a^np_n(x;a,b,c,d|q)}{(ab,ac,ad;q)_n}\nonumber\\
& &{}=\qhyp{4}{3}{q^{-n},abcdq^{n-1},a\e^{i\theta},a\e^{-i\theta}}{ab,ac,ad}{q},\quad x=\cos\theta.
\end{eqnarray}
The Askey-Wilson polynomials are $q$-analogues of the Wilson polynomials given by (\ref{DefWilson}).

\subsection*{Orthogonality relation}
If $a,b,c,d$ are real, or occur in complex conjugate pairs if complex, and\\
$\max(|a|,|b|,|c|,|d|)<1$, then we have the following orthogonality relation
\begin{equation}
\label{OrtAskeyWilsonI}
\frac{1}{2\pi}\int_{-1}^1\frac{w(x)}{\sqrt{1-x^2}}p_m(x;a,b,c,d|q)p_n(x;a,b,c,d|q)\,dx=h_n\,\delta_{mn},
\end{equation}
where
\begin{eqnarray*}
w(x):=w(x;a,b,c,d|q)&=&\left|\frac{(\e^{2i\theta};q)_{\infty}}{(a\e^{i\theta},b\e^{i\theta},
c\e^{i\theta},d\e^{i\theta};q)_{\infty}}\right|^2\\
&=&\frac{h(x,1)h(x,-1)h(x,q^{\frac{1}{2}})h(x,-q^{\frac{1}{2}})}
{h(x,a)h(x,b)h(x,c)h(x,d)},
\end{eqnarray*}
with
$$h(x,\alpha):=\prod_{k=0}^{\infty}\left(1-2\alpha xq^k+\alpha^2q^{2k}\right)
=\left(\alpha\e^{i\theta},\alpha\e^{-i\theta};q\right)_{\infty},\quad x=\cos\theta$$
and
$$h_n=\frac{(abcdq^{n-1};q)_n(abcdq^{2n};q)_{\infty}}{(q^{n+1},abq^n,acq^n,adq^n,bcq^n,bdq^n,cdq^n;q)_{\infty}}.$$
If $a>1$ and $b,c,d$ are real or one is real and the other two are complex conjugates,\\
$\max(|b|,|c|,|d|)<1$ and the pairwise products of $a,b,c$ and $d$ have
absolute value less than $1$, then we have another orthogonality relation
given by:
\begin{eqnarray}
\label{OrtAskeyWilsonII}
& &\frac{1}{2\pi}\int_{-1}^1\frac{w(x)}{\sqrt{1-x^2}}p_m(x;a,b,c,d|q)p_n(x;a,b,c,d|q)\,dx\nonumber\\
& &{}\mathindent{}+\sum_{\begin{array}{c}\scriptstyle k\\ \scriptstyle 1<aq^k\leq a\end{array}}
w_kp_m(x_k;a,b,c,d|q)p_n(x_k;a,b,c,d|q)=h_n\,\delta_{mn},
\end{eqnarray}
where $w(x)$ and $h_n$ are as before,
$$x_k=\frac{aq^k+\left(aq^k\right)^{-1}}{2}$$
and
\begin{eqnarray*}
w_k&=&\frac{(a^{-2};q)_{\infty}}{(q,ab,ac,ad,a^{-1}b,a^{-1}c,a^{-1}d;q)_{\infty}}\\
& &{}\mathindent\times\frac{(1-a^2q^{2k})(a^2,ab,ac,ad;q)_k}
{(1-a^2)(q,ab^{-1}q,ac^{-1}q,ad^{-1}q;q)_k}\left(\frac{q}{abcd}\right)^k.
\end{eqnarray*}

\subsection*{Recurrence relation}
\begin{equation}
\label{RecAskeyWilson}
2x{\tilde p}_n(x)=A_n{\tilde p}_{n+1}(x)+\left[a+a^{-1}-\left(A_n+C_n\right)\right]{\tilde p}_n(x)+C_n{\tilde p}_{n-1}(x),
\end{equation}
where
$${\tilde p}_n(x):={\tilde p}_n(x;a,b,c,d|q)=\frac{a^np_n(x;a,b,c,d|q)}{(ab,ac,ad;q)_n}$$
and
$$\left\{\begin{array}{l}
\displaystyle A_n=\frac{(1-abq^n)(1-acq^n)(1-adq^n)(1-abcdq^{n-1})}{a(1-abcdq^{2n-1})(1-abcdq^{2n})}\\
\\
\displaystyle C_n=\frac{a(1-q^n)(1-bcq^{n-1})(1-bdq^{n-1})(1-cdq^{n-1})}{(1-abcdq^{2n-2})(1-abcdq^{2n-1})}.
\end{array}\right.$$

\subsection*{Normalized recurrence relation}
\begin{equation}
\label{NormRecAskeyWilson}
xp_n(x)=p_{n+1}(x)+\frac{1}{2}\left[a+a^{-1}-(A_n+C_n)\right]p_n(x)+
\frac{1}{4}A_{n-1}C_np_{n-1}(x),
\end{equation}
where
$$p_n(x;a,b,c,d|q)=2^n(abcdq^{n-1};q)_np_n(x).$$

\subsection*{$q$-Difference equation}
\begin{eqnarray}
\label{dvAskeyWilson1}
& &(1-q)^2D_q\left[{\tilde w}(x;aq^{\frac{1}{2}},bq^{\frac{1}{2}},
cq^{\frac{1}{2}},dq^{\frac{1}{2}}|q)D_qy(x)\right]\nonumber\\
& &{}\mathindent{}+\lambda_n{\tilde w}(x;a,b,c,d|q)y(x)=0,\quad y(x)=p_n(x;a,b,c,d|q),
\end{eqnarray}
where
$${\tilde w}(x;a,b,c,d|q):=\frac{w(x;a,b,c,d|q)}{\sqrt{1-x^2}}$$
and
$$\lambda_n=4q^{-n+1}(1-q^n)(1-abcdq^{n-1}).$$
If we define
$${\mathcal P}_n(z):=\frac{(ab,ac,ad;q)_n}{a^n}\,\qhyp{4}{3}{q^{-n},abcdq^{n-1},az,az^{-1}}{ab,ac,ad}{q}$$
then the $q$-difference equation can also be written in the form
\begin{eqnarray}
\label{dvAskeyWilson2}
& &q^{-n}(1-q^n)(1-abcdq^{n-1}){\mathcal P}_n(z)\nonumber\\
& &{}=A(z){\mathcal P}_n(qz)-\left[A(z)+A(z^{-1})\right]{\mathcal P}_n(z)+A(z^{-1}){\mathcal P}_n(q^{-1}z),
\end{eqnarray}
where
$$A(z)=\frac{(1-az)(1-bz)(1-cz)(1-dz)}{(1-z^2)(1-qz^2)}.$$

\subsection*{Forward shift operator}
\begin{eqnarray}
\label{shift1AskeyWilsonI}
\delta_q p_n(x;a,b,c,d|q)&=&-q^{-\frac{1}{2}n}(1-q^n)(1-abcdq^{n-1})(\e^{i\theta}-\e^{-i\theta})\nonumber\\
& &{}\mathindent{}\times p_{n-1}(x;aq^{\frac{1}{2}},bq^{\frac{1}{2}},cq^{\frac{1}{2}},dq^{\frac{1}{2}}|q),
\end{eqnarray}
where $x=\cos\theta$, or equivalently
\begin{eqnarray}
\label{shift1AskeyWilsonII}
D_q p_n(x;a,b,c,d|q)&=&2q^{-\frac{1}{2}(n-1)}\frac{(1-q^n)(1-abcdq^{n-1})}{1-q}\nonumber\\
& &{}\mathindent\times p_{n-1}(x;aq^{\frac{1}{2}},bq^{\frac{1}{2}},cq^{\frac{1}{2}},dq^{\frac{1}{2}}|q).
\end{eqnarray}

\subsection*{Backward shift operator}
\begin{eqnarray}
\label{shift2AskeyWilsonI}
& &\delta_q\left[{\tilde w}(x;a,b,c,d|q)p_n(x;a,b,c,d|q)\right]\nonumber\\
& &{}=q^{-\frac{1}{2}(n+1)}(\e^{i\theta}-\e^{-i\theta}){\tilde w}(x;aq^{-\frac{1}{2}},bq^{-\frac{1}{2}},cq^{-\frac{1}{2}},dq^{-\frac{1}{2}}|q)\nonumber\\
& &{}\mathindent{}\times p_{n+1}(x;aq^{-\frac{1}{2}},bq^{-\frac{1}{2}},cq^{-\frac{1}{2}},dq^{-\frac{1}{2}}|q),
\quad x=\cos\theta
\end{eqnarray}
or equivalently
\begin{eqnarray}
\label{shift2AskeyWilsonII}
& &D_q\left[{\tilde w}(x;a,b,c,d|q)p_n(x;a,b,c,d|q)\right]\nonumber\\
& &{}=-\frac{2q^{-\frac{1}{2}n}}{1-q}{\tilde w}(x;aq^{-\frac{1}{2}},bq^{-\frac{1}{2}},cq^{-\frac{1}{2}},dq^{-\frac{1}{2}}|q)\nonumber\\
& &{}\mathindent{}\times p_{n+1}(x;aq^{-\frac{1}{2}},bq^{-\frac{1}{2}},cq^{-\frac{1}{2}},dq^{-\frac{1}{2}}|q).
\end{eqnarray}

\newpage

\subsection*{Rodrigues-type formula}
\begin{eqnarray}
\label{RodAskeyWilson}
& &{\tilde w}(x;a,b,c,d|q)p_n(x;a,b,c,d|q)\nonumber\\
& &{}=\left(\frac{q-1}{2}\right)^nq^{\frac{1}{4}n(n-1)}\left(D_q\right)^n
\left[{\tilde w}(x;aq^{\frac{1}{2}n},bq^{\frac{1}{2}n},cq^{\frac{1}{2}n},dq^{\frac{1}{2}n}|q)\right].
\end{eqnarray}

\subsection*{Generating functions}
\begin{eqnarray}
\label{GenAskeyWilson1}
& &\qhyp{2}{1}{a\e^{i\theta},b\e^{i\theta}}{ab}{\e^{-i\theta}t}\,
\qhyp{2}{1}{c\e^{-i\theta},d\e^{-i\theta}}{cd}{\e^{i\theta}t}\nonumber\\
& &{}=\sum_{n=0}^{\infty}\frac{p_n(x;a,b,c,d|q)}{(ab,cd,q;q)_n}t^n,
\quad x=\cos\theta.
\end{eqnarray}

\begin{eqnarray}
\label{GenAskeyWilson2}
& &\qhyp{2}{1}{a\e^{i\theta},c\e^{i\theta}}{ac}{\e^{-i\theta}t}\,
\qhyp{2}{1}{b\e^{-i\theta},d\e^{-i\theta}}{bd}{\e^{i\theta}t}\nonumber\\
& &{}=\sum_{n=0}^{\infty}\frac{p_n(x;a,b,c,d|q)}{(ac,bd,q;q)_n}t^n,
\quad x=\cos\theta.
\end{eqnarray}

\begin{eqnarray}
\label{GenAskeyWilson3}
& &\qhyp{2}{1}{a\e^{i\theta},d\e^{i\theta}}{ad}{\e^{-i\theta}t}\,
\qhyp{2}{1}{b\e^{-i\theta},c\e^{-i\theta}}{bc}{\e^{i\theta}t}\nonumber\\
& &{}=\sum_{n=0}^{\infty}\frac{p_n(x;a,b,c,d|q)}{(ad,bc,q;q)_n}t^n,
\quad x=\cos\theta.
\end{eqnarray}

\subsection*{Limit relations}

\subsubsection*{Askey-Wilson $\rightarrow$ Continuous dual $q$-Hahn}
The continuous dual $q$-Hahn polynomials given by (\ref{DefContDualqHahn}) simply follow
from the Askey-Wilson polynomials given by (\ref{DefAskeyWilson}) by setting
$d=0$ in (\ref{DefAskeyWilson}):
\begin{equation}
p_n(x;a,b,c,0|q)=p_n(x;a,b,c|q).
\end{equation}

\subsubsection*{Askey-Wilson $\rightarrow$ Continuous $q$-Hahn}
The continuous $q$-Hahn polynomials given by (\ref{DefContqHahn}) can be obtained
from the Askey-Wilson polynomials given by (\ref{DefAskeyWilson}) by the substitutions
$\theta\rightarrow\theta+\phi$, $a\rightarrow a\e^{i\phi}$, $b\rightarrow b\e^{i\phi}$,
$c\rightarrow c\e^{-i\phi}$ and $d\rightarrow d\e^{-i\phi}$:
\begin{equation}
p_n(\cos(\theta+\phi);a\e^{i\phi},b\e^{i\phi},c\e^{-i\phi},d\e^{-i\phi}|q)=p_n(\cos(\theta+\phi);a,b,c,d;q).
\end{equation}

\subsubsection*{Askey-Wilson $\rightarrow$ Big $q$-Jacobi}
The big $q$-Jacobi polynomials given by (\ref{DefBigqJacobi}) can be obtained from the
Askey-Wilson polynomials by setting $x\rightarrow \frac{1}{2}a^{-1}x$, $b=a^{-1}\alpha q$,
$c=a^{-1}\gamma q$ and $d=a\beta \gamma^{-1}$ in
$${\tilde p}_n(x;a,b,c,d|q)=\frac{a^np_n(x;a,b,c,d|q)}{(ab,ac,ad;q)_n}$$
given by (\ref{DefAskeyWilson}) and then taking the limit $a\rightarrow 0$:
\begin{equation}
\lim_{a\rightarrow 0}{\tilde p}_n(\textstyle\frac{1}{2}a^{-1}x;a,a^{-1}\alpha q,
a^{-1}\gamma q,a\beta \gamma^{-1}|q)=P_n(x;\alpha,\beta,\gamma;q).
\end{equation}

\subsubsection*{Askey-Wilson $\rightarrow$ Continuous $q$-Jacobi}
If we take $a=q^{\frac{1}{2}\alpha+\frac{1}{4}}$, $b=q^{\frac{1}{2}\alpha+\frac{3}{4}}$,
$c=-q^{\frac{1}{2}\beta+\frac{1}{4}}$ and $d=-q^{\frac{1}{2}\beta+\frac{3}{4}}$ in
the definition (\ref{DefAskeyWilson}) of the Askey-Wilson polynomials and
change the normalization we find the continuous $q$-Jacobi polynomials given
by (\ref{DefContqJacobi}):
\begin{equation}
\frac{q^{(\frac{1}{2}\alpha+\frac{1}{4})n}p_n(x;q^{\frac{1}{2}\alpha+\frac{1}{4}},q^{\frac{1}{2}\alpha+\frac{3}{4}},
-q^{\frac{1}{2}\beta+\frac{1}{4}},-q^{\frac{1}{2}\beta+\frac{3}{4}}|q)}
{(q,-q^{\frac{1}{2}(\alpha+\beta+1)},-q^{\frac{1}{2}(\alpha+\beta+2)};q)_n}
=P_n^{(\alpha,\beta)}(x|q).
\end{equation}

\subsubsection*{Askey-Wilson $\rightarrow$ Continuous $q$-ultraspherical / Rogers}
If we set $a=\beta^{\frac{1}{2}}$, $b=\beta^{\frac{1}{2}}q^{\frac{1}{2}}$,
$c=-\beta^{\frac{1}{2}}$ and $d=-\beta^{\frac{1}{2}}q^{\frac{1}{2}}$ in the definition
(\ref{DefAskeyWilson}) of the Askey-Wilson polynomials and change the
normalization we obtain the continuous $q$-ultraspherical (or Rogers)
polynomials given by (\ref{DefContqUltra}). In fact we have:
\begin{equation}
\frac{(\beta^2;q)_np_n(x;\beta^{\frac{1}{2}},\beta^{\frac{1}{2}}q^{\frac{1}{2}},
-\beta^{\frac{1}{2}},-\beta^{\frac{1}{2}}q^{\frac{1}{2}}|q)}
{(\beta q^{\frac{1}{2}},-\beta,-\beta q^{\frac{1}{2}},q;q)_n}=C_n(x;\beta|q).
\end{equation}

\subsubsection*{Askey-Wilson $\rightarrow$ Wilson}
To find the Wilson polynomials given by (\ref{DefWilson}) from the Askey-Wilson polynomials we
set $a\rightarrow q^a$, $b\rightarrow q^b$, $c\rightarrow q^c$, $d\rightarrow q^d$ and
$\e^{i\theta}=q^{ix}$ (or $\theta=\ln q^x$) in the definition (\ref{DefAskeyWilson}) and
take the limit $q\rightarrow 1$:
\begin{equation}
\lim_{q\rightarrow 1}\frac{p_n(\frac{1}{2}\left(q^{ix}+q^{-ix}\right);q^a,q^b,q^c,q^d|q)}{(1-q)^{3n}}=W_n(x^2;a,b,c,d).
\end{equation}

\subsection*{Remarks}
The $q$-Racah polynomials given by (\ref{DefqRacah}) and the Askey-Wilson
polynomials given by (\ref{DefAskeyWilson}) are related in the following way.
If we substitute $a^2=\gamma\delta q$, $b^2=\alpha^2\gamma^{-1}\delta^{-1}q$,
$c^2=\beta^2\gamma^{-1}\delta q$, $d^2=\gamma\delta^{-1}q$ and $\e^{2i\theta}=\gamma\delta q^{2x+1}$
in the definition (\ref{DefAskeyWilson})
of the Askey-Wilson polynomials we find:
\begin{eqnarray*}
& &R_n(\mu(x);\alpha,\beta,\gamma,\delta|q)\\
& &{}=\frac{(\gamma\delta q)^{\frac{1}{2}n}
p_n(\nu(x);\gamma^{\frac{1}{2}}\delta^{\frac{1}{2}}q^{\frac{1}{2}},
\alpha\gamma^{-\frac{1}{2}}\delta^{-\frac{1}{2}}q^{\frac{1}{2}},
\beta\gamma^{-\frac{1}{2}}\delta^{\frac{1}{2}}q^{\frac{1}{2}},
\gamma^{\frac{1}{2}}\delta^{-\frac{1}{2}}q^{\frac{1}{2}}|q)}{(\alpha q,\beta\delta q,\gamma q;q)_n},
\end{eqnarray*}
where
$$\nu(x)=\textstyle\frac{1}{2}\gamma^{\frac{1}{2}}\delta^{\frac{1}{2}}q^{x+\frac{1}{2}}
+\textstyle\frac{1}{2}\gamma^{-\frac{1}{2}}\delta^{-\frac{1}{2}}q^{-x-\frac{1}{2}}.$$

\noindent
If we replace $q$ by $q^{-1}$ we find
$${\tilde p}_n(x;a,b,c,d|q^{-1})={\tilde p}_n(x;a^{-1},b^{-1},c^{-1},d^{-1}|q).$$

\subsection*{References}
\cite{Abreu}, \cite{AlSalam90}, \cite{AndrewsAskey85}, \cite{Askey89I},
\cite{AskeyWilson85}, \cite{AskeyRahmanSuslov}, \cite{Atak97}, \cite{AtakRahmanSuslov},
\cite{AtakSuslov88}, \cite{AtakSuslov92}, \cite{Bang}, \cite{BrownEvansIsmail},
\cite{BrownIsmail}, \cite{GasperRahman86}, \cite{GasperRahman90}, \cite{GrunbaumHaine96},
\cite{GrunbaumHaine97}, \cite{Ismail86I}, \cite{Ismail2003}, \cite{IsmailLetMasVal},
\cite{IsmailLetValWimp91}, \cite{IsmailMasson95}, \cite{IsmailMassonRahman},
\cite{IsmailRahman91}, \cite{IsmailStanton88}, \cite{IsmailWilson}, \cite{KalninsMiller89},
\cite{Koelink94}, \cite{Koelink96I}, \cite{Koorn92}, \cite{Koorn93}, \cite{Koorn2007},
\cite{Leonard}, \cite{Miller89}, \cite{MimachiI}, \cite{NassrallahRahman}, \cite{Nikiforov+},
\cite{NoumiMimachi90I}, \cite{NoumiMimachi92}, \cite{Rahman82}, \cite{Rahman85},
\cite{Rahman86II}, \cite{Rahman88}, \cite{Rahman92}, \cite{Rahman96}, \cite{RahmanSuslov},
\cite{RahmanVerma91}, \cite{Spiridonov97}, \cite{Szwarc}, \cite{Vinet}, \cite{Wilson91}.

\newpage

\section{$q$-Racah}\index{q-Racah polynomials@$q$-Racah polynomials}
\par\setcounter{equation}{0}

\subsection*{Basic hypergeometric representation}
\begin{eqnarray}
\label{DefqRacah}
& &R_n(\mu(x);\alpha,\beta,\gamma,\delta|q)\nonumber\\
& &{}=\qhyp{4}{3}{q^{-n},\alpha\beta q^{n+1},q^{-x},\gamma\delta q^{x+1}}{\alpha q,\beta\delta q,\gamma q}{q},
\quad n=0,1,2,\ldots,N,
\end{eqnarray}
where
$$\mu(x):=q^{-x}+\gamma\delta q^{x+1}$$
and
$$\alpha q=q^{-N}\quad\textrm{or}\quad\beta\delta q=q^{-N}\quad\textrm{or}\quad\gamma q=q^{-N},$$
with $N$ a nonnegative integer.
Since
$$(q^{-x},\gamma\delta q^{x+1};q)_k=\prod_{j=0}^{k-1}\left(1-\mu(x)q^j+\gamma\delta q^{2j+1}\right),$$
it is clear that $R_n(\mu(x);\alpha,\beta,\gamma,\delta|q)$ is a polynomial of degree $n$ in $\mu(x)$.

\subsection*{Orthogonality relation}
\begin{eqnarray}
\label{OrtqRacah}
& &\sum_{x=0}^N\frac{(\alpha q,\beta\delta q,\gamma q,\gamma\delta q;q)_x}
{(q,\alpha^{-1}\gamma\delta q,\beta^{-1}\gamma q,\delta q;q)_x}\nonumber\\
& &{}\mathindent\times
\frac{(1-\gamma\delta q^{2x+1})}{(\alpha\beta q)^x(1-\gamma\delta q)} R_m(\mu(x))R_n(\mu(x))
=h_n\,\delta_{mn},
\end{eqnarray}
where
$$R_n(\mu(x)):=R_n(\mu(x);\alpha,\beta,\gamma,\delta|q)$$
and
\begin{eqnarray*}
h_n&=&\frac{(\alpha^{-1}\beta^{-1}\gamma,\alpha^{-1}\delta,\beta^{-1},\gamma\delta q^2;q)_{\infty}}
{(\alpha^{-1}\beta^{-1}q^{-1},\alpha^{-1}\gamma\delta q,\beta^{-1}\gamma q,\delta q;q)_{\infty}}\\
& &{}\mathindent\times\frac{(1-\alpha\beta q)(\gamma\delta q)^n}{(1-\alpha\beta q^{2n+1})}
\frac{(q,\alpha\beta\gamma^{-1}q,\alpha\delta^{-1}q,\beta q;q)_n}{(\alpha q,\alpha\beta q,\beta\delta q,\gamma q;q)_n}.
\end{eqnarray*}
This implies
$$h_n=\left\{\begin{array}{ll}
\displaystyle\frac{(\beta^{-1},\gamma\delta q^2;q)_N}{(\beta^{-1}\gamma q,\delta q;q)_N}
\frac{(1-\beta q^{-N})(\gamma\delta q)^n}{(1-\beta q^{2n-N})}
\frac{(q,\beta q,\beta\gamma^{-1}q^{-N},\delta^{-1}q^{-N};q)_n}{(\beta q^{-N},\beta\delta q,\gamma q,q^{-N};q)_n}
&\quad\textrm{if}\quad\alpha q=q^{-N}\\
\\
\displaystyle\frac{(\alpha\beta q^2,\beta\gamma^{-1};q)_N}{(\alpha\beta\gamma^{-1}q,\beta q;q)_N}
\frac{(1-\alpha\beta q)(\beta^{-1}\gamma q^{-N})^n}{(1-\alpha\beta q^{2n+1})}
\frac{(q,\alpha\beta q^{N+2},\alpha\beta\gamma^{-1}q,\beta q;q)_n}{(\alpha q,\alpha\beta q,\gamma q,q^{-N};q)_n}
&\quad\textrm{if}\quad\beta\delta q=q^{-N}\\
\\
\displaystyle\frac{(\alpha\beta q^2,\delta^{-1};q)_N}{(\alpha\delta^{-1}q,\beta q;q)_N}
\frac{(1-\alpha\beta q)(\delta q^{-N})^n}{(1-\alpha\beta q^{2n+1})}
\frac{(q,\alpha\beta q^{N+2},\alpha\delta^{-1}q,\beta q;q)_n}
{(\alpha q,\alpha\beta q,\beta\delta q,q^{-N};q)_n}
&\quad\textrm{if}\quad\gamma q=q^{-N}.
\end{array}\right.$$

\subsection*{Recurrence relation}
\begin{eqnarray}
\label{RecqRacah}
& &-\left(1-q^{-x}\right)\left(1-\gamma\delta q^{x+1}\right)R_n(\mu(x))\nonumber\\
& &{}=A_nR_{n+1}(\mu(x))-\left(A_n+C_n\right)R_n(\mu(x))+C_nR_{n-1}(\mu(x)),
\end{eqnarray}
where
$$\left\{\begin{array}{l}
\displaystyle A_n=\frac{(1-\alpha q^{n+1})(1-\alpha\beta q^{n+1})(1-\beta\delta q^{n+1})(1-\gamma q^{n+1})}
{(1-\alpha\beta q^{2n+1})(1-\alpha\beta q^{2n+2})}\\
\\
\displaystyle C_n=\frac{q(1-q^n)(1-\beta q^n)(\gamma-\alpha\beta q^n)(\delta-\alpha q^n)}
{(1-\alpha\beta q^{2n})(1-\alpha\beta q^{2n+1})}.
\end{array}\right.$$

\subsection*{Normalized recurrence relation}
\begin{equation}
\label{NormRecqRacah}
xp_n(x)=p_{n+1}(x)+\left[1+\gamma\delta q-(A_n+C_n)\right]p_n(x)+A_{n-1}C_np_{n-1}(x),
\end{equation}
where
$$R_n(\mu(x);\alpha,\beta,\gamma,\delta|q)=
\frac{(\alpha\beta q^{n+1};q)_n}{(\alpha q,\beta\delta q,\gamma q;q)_n}p_n(\mu(x)).$$

\newpage

\subsection*{$q$-Difference equation}
\begin{eqnarray}
\label{dvqRacah1}
& &\Delta\left[w(x-1)B(x-1)\Delta y(x-1)\right]\nonumber\\
& &{}\mathindent{}-q^{-n}(1-q^n)(1-\alpha\beta q^{n+1})w(x)y(x)=0,
\end{eqnarray}
where
$$y(x)=R_n(\mu(x);\alpha,\beta,\gamma,\delta|q)$$
and
$$w(x):=w(x;\alpha,\beta,\gamma,\delta|q)=
\frac{(\alpha q,\beta\delta q,\gamma q,\gamma\delta q;q)_x}{(q,\alpha^{-1}\gamma\delta q,\beta^{-1}\gamma q,\delta q;q)_x}
\frac{(1-\gamma\delta q^{2x+1})}{(\alpha\beta q)^x(1-\gamma\delta q)}$$
and $B(x)$ as below. This $q$-difference equation can also be written in the
form
\begin{eqnarray}
\label{dvqRacah2}
& &q^{-n}(1-q^n)(1-\alpha\beta q^{n+1})y(x)\nonumber\\
& &{}=B(x)y(x+1)-\left[B(x)+D(x)\right]y(x)+D(x)y(x-1),
\end{eqnarray}
where
$$\left\{\begin{array}{l}\displaystyle B(x)=\frac{(1-\alpha q^{x+1})(1-\beta\delta q^{x+1})(1-\gamma q^{x+1})(1-\gamma\delta q^{x+1})}
{(1-\gamma\delta q^{2x+1})(1-\gamma\delta q^{2x+2})}\\
\\
\displaystyle D(x)=\frac{q(1-q^x)(1-\delta q^x)(\beta-\gamma q^x)(\alpha-\gamma\delta q^x)}
{(1-\gamma\delta q^{2x})(1-\gamma\delta q^{2x+1})}.\end{array}\right.$$

\subsection*{Forward shift operator}
\begin{eqnarray}
\label{shift1qRacahI}
& &R_n(\mu(x+1);\alpha,\beta,\gamma,\delta|q)-R_n(\mu(x);\alpha,\beta,\gamma,\delta|q)\nonumber\\
& &{}=\frac{q^{-n-x}(1-q^n)(1-\alpha\beta q^{n+1})(1-\gamma\delta q^{2x+2})}
{(1-\alpha q)(1-\beta\delta q)(1-\gamma q)}\nonumber\\
& &{}\mathindent\times R_{n-1}(\mu(x);\alpha q,\beta q,\gamma q,\delta|q)
\end{eqnarray}
or equivalently
\begin{eqnarray}
\label{shift1qRacahII}
& &\frac{\Delta R_n(\mu(x);\alpha,\beta,\gamma,\delta|q)}{\Delta\mu(x)}\nonumber\\
& &{}=\frac{q^{-n+1}(1-q^n)(1-\alpha\beta q^{n+1})}{(1-q)(1-\alpha q)(1-\beta\delta q)(1-\gamma q)}\nonumber\\
& &{}\mathindent\times R_{n-1}(\mu(x);\alpha q,\beta q,\gamma q,\delta|q).
\end{eqnarray}

\subsection*{Backward shift operator}
\begin{eqnarray}
\label{shift2qRacahI}
& &(1-\alpha q^x)(1-\beta\delta q^x)(1-\gamma q^x)(1-\gamma\delta q^x)R_n(\mu(x);\alpha,\beta,\gamma,\delta|q)\nonumber\\
& &{}\mathindent{}-(1-q^x)(1-\delta q^x)(\alpha-\gamma\delta q^x)(\beta-\gamma q^x)R_n(\mu(x-1);\alpha,\beta,\gamma,\delta|q)\nonumber\\
& &{}=q^x(1-\alpha)(1-\beta\delta)(1-\gamma)(1-\gamma\delta q^{2x})\nonumber\\
& &{}\mathindent\times R_{n+1}(\mu(x);\alpha q^{-1},\beta q^{-1},\gamma q^{-1},\delta|q)
\end{eqnarray}
or equivalently
\begin{eqnarray}
\label{shift2qRacahII}
& &\frac{\nabla\left[{\tilde w}(x;\alpha,\beta,\gamma,\delta|q)R_n(\mu(x);\alpha,\beta,\gamma,\delta|q)\right]}{\nabla\mu(x)}\nonumber\\
& &{}=\frac{1}{(1-q)(1-\gamma\delta)}{\tilde w}(x;\alpha q^{-1},\beta q^{-1},\gamma q^{-1},\delta|q)\nonumber\\
& &{}\mathindent{}\times R_{n+1}(\mu(x);\alpha q^{-1},\beta q^{-1},\gamma q^{-1},\delta|q),
\end{eqnarray}
where
$${\tilde w}(x;\alpha,\beta,\gamma,\delta|q)=\frac{(\alpha q,\beta\delta q,\gamma q,\gamma\delta q;q)_x}
{(q,\alpha^{-1}\gamma\delta q,\beta^{-1}\gamma q,\delta q;q)_x(\alpha\beta)^x}.$$

\subsection*{Rodrigues-type formula}
\begin{eqnarray}
\label{RodqRacah}
& &{\tilde w}(x;\alpha,\beta,\gamma,\delta|q)R_n(\mu(x);\alpha,\beta,\gamma,\delta|q)\nonumber\\
& &{}=(1-q)^n(\gamma\delta q;q)_n\left(\nabla_{\mu}\right)^n
\left[{\tilde w}(x;\alpha q^n,\beta q^n,\gamma q^n,\delta|q)\right],
\end{eqnarray}
where
$$\nabla_{\mu}:=\frac{\nabla}{\nabla\mu(x)}.$$

\subsection*{Generating functions} For $x=0,1,2,\ldots,N$ we have
\begin{equation}
\label{GenqRacah1}
\qhyp{2}{1}{q^{-x},\alpha\gamma^{-1}\delta^{-1}q^{-x}}{\alpha q}{\gamma\delta q^{x+1}t}\,
\qhyp{2}{1}{\beta\delta q^{x+1},\gamma q^{x+1}}{\beta q}{q^{-x}t}
{}=\sum_{n=0}^N\frac{(\beta\delta q,\gamma q;q)_n}{(\beta q,q;q)_n}
R_n(\mu(x);\alpha,\beta,\gamma,\delta|q)t^n,
%  \constraint{ $\beta\delta q=q^{-N}\textrm{or}\gamma q=q^{-N}$ }
\end{equation}

\begin{equation}
\label{GenqRacah2}
\qhyp{2}{1}{q^{-x},\beta\gamma^{-1}q^{-x}}{\beta\delta q}{\gamma\delta q^{x+1}t}\,
\qhyp{2}{1}{\alpha q^{x+1},\gamma q^{x+1}}{\alpha\delta^{-1}q}{q^{-x}t}
{}=\sum_{n=0}^N\frac{(\alpha q,\gamma q;q)_n}{(\alpha\delta^{-1}q,q;q)_n}
R_n(\mu(x);\alpha,\beta,\gamma,\delta|q)t^n,
%  \constraint{ $\beta\delta q=q^{-N}\textrm{or}\gamma q=q^{-N}$ }
\end{equation}

\begin{equation}
\label{GenqRacah3}
\qhyp{2}{1}{q^{-x},\delta^{-1}q^{-x}}{\gamma q}{\gamma\delta q^{x+1}t}\,
\qhyp{2}{1}{\alpha q^{x+1},\beta\delta q^{x+1}}{\alpha\beta\gamma^{-1}q}{q^{-x}t}
{}=\sum_{n=0}^N\frac{(\alpha q,\beta\delta q;q)_n}{(\alpha\beta\gamma^{-1}q,q;q)_n}
R_n(\mu(x);\alpha,\beta,\gamma,\delta|q)t^n,
%  \constraint{ $\beta\delta q=q^{-N}\textrm{or}\gamma q=q^{-N}$ }
\end{equation}

\subsection*{Limit relations}

\subsubsection*{$q$-Racah $\rightarrow$ Big $q$-Jacobi}
The big $q$-Jacobi polynomials given by (\ref{DefBigqJacobi}) can be obtained
from the $q$-Racah polynomials by setting $\delta=0$ in the definition
(\ref{DefqRacah}):
\begin{equation}
R_n(\mu(x);a,b,c,0|q)=P_n(q^{-x};a,b,c;q).
\end{equation}

\subsubsection*{$q$-Racah $\rightarrow$ $q$-Hahn}
The $q$-Hahn polynomials follow from the $q$-Racah polynomials by the substitution
$\delta=0$ and $\gamma q=q^{-N}$ in the definition (\ref{DefqRacah}) of the
$q$-Racah polynomials:
\begin{equation}
R_n(\mu(x);\alpha,\beta,q^{-N-1},0|q)=Q_n(q^{-x};\alpha,\beta,N|q).
\end{equation}
Another way to obtain the $q$-Hahn polynomials from the $q$-Racah
polynomials is by setting $\gamma=0$ and $\delta=\beta^{-1}q^{-N-1}$ in the definition
(\ref{DefqRacah}):
\begin{equation}
R_n(\mu(x);\alpha,\beta,0,\beta^{-1}q^{-N-1}|q)=Q_n(q^{-x};\alpha,\beta,N|q).
\end{equation}
And if we take $\alpha q=q^{-N}$, $\beta\rightarrow\beta\gamma q^{N+1}$ and $\delta=0$ in the
definition (\ref{DefqRacah}) of the $q$-Racah polynomials we find the
$q$-Hahn polynomials given by (\ref{DefqHahn}) in the following way:
\begin{equation}
R_n(\mu(x);q^{-N-1},\beta\gamma q^{N+1},\gamma,0|q)=Q_n(q^{-x};\gamma,\beta,N|q).
\end{equation}
Note that $\mu(x)=q^{-x}$ in each case.

\subsubsection*{$q$-Racah $\rightarrow$ Dual $q$-Hahn}
To obtain the dual $q$-Hahn polynomials from the $q$-Racah polynomials we have to
take $\beta=0$ and $\alpha q=q^{-N}$ in (\ref{DefqRacah}):
\begin{equation}
R_n(\mu(x);q^{-N-1},0,\gamma,\delta|q)=R_n(\mu(x);\gamma,\delta,N|q),
\end{equation}
with
$$\mu(x)=q^{-x}+\gamma\delta q^{x+1}.$$
We may also take $\alpha=0$ and $\beta=\delta^{-1}q^{-N-1}$ in (\ref{DefqRacah}) to obtain
the dual $q$-Hahn polynomials from the $q$-Racah polynomials:
\begin{equation}
R_n(\mu(x);0,\delta^{-1}q^{-N-1},\gamma,\delta|q)=R_n(\mu(x);\gamma,\delta,N|q).
\end{equation}
And if we take $\gamma q=q^{-N}$, $\delta\rightarrow\alpha\delta q^{N+1}$ and $\beta=0$ in the
definition (\ref{DefqRacah}) of the $q$-Racah polynomials we find the dual
$q$-Hahn polynomials given by (\ref{DefDualqHahn}) in the following way:
\begin{equation}
R_n(\mu(x);\alpha,0,q^{-N-1},\alpha\delta q^{N+1}|q)=R_n({\tilde \mu}(x);\alpha,\delta,N|q),
\end{equation}
with
$${\tilde \mu}(x)=q^{-x}+\alpha\delta q^{x+1}.$$

\subsubsection*{$q$-Racah $\rightarrow$ $q$-Krawtchouk}
The $q$-Krawtchouk polynomials given by (\ref{DefqKrawtchouk}) can be obtained from
the $q$-Racah polynomials by setting $\alpha q=q^{-N}$, $\beta=-pq^N$ and
$\gamma=\delta=0$ in the definition (\ref{DefqRacah}) of the $q$-Racah polynomials:
\begin{equation}
R_n(q^{-x};q^{-N-1},-pq^N,0,0|q)=K_n(q^{-x};p,N;q).
\end{equation}
Note that $\mu(x)=q^{-x}$ in this case.

\subsubsection*{$q$-Racah $\rightarrow$ Dual $q$-Krawtchouk}
The dual $q$-Krawtchouk polynomials given by (\ref{DefDualqKrawtchouk}) easily
follow from the $q$-Racah polynomials given by (\ref{DefqRacah}) by using the
substitutions $\alpha=\beta=0$, $\gamma q=q^{-N}$ and $\delta=c$:
\begin{equation}
R_n(\mu(x);0,0,q^{-N-1},c|q)=K_n(\lambda(x);c,N|q).
\end{equation}
Note that
$$\mu(x)=\lambda(x)=q^{-x}+cq^{x-N}.$$

\subsubsection*{$q$-Racah $\rightarrow$ Racah}
If we set $\alpha\rightarrow q^{\alpha}$, $\beta\rightarrow q^{\beta}$, $\gamma\rightarrow q^{\gamma}$,
$\delta\rightarrow q^{\delta}$ in the definition (\ref{DefqRacah}) of the $q$-Racah polynomials
and let $q\rightarrow 1$ we easily obtain the Racah polynomials given by (\ref{DefRacah}):
\begin{equation}
\lim_{q\rightarrow 1}R_n(\mu(x);q^{\alpha},q^{\beta},q^{\gamma},q^{\delta}|q)
=R_n(\lambda(x);\alpha,\beta,\gamma,\delta),
\end{equation}
where
$$\left\{\begin{array}{l}
\displaystyle\mu(x)=q^{-x}+q^{x+\gamma+\delta+1}\\[5mm]
\displaystyle\lambda(x)=x(x+\gamma+\delta+1).
\end{array}\right.$$

\subsection*{Remarks}
The Askey-Wilson polynomials given by (\ref{DefAskeyWilson}) and the $q$-Racah
polynomials given by (\ref{DefqRacah}) are related in the following way.
If we substitute $\alpha=abq^{-1}$, $\beta=cdq^{-1}$, $\gamma=adq^{-1}$,
$\delta=ad^{-1}$ and $q^x=a^{-1}\e^{-i\theta}$ in the definition
(\ref{DefqRacah}) of the $q$-Racah polynomials we find:
$$\mu(x)=2a\cos\theta$$
and
$$R_n(2a\cos\theta;abq^{-1},cdq^{-1},adq^{-1},ad^{-1}|q)
=\frac{a^np_n(x;a,b,c,d|q)}{(ab,ac,ad;q)_n}.$$

\noindent
If we replace $q$ by $q^{-1}$ we find
$$R_n(\mu(x);\alpha,\beta,\gamma,\delta|q^{-1})=R_n({\tilde\mu}(x);\alpha^{-1},\beta^{-1},\gamma^{-1},\delta^{-1}|q),$$
where
$${\tilde\mu}(x):=q^{-x}+\gamma^{-1}\delta^{-1}q^{x+1}.$$

\subsection*{References}
\cite{AlSalam90}, \cite{AlSalamVerma82II}, \cite{AndrewsAskey85},
\cite{AskeyWilson79}, \cite{AskeyWilson85}, \cite{AtakRahmanSuslov},
\cite{LChihara87}, \cite{LChihara93}, \cite{Fischer94},
\cite{GasperRahman83I}, \cite{GasperRahman84}, \cite{GasperRahman90},
\cite{Ismail86I}, \cite{Jain92}, \cite{Koorn90II}, \cite{Nikiforov+},
\cite{Perlstadt}, \cite{Rahman82}.


\section{Continuous dual $q$-Hahn}
\index{Continuous dual q-Hahn polynomials@Continuous dual $q$-Hahn polynomials}
\index{Dual q-Hahn polynomials@Dual $q$-Hahn polynomials!Continuous}
\index{q-Hahn polynomials@$q$-Hahn polynomials!Continuous dual}
\par\setcounter{equation}{0}

\subsection*{Basic hypergeometric representation}
\begin{equation}
\label{DefContDualqHahn}
\frac{a^np_n(x;a,b,c|q)}{(ab,ac;q)_n}
=\qhyp{3}{2}{q^{-n},a\e^{i\theta},a\e^{-i\theta}}{ab,ac}{q},\quad x=\cos\theta.
\end{equation}

\subsection*{Orthogonality relation}
If $a,b,c$ are real, or one is real and the other two are complex conjugates, and
$\max(|a|,|b|,|c|)<1$, then we have the following orthogonality relation
\begin{equation}
\label{OrtContDualqHahnI}
\frac{1}{2\pi}\int_{-1}^1\frac{w(x)}{\sqrt{1-x^2}}p_m(x;a,b,c|q)p_n(x;a,b,c|q)\,dx
=h_n\,\delta_{mn},
\end{equation}
where
$$w(x):=w(x;a,b,c|q)=\left|\frac{(\e^{2i\theta};q)_{\infty}}
{(a\e^{i\theta},b\e^{i\theta},c\e^{i\theta};q)_{\infty}}\right|^2=
\frac{h(x,1)h(x,-1)h(x,q^{\frac{1}{2}})h(x,-q^{\frac{1}{2}})}{h(x,a)h(x,b)h(x,c)},$$
with
$$h(x,\alpha):=\prod_{k=0}^{\infty}\left(1-2\alpha xq^k+\alpha^2q^{2k}\right)
=\left(\alpha\e^{i\theta},\alpha\e^{-i\theta};q\right)_{\infty},\quad x=\cos\theta$$
and
$$h_n=\frac{1}{(q^{n+1},abq^n,acq^n,bcq^n;q)_{\infty}}.$$
If $a>1$ and $b$ and $c$ are real or complex conjugates,
$\max(|b|,|c|)<1$ and the pairwise products of $a,b$ and $c$ have
absolute value less than $1$, then we have another orthogonality relation
given by:
\begin{eqnarray}
\label{OrtContDualqHahnII}
& &\frac{1}{2\pi}\int_{-1}^1\frac{w(x)}{\sqrt{1-x^2}}p_m(x;a,b,c|q)p_n(x;a,b,c|q)\,dx\nonumber\\
& &{}\mathindent{}+\sum_{\begin{array}{c}\scriptstyle k\\ \scriptstyle 1<aq^k\leq a\end{array}}
w_kp_m(x_k;a,b,c|q)p_n(x_k;a,b,c|q)=h_n\,\delta_{mn},
\end{eqnarray}
where $w(x)$ and $h_n$ are as before,
$$x_k=\frac{aq^k+\left(aq^k\right)^{-1}}{2}$$
and
$$w_k=\frac{(a^{-2};q)_{\infty}}{(q,ab,ac,a^{-1}b,a^{-1}c;q)_{\infty}}
\frac{(1-a^2q^{2k})(a^2,ab,ac;q)_k}{(1-a^2)(q,ab^{-1}q,ac^{-1}q;q)_k}
(-1)^kq^{-\binom{k}{2}}\left(\frac{1}{a^2bc}\right)^k.$$

\subsection*{Recurrence relation}
\begin{equation}
\label{RecContDualqHahn}
2x{\tilde p}_n(x)=A_n{\tilde p}_{n+1}(x)+\left[a+a^{-1}-\left(A_n+C_n\right)\right]{\tilde p}_n(x)+C_n{\tilde p}_{n-1}(x),
\end{equation}
where
$${\tilde p}_n(x):={\tilde p}_n(x;a,b,c|q)=\frac{a^np_n(x;a,b,c|q)}{(ab,ac;q)_n}$$
and
$$\left\{\begin{array}{l}
\displaystyle A_n=a^{-1}(1-abq^n)(1-acq^n)\\
\\
\displaystyle C_n=a(1-q^n)(1-bcq^{n-1}).
\end{array}\right.$$

\subsection*{Normalized recurrence relation}
\begin{eqnarray}
\label{NormRecContDualqHahn}
xp_n(x)&=&p_{n+1}(x)+\frac{1}{2}\left[a+a^{-1}-(A_n+C_n)\right]p_n(x)\nonumber\\
& &{}\mathindent{}+\frac{1}{4}(1-q^n)(1-abq^{n-1})\nonumber\\
& &{}\mathindent\mathindent{}\times(1-acq^{n-1})(1-bcq^{n-1})p_{n-1}(x),
\end{eqnarray}
where
$$p_n(x;a,b,c|q)=2^np_n(x).$$

\subsection*{$q$-Difference equation}
\begin{eqnarray}
\label{dvContDualqHahn1}
& &(1-q)^2D_q\left[{\tilde w}(x;aq^{\frac{1}{2}},bq^{\frac{1}{2}},
cq^{\frac{1}{2}}|q)D_qy(x)\right]\nonumber\\
& &{}\mathindent{}+4q^{-n+1}(1-q^n){\tilde w}(x;a,b,c|q)y(x)=0,
\end{eqnarray}
where
$$y(x)=p_n(x;a,b,c|q)$$
and
$${\tilde w}(x;a,b,c|q):=\frac{w(x;a,b,c|q)}{\sqrt{1-x^2}}.$$
If we define
$${\mathcal P}_n(z):=\frac{(ab,ac;q)_n}{a^n}\,\qhyp{3}{2}{q^{-n},az,az^{-1}}{ab,ac}{q}$$
then the $q$-difference equation can also be written in the form
\begin{equation}
\label{dvContDualqHahn2}
q^{-n}(1-q^n){\mathcal P}_n(z)=A(z){\mathcal P}_n(qz)-\left[A(z)+A(z^{-1})\right]{\mathcal P}_n(z)
+A(z^{-1}){\mathcal P}_n(q^{-1}z),
\end{equation}
where
$$A(z)=\frac{(1-az)(1-bz)(1-cz)}{(1-z^2)(1-qz^2)}.$$

\subsection*{Forward shift operator}
\begin{eqnarray}
\label{shift1ContDualqHahnI}
\delta_q p_n(x;a,b,c|q)&=&-q^{-\frac{1}{2}n}(1-q^n)(\e^{i\theta}-\e^{-i\theta})\nonumber\\
& &{}\mathindent{}\times p_{n-1}(x;aq^{\frac{1}{2}},bq^{\frac{1}{2}},cq^{\frac{1}{2}}|q),\quad x=\cos\theta
\end{eqnarray}
or equivalently
\begin{equation}
\label{shift1ContDualqHahnII}
D_q p_n(x;a,b,c|q)=
2q^{-\frac{1}{2}(n-1)}\frac{1-q^n}{1-q}p_{n-1}(x;aq^{\frac{1}{2}},bq^{\frac{1}{2}},cq^{\frac{1}{2}}|q).
\end{equation}

\subsection*{Backward shift operator}
\begin{eqnarray}
\label{shift2ContDualqHahnI}
& &\delta_q\left[{\tilde w}(x;a,b,c|q)p_n(x;a,b,c|q)\right]\nonumber\\
& &{}=q^{-\frac{1}{2}(n+1)}(\e^{i\theta}-\e^{-i\theta})
{\tilde w}(x;aq^{-\frac{1}{2}},bq^{-\frac{1}{2}},cq^{-\frac{1}{2}}|q)\nonumber\\
& &{}\mathindent{}\times p_{n+1}(x;aq^{-\frac{1}{2}},bq^{-\frac{1}{2}},cq^{-\frac{1}{2}}|q),
\quad x=\cos\theta
\end{eqnarray}
or equivalently
\begin{eqnarray}
\label{shift2ContDualqHahnII}
& &D_q\left[{\tilde w}(x;a,b,c|q)p_n(x;a,b,c|q)\right]\nonumber\\
& &{}=-\frac{2q^{-\frac{1}{2}n}}{1-q}{\tilde w}(x;aq^{-\frac{1}{2}},bq^{-\frac{1}{2}},cq^{-\frac{1}{2}}|q)
p_{n+1}(x;aq^{-\frac{1}{2}},bq^{-\frac{1}{2}},cq^{-\frac{1}{2}}|q).
\end{eqnarray}

\subsection*{Rodrigues-type formula}
\begin{eqnarray}
\label{RodContDualqHahn}
& &{\tilde w}(x;a,b,c|q)p_n(x;a,b,c|q)\nonumber\\
& &{}=\left(\frac{q-1}{2}\right)^nq^{\frac{1}{4}n(n-1)}\left(D_q\right)^n\left[{\tilde w}(x;aq^{\frac{1}{2}n},bq^{\frac{1}{2}n},cq^{\frac{1}{2}n}|q)\right].
\end{eqnarray}

\subsection*{Generating functions}
\begin{eqnarray}
\label{GenContDualqHahn1}
& &\frac{(ct;q)_{\infty}}{(\e^{i\theta}t;q)_{\infty}}\,
\qhyp{2}{1}{a\e^{i\theta},b\e^{i\theta}}{ab}{\e^{-i\theta}t}\nonumber\\
& &{}=\sum_{n=0}^{\infty}\frac{p_n(x;a,b,c|q)}{(ab,q;q)_n}t^n,
\quad x=\cos\theta.
\end{eqnarray}

\begin{eqnarray}
\label{GenContDualqHahn2}
& &\frac{(bt;q)_{\infty}}{(\e^{i\theta}t;q)_{\infty}}\,
\qhyp{2}{1}{a\e^{i\theta},c\e^{i\theta}}{ac}{\e^{-i\theta}t}\nonumber\\
& &{}=\sum_{n=0}^{\infty}\frac{p_n(x;a,b,c|q)}{(ac,q;q)_n}t^n,
\quad x=\cos\theta.
\end{eqnarray}

\begin{eqnarray}
\label{GenContDualqHahn3}
& &\frac{(at;q)_{\infty}}{(\e^{i\theta}t;q)_{\infty}}\,
\qhyp{2}{1}{b\e^{i\theta},c\e^{i\theta}}{bc}{\e^{-i\theta}t}\nonumber\\
& &{}=\sum_{n=0}^{\infty}\frac{p_n(x;a,b,c|q)}{(bc,q;q)_n}t^n,
\quad x=\cos\theta.
\end{eqnarray}

\begin{eqnarray}
& &\label{GenContDualqHahn4}
(t;q)_{\infty}\cdot\qhyp{3}{2}{a\e^{i\theta},a\e^{-i\theta},0}{ab,ac}{t}\nonumber\\
& &{}=\sum_{n=0}^{\infty}\frac{(-1)^na^nq^{\binom{n}{2}}}{(ab,ac,q;q)_n}
p_n(x;a,b,c|q)t^n,\quad x=\cos\theta.
\end{eqnarray}

\subsection*{Limit relations}

\subsubsection*{Askey-Wilson $\rightarrow$ Continuous dual $q$-Hahn}
The continuous dual $q$-Hahn polynomials given by (\ref{DefContDualqHahn}) simply follow
from the Askey-Wilson polynomials given by (\ref{DefAskeyWilson}) by setting
$d=0$ in (\ref{DefAskeyWilson}):
$$p_n(x;a,b,c,0|q)=p_n(x;a,b,c|q).$$

\subsubsection*{Continuous dual $q$-Hahn $\rightarrow$ Al-Salam-Chihara}
The Al-Salam-Chihara polynomials given by (\ref{DefAlSalamChihara}) simply follow
from the continuous dual $q$-Hahn polynomials by taking $c=0$ in the
definition (\ref{DefContDualqHahn}) of the continuous dual $q$-Hahn polynomials:
\begin{equation}
p_n(x;a,b,0|q)=Q_n(x;a,b|q).
\end{equation}

\subsubsection*{Continuous dual $q$-Hahn $\rightarrow$ Continuous dual Hahn}
To find the continuous dual Hahn polynomials given by (\ref{DefContDualHahn})
from the continuous dual $q$-Hahn polynomials we set $a\rightarrow q^a$,
$b\rightarrow q^b$, $c\rightarrow q^c$ and $\e^{i\theta}=q^{ix}$
(or $\theta=\ln q^x$) in the definition (\ref{DefContDualqHahn}) and take the
limit $q\rightarrow 1$:
\begin{equation}
\lim_{q\rightarrow 1}\frac{p_n(\frac{1}{2}\left(q^{ix}+q^{-ix}\right);q^a,q^b,q^c|q)}{(1-q)^{2n}}=S_n(x^2;a,b,c).
\end{equation}

\subsection*{References}
\cite{AskeyRahmanSuslov}, \cite{Gupta96}.


\section{Continuous $q$-Hahn}
\index{Continuous q-Hahn polynomials@Continuous $q$-Hahn polynomials}
\index{q-Hahn polynomials@$q$-Hahn polynomials!Continuous}
\par\setcounter{equation}{0}

\subsection*{Basic hypergeometric representation}
\begin{eqnarray}
\label{DefContqHahn}
& &\frac{(a\e^{i\phi})^np_n(x;a,b,c,d;q)}{(ab\e^{2i\phi},ac,ad;q)_n}\nonumber\\
& &{}=\qhyp{4}{3}{q^{-n},abcdq^{n-1},a\e^{i(\theta+2\phi)},a\e^{-i\theta}}
{ab\e^{2i\phi},ac,ad}{q},\quad x=\cos(\theta+\phi).
\end{eqnarray}

\subsection*{Orthogonality relation}
If $c=a$ and $d=b$ then we have, if $a$ and $b$ are real and
$\max(|a|,|b|)<1$, or if $b=\overline{a}$ and $|a|<1$:
\begin{eqnarray}
\label{OrtContqHahn}
& &\frac{1}{4\pi}\int_{-\pi}^{\pi}w(\cos(\theta+\phi))
p_m(\cos(\theta+\phi);a,b,c,d;q)p_n(\cos(\theta+\phi);a,b,c,d;q)\,d\theta\nonumber\\
& &{}=\frac{(abcdq^{n-1};q)_n(abcdq^{2n};q)_{\infty}}
{(q^{n+1},abq^n\e^{2i\phi},acq^n,adq^n,bcq^n,bdq^n,cdq^n\e^{-2i\phi};q)_{\infty}}\,\delta_{mn},
\end{eqnarray}
where
\begin{eqnarray*} w(x):=w(x;a,b,c,d;q)&=&\left|\frac{(\e^{2i(\theta+\phi)};q)_{\infty}}
{(a\e^{i(\theta+2\phi)},b\e^{i(\theta+2\phi)},
c\e^{i\theta},d\e^{i\theta};q)_{\infty}}\right|^2\\
&=&\frac{h(x,1)h(x,-1)h(x,q^{\frac{1}{2}})h(x,-q^{\frac{1}{2}})}
{h(x,a\e^{i\phi})h(x,b\e^{i\phi})h(x,c\e^{-i\phi})h(x,d\e^{-i\phi})},
\end{eqnarray*}
with
$$h(x,\alpha):=\prod_{k=0}^{\infty}\left(1-2\alpha xq^k+\alpha^2q^{2k}\right)
=\left(\alpha\e^{i(\theta+\phi)},\alpha\e^{-i(\theta+\phi)};q\right)_{\infty},
\quad x=\cos(\theta+\phi).$$

\subsection*{Recurrence relation}
\begin{equation}
\label{RecContqHahn}
2x{\tilde p}_n(x)=A_n{\tilde p}_{n+1}(x)+\left[a\e^{i\phi}+a^{-1}\e^{-i\phi}-\left(A_n+C_n\right)\right]{\tilde p}_n(x)+C_n{\tilde p}_{n-1}(x),
\end{equation}
where
$${\tilde p}_n(x):={\tilde p}_n(x;a,b,c,d;q)=\frac{(a\e^{i\phi})^np_n(x;a,b,c,d;q)}{(ab\e^{2i\phi},ac,ad;q)_n}$$
and
$$\left\{\begin{array}{l}
\displaystyle A_n=\frac{(1-ab\e^{2i\phi}q^n)(1-acq^n)(1-adq^n)(1-abcdq^{n-1})}{a\e^{i\phi}(1-abcdq^{2n-1})(1-abcdq^{2n})}\\
\\
\displaystyle C_n=\frac{a\e^{i\phi}(1-q^n)(1-bcq^{n-1})(1-bdq^{n-1})(1-cd\e^{-2i\phi}q^{n-1})}{(1-abcdq^{2n-2})(1-abcdq^{2n-1})}.
\end{array}\right.$$

\newpage

\subsection*{Normalized recurrence relation}
\begin{eqnarray}
\label{NormRecContqHahn}
xp_n(x)&=&p_{n+1}(x)+\frac{1}{2}\left[a\e^{i\phi}+a^{-1}\e^{-i\phi}-(A_n+C_n)\right]p_n(x)\nonumber\\
& &{}\mathindent{}+\frac{1}{4}A_{n-1}C_np_{n-1}(x),
\end{eqnarray}
where
$$p_n(x;a,b,c,d;q)=2^n(abcdq^{n-1};q)_np_n(x).$$

\subsection*{$q$-Difference equation}
\begin{eqnarray}
\label{dvContqHahn}
& &(1-q)^2D_q\left[{\tilde w}(x;aq^{\frac{1}{2}},bq^{\frac{1}{2}},
cq^{\frac{1}{2}},dq^{\frac{1}{2}};q)D_qy(x)\right]\nonumber\\
& &{}\mathindent{}+\lambda_n{\tilde w}(x;a,b,c,d;q)y(x)=0,
\quad y(x)=p_n(x;a,b,c,d;q),
\end{eqnarray}
where
$${\tilde w}(x;a,b,c,d;q):=\frac{w(x;a,b,c,d;q)}{\sqrt{1-x^2}}$$
and
$$\lambda_n=4q^{-n+1}(1-q^n)(1-abcdq^{n-1}).$$

\subsection*{Forward shift operator}
\begin{eqnarray}
\label{shift1ContqHahnI}
& &\delta_q p_n(x;a,b,c,d;q)\nonumber\\
& &{}=-q^{-\frac{1}{2}n}(1-q^n)(1-abcdq^{n-1})(\e^{i(\theta+\phi)}-\e^{-i(\theta+\phi)})\nonumber\\
& &{}\mathindent{}\times p_{n-1}(x;aq^{\frac{1}{2}},bq^{\frac{1}{2}},cq^{\frac{1}{2}},dq^{\frac{1}{2}};q),
\quad x=\cos(\theta+\phi)
\end{eqnarray}
or equivalently
\begin{eqnarray}
\label{shift1ContqHahnII}
D_q p_n(x;a,b,c,d;q)&=&2q^{-\frac{1}{2}(n-1)}\frac{(1-q^n)(1-abcdq^{n-1})}{1-q}\nonumber\\
& &{}\mathindent{}\times p_{n-1}(x;aq^{\frac{1}{2}},bq^{\frac{1}{2}},cq^{\frac{1}{2}},dq^{\frac{1}{2}};q).
\end{eqnarray}

\newpage

\subsection*{Backward shift operator}
\begin{eqnarray}
\label{shift2ContqHahnI}
& &\delta_q\left[{\tilde w}(x;a,b,c,d;q)p_n(x;a,b,c,d;q)\right]\nonumber\\
& &{}=q^{-\frac{1}{2}(n+1)}(\e^{i(\theta+\phi)}-\e^{-i(\theta+\phi)})
{\tilde w}(x;aq^{-\frac{1}{2}},bq^{-\frac{1}{2}},cq^{-\frac{1}{2}},dq^{-\frac{1}{2}};q)\nonumber\\
& &{}\mathindent{}\times p_{n+1}(x;aq^{-\frac{1}{2}},bq^{-\frac{1}{2}},cq^{-\frac{1}{2}},dq^{-\frac{1}{2}};q),
\quad x=\cos(\theta+\phi)
\end{eqnarray}
or equivalently
\begin{eqnarray}
\label{shift2ContqHahnII}
& &D_q\left[{\tilde w}(x;a,b,c,d;q)p_n(x;a,b,c,d;q)\right]\nonumber\\
& &{}=-\frac{2q^{-\frac{1}{2}n}}{1-q}{\tilde w}(x;aq^{-\frac{1}{2}},bq^{-\frac{1}{2}},cq^{-\frac{1}{2}},dq^{-\frac{1}{2}};q)\nonumber\\
& &{}\mathindent{}\times p_{n+1}(x;aq^{-\frac{1}{2}},bq^{-\frac{1}{2}},cq^{-\frac{1}{2}},dq^{-\frac{1}{2}};q).
\end{eqnarray}

\subsection*{Rodrigues-type formula}
\begin{eqnarray}
\label{RodContqHahn}
& &{\tilde w}(x;a,b,c,d;q)p_n(x;a,b,c,d;q)\nonumber\\
& &{}=\left(\frac{q-1}{2}\right)^nq^{\frac{1}{4}n(n-1)}\left(D_q\right)^n
\left[{\tilde w}(x;aq^{\frac{1}{2}n},bq^{\frac{1}{2}n},cq^{\frac{1}{2}n},dq^{\frac{1}{2}n};q)\right].
\end{eqnarray}

\subsection*{Generating functions}
\begin{eqnarray}
\label{GenContqHahn1}
& &\qhyp{2}{1}{a\e^{i(\theta+2\phi)},b\e^{i(\theta+2\phi)}}{ab\e^{2i\phi}}{\e^{-i(\theta+\phi)}t}\nonumber\\
& &{}\mathindent\times
\qhyp{2}{1}{c\e^{-i(\theta+2\phi)},d\e^{-i(\theta+2\phi)}}{cd\e^{-2i\phi}}{\e^{i(\theta+\phi)}t}\nonumber\\
& &{}=\sum_{n=0}^{\infty}\frac{p_n(x;a,b,c,d;q)t^n}{(ab\e^{2i\phi},cd\e^{-2i\phi},q;q)_n},
\quad x=\cos(\theta+\phi).
\end{eqnarray}

\begin{eqnarray}
\label{GenContqHahn2}
& &\qhyp{2}{1}{a\e^{i(\theta+2\phi)},c\e^{i\theta}}{ac}{\e^{-i(\theta+\phi)}t}\,
\qhyp{2}{1}{b\e^{-i\theta},d\e^{-i(\theta+2\phi)}}{bd}{\e^{i(\theta+\phi)}t}\nonumber\\
& &{}=\sum_{n=0}^{\infty}\frac{p_n(x;a,b,c,d;q)}{(ac,bd,q;q)_n}t^n,\quad x=\cos(\theta+\phi).
\end{eqnarray}

\begin{eqnarray}
\label{GenContqHahn3}
& &\qhyp{2}{1}{a\e^{i(\theta+2\phi)},d\e^{i\theta}}{ad}{\e^{-i(\theta+\phi)}t}\,
\qhyp{2}{1}{b\e^{-i\theta},c\e^{-i(\theta+2\phi)}}{bc}{\e^{i(\theta+\phi)}t}\nonumber\\
& &{}=\sum_{n=0}^{\infty}\frac{p_n(x;a,b,c,d;q)}{(ad,bc,q;q)_n}t^n,\quad x=\cos(\theta+\phi).
\end{eqnarray}

\subsection*{Limit relations}

\subsubsection*{Askey-Wilson $\rightarrow$ Continuous $q$-Hahn}
The continuous $q$-Hahn polynomials given by (\ref{DefContqHahn}) can be obtained
from the Askey-Wilson polynomials given by (\ref{DefAskeyWilson}) by the substitutions
$\theta\rightarrow\theta+\phi$, $a\rightarrow a\e^{i\phi}$, $b\rightarrow b\e^{i\phi}$,
$c\rightarrow c\e^{-i\phi}$ and $d\rightarrow d\e^{-i\phi}$:
$$p_n(\cos(\theta+\phi);a\e^{i\phi},b\e^{i\phi},c\e^{-i\phi},d\e^{-i\phi}|q)=p_n(\cos(\theta+\phi);a,b,c,d;q).$$

\subsubsection*{Continuous $q$-Hahn $\rightarrow$ $q$-Meixner-Pollaczek}
The $q$-Meixner-Pollaczek polynomials given by (\ref{DefqMP}) simply follow
from the continuous $q$-Hahn polynomials if we set $d=a$ and $b=c=0$ in the
definition (\ref{DefContqHahn}) of the continuous $q$-Hahn polynomials:
\begin{equation}
\frac{p_n(\cos(\theta+\phi);a,0,0,a;q)}{(q;q)_n}=P_n(\cos(\theta+\phi);a|q).
\end{equation}

\subsubsection*{Continuous $q$-Hahn $\rightarrow$ Continuous Hahn}
If we set $a\rightarrow q^a$, $b\rightarrow q^b$, $c\rightarrow q^c$, $d\rightarrow q^d$
and $\e^{-i\theta}=q^{ix}$ (or $\theta=\ln q^{-x}$) in the definition (\ref{DefContqHahn})
of the continuous $q$-Hahn polynomials and take the limit $q\rightarrow 1$ we find
the continuous Hahn polynomials given by (\ref{DefContHahn}) in the following way:
\begin{equation}
\lim_{q\rightarrow 1}\frac{p_n(\cos(\ln q^{-x}+\phi);q^a,q^b,q^c,q^d;q)}
{(1-q)^n(q;q)_n}=(-2\sin\phi)^np_n(x;a,b,c,d).
\end{equation}

\newpage

\subsection*{Remark}
If we replace $q$ by $q^{-1}$ we find
$${\tilde p}_n(x;a,b,c,d;q^{-1})={\tilde p}_n(x;a^{-1},b^{-1},c^{-1},d^{-1};q).$$

\subsection*{References}
\cite{AndrewsAskey85}, \cite{AskeyWilson85}, \cite{GasperRahman90}.


\section{Big $q$-Jacobi}
\index{Big q-Jacobi polynomials@Big $q$-Jacobi polynomials}
\index{q-Jacobi polynomials@$q$-Jacobi polynomials!Big}
\par\setcounter{equation}{0}

\subsection*{Basic hypergeometric representation}
\begin{equation}
\label{DefBigqJacobi}
P_n(x;a,b,c;q)=\qhyp{3}{2}{q^{-n},abq^{n+1},x}{aq,cq}{q}.
\end{equation}

\subsection*{Orthogonality relation}
For $0<aq<1$, $0\leq bq<1$ and $c<0$ we have
\begin{eqnarray}
\label{OrtBigqJacobi}
& &\int_{cq}^{aq}\frac{(a^{-1}x,c^{-1}x;q)_{\infty}}{(x,bc^{-1}x;q)_{\infty}}
P_m(x;a,b,c;q)P_n(x;a,b,c;q)\,d_qx\nonumber\\
& &{}=aq(1-q)\frac{(q,abq^2,a^{-1}c,ac^{-1}q;q)_{\infty}}{(aq,bq,cq,abc^{-1}q;q)_{\infty}}\nonumber\\
& &{}\mathindent{}\times\frac{(1-abq)}{(1-abq^{2n+1})}
\frac{(q,bq,abc^{-1}q;q)_n}{(aq,abq,cq;q)_n}(-acq^2)^nq^{\binom{n}{2}}\,\delta_{mn}.
\end{eqnarray}

\subsection*{Recurrence relation}
\begin{eqnarray}
\label{RecBigqJacobi}
(x-1)P_n(x;a,b,c;q)&=&A_nP_{n+1}(x;a,b,c;q)-\left(A_n+C_n\right)P_n(x;a,b,c;q)\nonumber\\
& &{}\mathindent{}+C_nP_{n-1}(x;a,b,c;q),
\end{eqnarray}
where
$$\left\{\begin{array}{l}
\displaystyle A_n=\frac{(1-aq^{n+1})(1-abq^{n+1})(1-cq^{n+1})}{(1-abq^{2n+1})(1-abq^{2n+2})}\\
\\
\displaystyle C_n=-acq^{n+1}\frac{(1-q^n)(1-abc^{-1}q^n)(1-bq^n)}{(1-abq^{2n})(1-abq^{2n+1})}.
\end{array}\right.$$

\subsection*{Normalized recurrence relation}
\begin{equation}
\label{NormRecBigqJacobi}
xp_n(x)=p_{n+1}(x)+\left[1-(A_n+C_n)\right]p_n(x)+A_{n-1}C_np_{n-1}(x),
\end{equation}
where
$$P_n(x;a,b,c;q)=\frac{(abq^{n+1};q)_n}{(aq,cq;q)_n}p_n(x).$$

\subsection*{$q$-Difference equation}
\begin{eqnarray}
\label{dvBigqJacobi}
& &q^{-n}(1-q^n)(1-abq^{n+1})x^2y(x)\nonumber\\
& &{}=B(x)y(qx)-\left[B(x)+D(x)\right]y(x)+D(x)y(q^{-1}x),
\end{eqnarray}
where
$$y(x)=P_n(x;a,b,c;q)$$
and
$$\left\{\begin{array}{l}\displaystyle B(x)=aq(x-1)(bx-c)\\
\\
\displaystyle D(x)=(x-aq)(x-cq).\end{array}\right.$$

\subsection*{Forward shift operator}
\begin{eqnarray}
\label{shift1BigqJacobiI}
& &P_n(x;a,b,c;q)-P_n(qx;a,b,c;q)\nonumber\\
& &{}=\frac{q^{-n+1}(1-q^n)(1-abq^{n+1})}{(1-aq)(1-cq)}xP_{n-1}(qx;aq,bq,cq;q)
\end{eqnarray}
or equivalently
\begin{equation}
\label{shift1BigqJacobiII}
\mathcal{D}_qP_n(x;a,b,c;q)=\frac{q^{-n+1}(1-q^n)(1-abq^{n+1})}
{(1-q)(1-aq)(1-cq)}P_{n-1}(qx;aq,bq,cq;q).
\end{equation}

\subsection*{Backward shift operator}
\begin{eqnarray}
\label{shift2BigqJacobiI}
& &(x-a)(x-c)P_n(x;a,b,c;q)-a(x-1)(bx-c)P_n(qx;a,b,c;q)\nonumber\\
& &{}=(1-a)(1-c)xP_{n+1}(x;aq^{-1},bq^{-1},cq^{-1};q)
\end{eqnarray}
or equivalently
\begin{eqnarray}
\label{shift2BigqJacobiII}
& &\mathcal{D}_q\left[w(x;a,b,c;q)P_n(x;a,b,c;q)\right]\nonumber\\
& &{}=\frac{(1-a)(1-c)}{ac(1-q)}w(x;aq^{-1},bq^{-1},cq^{-1};q)\nonumber\\
& &{}\mathindent{}\times P_{n+1}(x;aq^{-1},bq^{-1},cq^{-1};q),
\end{eqnarray}
where
$$w(x;a,b,c;q)=\frac{(a^{-1}x,c^{-1}x;q)_{\infty}}{(x,bc^{-1}x;q)_{\infty}}.$$

\subsection*{Rodrigues-type formula}
\begin{eqnarray}
\label{RodBigqJacobi}
& &w(x;a,b,c;q)P_n(x;a,b,c;q)\nonumber\\
& &{}=\frac{a^nc^nq^{n(n+1)}(1-q)^n}
{(aq,cq;q)_n}\left(\mathcal{D}_q\right)^n\left[w(x;aq^n,bq^n,cq^n;q)\right].
\end{eqnarray}

\subsection*{Generating functions}
\begin{eqnarray}
\label{GenBigqJacobi1}
& &\qhyp{2}{1}{aqx^{-1},0}{aq}{xt}\,\qhyp{1}{1}{bc^{-1}x}{bq}{cqt}\nonumber\\
& &{}=\sum_{n=0}^{\infty}\frac{(cq;q)_n}{(bq,q;q)_n}P_n(x;a,b,c;q)t^n.
\end{eqnarray}

\begin{eqnarray}
\label{GenBigqJacobi2}
& &\qhyp{2}{1}{cqx^{-1},0}{cq}{xt}\,\qhyp{1}{1}{bc^{-1}x}{abc^{-1}q}{aqt}\nonumber\\
& &{}=\sum_{n=0}^{\infty}\frac{(aq;q)_n}{(abc^{-1}q,q;q)_n}P_n(x;a,b,c;q)t^n.
\end{eqnarray}

\subsection*{Limit relations}

\subsubsection*{Askey-Wilson $\rightarrow$ Big $q$-Jacobi}
The big $q$-Jacobi polynomials given by (\ref{DefBigqJacobi}) can be obtained from the
Askey-Wilson polynomials by setting $x\rightarrow \frac{1}{2}a^{-1}x$, $b=a^{-1}\alpha q$,
$c=a^{-1}\gamma q$ and $d=a\beta \gamma^{-1}$ in
$${\tilde p}_n(x;a,b,c,d|q)=\frac{a^np_n(x;a,b,c,d|q)}{(ab,ac,ad;q)_n}$$
given by (\ref{DefAskeyWilson}) and then taking the limit $a\rightarrow 0$:
$$\lim_{a\rightarrow 0}{\tilde p}_n(\textstyle\frac{1}{2}a^{-1}x;a,a^{-1}\alpha q,
a^{-1}\gamma q,a\beta \gamma^{-1}|q)=P_n(x;\alpha,\beta,\gamma;q).$$

\subsubsection*{$q$-Racah $\rightarrow$ Big $q$-Jacobi}
The big $q$-Jacobi polynomials given by (\ref{DefBigqJacobi}) can be obtained
from the $q$-Racah polynomials by setting $\delta=0$ in the definition
(\ref{DefqRacah}):
$$R_n(\mu(x);a,b,c,0|q)=P_n(q^{-x};a,b,c;q).$$

\subsubsection*{Big $q$-Jacobi $\rightarrow$ Big $q$-Laguerre}
If we set $b=0$ in the definition (\ref{DefBigqJacobi}) of
the big $q$-Jacobi polynomials we obtain the big $q$-Laguerre
polynomials given by (\ref{DefBigqLaguerre}):
\begin{equation}
P_n(x;a,0,c;q)=P_n(x;a,c;q).
\end{equation}

\subsubsection*{Big $q$-Jacobi $\rightarrow$ Little $q$-Jacobi}
The little $q$-Jacobi polynomials given by (\ref{DefLittleqJacobi}) can be obtained
from the big $q$-Jacobi polynomials by the substitution $x\rightarrow cqx$ in the definition
(\ref{DefBigqJacobi}) and then by the limit $c\rightarrow -\infty$:
\begin{equation}
\lim_{c\rightarrow -\infty}P_n(cqx;a,b,c;q)=p_n(x;a,b|q).
\end{equation}

\subsubsection*{Big $q$-Jacobi $\rightarrow$ $q$-Meixner}
If we set $b=-a^{-1}cd^{-1}$ (with $d>0$) in the definition (\ref{DefBigqJacobi})
of the big $q$-Jacobi polynomials and take the limit $c\rightarrow -\infty$ we
obtain the $q$-Meixner polynomials given by (\ref{DefqMeixner}):
\begin{equation}
\lim_{c\rightarrow -\infty}P_n(q^{-x};a,-a^{-1}cd^{-1},c;q)=M_n(q^{-x};a,d;q).
\end{equation}

\subsubsection*{Big $q$-Jacobi $\rightarrow$ Jacobi}
If we set $c=0$, $a=q^{\alpha}$ and $b=q^{\beta}$ in the definition (\ref{DefBigqJacobi})
of the big $q$-Jacobi polynomials and let $q\rightarrow 1$ we find the Jacobi
polynomials given by (\ref{DefJacobi}):
\begin{equation}
\lim_{q\rightarrow 1}P_n(x;q^{\alpha},q^{\beta},0;q)=\frac{P_n^{(\alpha,\beta)}(2x-1)}{P_n^{(\alpha,\beta)}(1)}.
\end{equation}
If we take $c=-q^{\gamma}$ for arbitrary real $\gamma$ instead of $c=0$ we
find
\begin{equation}
\lim_{q\rightarrow 1}P_n(x;q^{\alpha},q^{\beta},-q^{\gamma};q)=\frac{P_n^{(\alpha,\beta)}(x)}{P_n^{(\alpha,\beta)}(1)}.
\end{equation}

\subsection*{Remarks} 
The big $q$-Jacobi polynomials with $c=0$ and the little
$q$-Jacobi polynomials given by (\ref{DefLittleqJacobi}) are related
in the following way:
$$P_n(x;a,b,0;q)=\frac{(bq;q)_n}{(aq;q)_n}(-1)^na^nq^{n+\binom{n}{2}}p_n(a^{-1}q^{-1}x;b,a|q).$$

\noindent
Sometimes the big $q$-Jacobi polynomials are defined in terms of four
parameters instead of three. In fact the polynomials given by the definition
$$P_n(x;a,b,c,d;q)=\qhyp{3}{2}{q^{-n},abq^{n+1},ac^{-1}qx}{aq,-ac^{-1}dq}{q}$$
are orthogonal on the interval $[-d,c]$ with respect to the weight function
$$\frac{(c^{-1}qx,-d^{-1}qx;q)_{\infty}}{(ac^{-1}qx,-bd^{-1}qx;q)_{\infty}}d_qx.$$
These polynomials are not really different from those given by
(\ref{DefBigqJacobi}) since we have
$$P_n(x;a,b,c,d;q)=P_n(ac^{-1}qx;a,b,-ac^{-1}d;q)$$
and
$$P_n(x;a,b,c;q)=P_n(x;a,b,aq,-cq;q).$$

\subsection*{References}
\cite{NAlSalam89}, \cite{AlSalam90}, \cite{AndrewsAskey85}, \cite{AtakKlimyk2004},
\cite{AtakRahmanSuslov}, \cite{DattaGriffin}, \cite{FloreaniniVinetII},
\cite{GasperRahman90}, \cite{GrunbaumHaine96}, \cite{Gupta92}, \cite{Hahn},
\cite{Ismail86I}, \cite{IsmailStanton97}, \cite{IsmailWilson}, \cite{KalninsMiller88},
\cite{KoelinkE}, \cite{Koorn90II}, \cite{Koorn93}, \cite{Koorn2007}, \cite{Miller89},
\cite{Nikiforov+}, \cite{NoumiMimachi90II}, \cite{NoumiMimachi90III},
\cite{NoumiMimachi91}, \cite{Spiridonov97}, \cite{SrivastavaJain90}.


\section*{Special case}

\subsection{Big $q$-Legendre}
\index{Big q-Legendre polynomials@Big $q$-Legendre polynomials}
\index{q-Legendre polynomials@$q$-Legendre polynomials!Big}
\par

\subsection*{Basic hypergeometric representation} The big $q$-Legendre polynomials are big $q$-Jacobi polynomials
with $a=b=1$:
\begin{equation}
\label{DefBigqLegendre}
P_n(x;c;q)=\qhyp{3}{2}{q^{-n},q^{n+1},x}{q,cq}{q}.
\end{equation}

\subsection*{Orthogonality relation}
\begin{eqnarray}
\label{OrtBigqLegendre}
& &\int_{cq}^{q}P_m(x;c;q)P_n(x;c;q)\,d_qx\nonumber\\
& &{}=q(1-c)\frac{(1-q)}{(1-q^{2n+1})}
\frac{(c^{-1}q;q)_n}{(cq;q)_n}(-cq^2)^nq^{\binom{n}{2}}\,\delta_{mn},\quad c<0.
\end{eqnarray}

\subsection*{Recurrence relation}
\begin{eqnarray}
\label{RecBigqLegendre}
(x-1)P_n(x;c;q)&=&A_nP_{n+1}(x;c;q)-\left(A_n+C_n\right)P_n(x;c;q)\nonumber\\
& &{}\mathindent{}+C_nP_{n-1}(x;c;q),
\end{eqnarray}
where
$$\left\{\begin{array}{l}
\displaystyle A_n=\frac{(1-q^{n+1})(1-cq^{n+1})}{(1+q^{n+1})(1-q^{2n+1})}\\
\\
\displaystyle C_n=-cq^{n+1}\frac{(1-q^n)(1-c^{-1}q^n)}{(1+q^n)(1-q^{2n+1})}.
\end{array}\right.$$

\subsection*{Normalized recurrence relation}
\begin{equation}
\label{NormRecBigqLegendre}
xp_n(x)=p_{n+1}(x)+\left[1-(A_n+C_n)\right]p_n(x)+A_{n-1}C_np_{n-1}(x),
\end{equation}
where
$$P_n(x;c;q)=\frac{(q^{n+1};q)_n}{(q,cq;q)_n}p_n(x).$$

\subsection*{$q$-Difference equation}
\begin{eqnarray}
\label{dvBigqLegendre}
& &q^{-n}(1-q^n)(1-q^{n+1})x^2y(x)\nonumber\\
& &{}=B(x)y(qx)-\left[B(x)+D(x)\right]y(x)+D(x)y(q^{-1}x),
\end{eqnarray}
where
$$y(x)=P_n(x;c;q)$$
and
$$\left\{\begin{array}{l}\displaystyle B(x)=q(x-1)(x-c)\\
\\
\displaystyle D(x)=(x-q)(x-cq).\end{array}\right.$$

\subsection*{Rodrigues-type formula}
\begin{eqnarray}
\label{RodBigqLegendre}
P_n(x;c;q)&=&\frac{c^nq^{n(n+1)}(1-q)^n}{(q,cq;q)_n}
\left(\mathcal{D}_q\right)^n\left[(q^{-n}x,c^{-1}q^{-n}x;q)_n\right]\\
&=&\frac{(1-q)^n}{(q,cq;q)_n}\left(\mathcal{D}_q\right)^n
\left[(qx^{-1},cqx^{-1};q)_nx^{2n}\right].\nonumber
\end{eqnarray}

\subsection*{Generating functions}
\begin{equation}
\label{GenBigqLegendre1}
\qhyp{2}{1}{qx^{-1},0}{q}{xt}\,\qhyp{1}{1}{c^{-1}x}{q}{cqt}
=\sum_{n=0}^{\infty}\frac{(cq;q)_n}{(q,q;q)_n}P_n(x;c;q)t^n.
\end{equation}

\begin{equation}
\label{GenBigqLegendre2}
\qhyp{2}{1}{cqx^{-1},0}{cq}{xt}\,\qhyp{1}{1}{c^{-1}x}{c^{-1}q}{qt}
=\sum_{n=0}^{\infty}\frac{P_n(x;c;q)}{(c^{-1}q;q)_n}t^n.
\end{equation}

\subsection*{Limit relations}

\subsubsection*{Big $q$-Legendre $\rightarrow$ Legendre / Spherical}
If we set $c=0$ in the definition (\ref{DefBigqLegendre}) of the big
$q$-Legendre polynomials and let $q\rightarrow 1$ we simply obtain the Legendre
(or spherical) polynomials given by (\ref{DefLegendre}):
\begin{equation}
\lim_{q\rightarrow 1}P_n(x;0;q)=P_n(2x-1).
\end{equation}
If we take $c=-q^{\gamma}$ for arbitrary real $\gamma$ instead of $c=0$ we
find
\begin{equation}
\lim_{q\rightarrow 1}P_n(x;-q^{\gamma};q)=P_n(x).
\end{equation}

\subsection*{References}
\cite{Koelink95I}, \cite{Koorn90II}.


\section{$q$-Hahn}\index{q-Hahn polynomials@$q$-Hahn polynomials}
\par\setcounter{equation}{0}

\subsection*{Basic hypergeometric representation}
\begin{equation}
\label{DefqHahn}
Q_n(q^{-x};\alpha,\beta,N|q)=\qhyp{3}{2}{q^{-n},\alpha\beta q^{n+1},q^{-x}}
{\alpha q,q^{-N}}{q},\quad n=0,1,2,\ldots,N.
\end{equation}

\subsection*{Orthogonality relation}
For $0<\alpha q<1$ and $0<\beta q<1$, or for $\alpha>q^{-N}$ and $\beta>q^{-N}$, we have
\begin{eqnarray}
\label{OrtqHahn}
& &\sum_{x=0}^N\frac{(\alpha q,q^{-N};q)_x}{(q,\beta^{-1}q^{-N};q)_x}(\alpha\beta q)^{-x}
Q_m(q^{-x};\alpha,\beta,N|q)Q_n(q^{-x};\alpha,\beta,N|q)\nonumber\\
& &{}=\frac{(\alpha\beta q^2;q)_N}{(\beta q;q)_N(\alpha q)^N}
\frac{(q,\alpha\beta q^{N+2},\beta q;q)_n}{(\alpha q,\alpha\beta q,q^{-N};q)_n}\,
\frac{(1-\alpha\beta q)(-\alpha q)^n}{(1-\alpha\beta q^{2n+1})}
q^{\binom{n}{2}-Nn}\,\delta_{mn}.
\end{eqnarray}

\newpage

\subsection*{Recurrence relation}
\begin{eqnarray}
\label{RecqHahn}
-\left(1-q^{-x}\right)Q_n(q^{-x})&=&A_nQ_{n+1}(q^{-x})-\left(A_n+C_n\right)Q_n(q^{-x})\nonumber\\
& &{}\mathindent{}+C_nQ_{n-1}(q^{-x}),
\end{eqnarray}
where
$$Q_n(q^{-x}):=Q_n(q^{-x};\alpha,\beta,N|q)$$
and
$$\left\{\begin{array}{l}
\displaystyle A_n=\frac{(1-q^{n-N})(1-\alpha q^{n+1})(1-\alpha\beta q^{n+1})}{(1-\alpha\beta q^{2n+1})(1-\alpha\beta q^{2n+2})}\\
\\
\displaystyle C_n=-\frac{\alpha q^{n-N}(1-q^n)(1-\alpha\beta q^{n+N+1})(1-\beta q^n)}{(1-\alpha\beta q^{2n})(1-\alpha\beta q^{2n+1})}.
\end{array}\right.$$

\subsection*{Normalized recurrence relation}
\begin{equation}
\label{NormRecqHahn}
xp_n(x)=p_{n+1}(x)+\left[1-(A_n+C_n)\right]p_n(x)+A_{n-1}C_np_{n-1}(x),
\end{equation}
where
$$Q_n(q^{-x};\alpha,\beta,N|q)=
\frac{(\alpha\beta q^{n+1};q)_n}{(\alpha q,q^{-N};q)_n}p_n(q^{-x}).$$

\subsection*{$q$-Difference equation}
\begin{eqnarray}
\label{dvqHahn}
& &q^{-n}(1-q^n)(1-\alpha\beta q^{n+1})y(x)\nonumber\\
& &{}=B(x)y(x+1)-\left[B(x)+D(x)\right]y(x)+D(x)y(x-1),
\end{eqnarray}
where
$$y(x)=Q_n(q^{-x};\alpha,\beta,N|q)$$
and
$$\left\{\begin{array}{l}\displaystyle B(x)=(1-q^{x-N})(1-\alpha q^{x+1})\\
\\
\displaystyle D(x)=\alpha q(1-q^x)(\beta-q^{x-N-1}).\end{array}\right.$$

\newpage

\subsection*{Forward shift operator}
\begin{eqnarray}
\label{shift1qHahnI}
& &Q_n(q^{-x-1};\alpha,\beta,N|q)-Q_n(q^{-x};\alpha,\beta,N|q)\nonumber\\
& &{}=\frac{q^{-n-x}(1-q^n)(1-\alpha\beta q^{n+1})}{(1-\alpha q)(1-q^{-N})}
Q_{n-1}(q^{-x};\alpha q,\beta q,N-1|q)
\end{eqnarray}
or equivalently
\begin{eqnarray}
\label{shift1qHahnII}
\frac{\Delta Q_n(q^{-x};\alpha,\beta,N|q)}{\Delta q^{-x}}
&=&\frac{q^{-n+1}(1-q^n)(1-\alpha\beta q^{n+1})}{(1-q)(1-\alpha q)(1-q^{-N})}\nonumber\\
& &{}\mathindent{}\times Q_{n-1}(q^{-x};\alpha q,\beta q,N-1|q).
\end{eqnarray}

\subsection*{Backward shift operator}
\begin{eqnarray}
\label{shift2qHahnI}
& &(1-\alpha q^x)(1-q^{x-N-1})Q_n(q^{-x};\alpha,\beta,N|q)\nonumber\\
& &{}\mathindent{}-\alpha(1-q^x)(\beta-q^{x-N-1})Q_n(q^{-x+1};\alpha,\beta,N|q)\nonumber\\
& &{}=q^x(1-\alpha)(1-q^{-N-1})Q_{n+1}(q^{-x};\alpha q^{-1},\beta q^{-1},N+1|q)
\end{eqnarray}
or equivalently
\begin{eqnarray}
\label{shift2qHahnII}
& &\frac{\nabla\left[w(x;\alpha,\beta,N|q)Q_n(q^{-x};\alpha,\beta,N|q)\right]}{\nabla q^{-x}}\nonumber\\
& &{}=\frac{1}{1-q}w(x;\alpha q^{-1},\beta q^{-1},N+1|q)Q_{n+1}(q^{-x};\alpha q^{-1},\beta q^{-1},N+1|q),
\end{eqnarray}
where
$$w(x;\alpha,\beta,N|q)=\frac{(\alpha q,q^{-N};q)_x}{(q,\beta^{-1}q^{-N};q)_x}(\alpha\beta)^{-x}.$$

\subsection*{Rodrigues-type formula}
\begin{eqnarray}
\label{RodqHahn}
& &w(x;\alpha,\beta,N|q)Q_n(q^{-x};\alpha,\beta,N|q)\nonumber\\
& &{}=(1-q)^n\left(\nabla_q\right)^n\left[w(x;\alpha q^n,\beta q^n,N-n|q)\right],
\end{eqnarray}
where
$$\nabla_q:=\frac{\nabla}{\nabla q^{-x}}.$$

\subsection*{Generating functions} For $x=0,1,2,\ldots,N$ we have
\begin{eqnarray}
\label{GenqHahn1}
& &\qhyp{1}{1}{q^{-x}}{\alpha q}{\alpha qt}\,\qhyp{2}{1}{q^{x-N},0}{\beta q}{q^{-x}t}\nonumber\\
& &{}=\sum_{n=0}^N\frac{(q^{-N};q)_n}{(\beta q,q;q)_n}Q_n(q^{-x};\alpha,\beta,N|q)t^n.
\end{eqnarray}

\begin{eqnarray}
\label{GenqHahn2}
& &\qhyp{2}{1}{q^{-x},\beta q^{N+1-x}}{0}{-\alpha q^{x-N+1}t}\,
\qhyp{2}{0}{q^{x-N},\alpha q^{x+1}}{-}{-q^{-x}t}\nonumber\\
& &{}=\sum_{n=0}^N\frac{(\alpha q,q^{-N};q)_n}{(q;q)_n}q^{-\binom{n}{2}}Q_n(q^{-x};\alpha,\beta,N|q)t^n.
\end{eqnarray}

\subsection*{Limit relations}

\subsubsection*{$q$-Racah $\rightarrow$ $q$-Hahn}
The $q$-Hahn polynomials follow from the $q$-Racah polynomials by the substitution
$\delta=0$ and $\gamma q=q^{-N}$ in the definition (\ref{DefqRacah}) of the
$q$-Racah polynomials:
$$R_n(\mu(x);\alpha,\beta,q^{-N-1},0|q)=Q_n(q^{-x};\alpha,\beta,N|q).$$
Another way to obtain the $q$-Hahn polynomials from the $q$-Racah
polynomials is by setting $\gamma=0$ and $\delta=\beta^{-1}q^{-N-1}$ in the definition
(\ref{DefqRacah}):
$$R_n(\mu(x);\alpha,\beta,0,\beta^{-1}q^{-N-1}|q)=Q_n(q^{-x};\alpha,\beta,N|q).$$
And if we take $\alpha q=q^{-N}$, $\beta\rightarrow\beta\gamma q^{N+1}$ and $\delta=0$ in the
definition (\ref{DefqRacah}) of the $q$-Racah polynomials we find the
$q$-Hahn polynomials given by (\ref{DefqHahn}) in the following way:
$$R_n(\mu(x);q^{-N-1},\beta\gamma q^{N+1},\gamma,0|q)=Q_n(q^{-x};\gamma,\beta,N|q).$$
Note that $\mu(x)=q^{-x}$ in each case.

\subsubsection*{$q$-Hahn $\rightarrow$ Little $q$-Jacobi}
If we set $x\rightarrow N-x$ in the definition (\ref{DefqHahn}) of the $q$-Hahn
polynomials and take the limit $N\rightarrow\infty$ we find the little $q$-Jacobi polynomials:
\begin{equation}
\lim _{N\rightarrow\infty} Q_n(q^{x-N};\alpha,\beta,N|q)=p_n(q^x;\alpha,\beta|q),
\end{equation}
where $p_n(q^x;\alpha,\beta|q)$ is given by (\ref{DefLittleqJacobi}).

\subsubsection*{$q$-Hahn $\rightarrow$ $q$-Meixner}
The $q$-Meixner polynomials given by (\ref{DefqMeixner}) can be obtained from the $q$-Hahn polynomials
by setting $\alpha=b$ and $\beta=-b^{-1}c^{-1}q^{-N-1}$ in the definition (\ref{DefqHahn}) of the
$q$-Hahn polynomials and letting $N\rightarrow\infty$:
\begin{equation}
\lim_{N\rightarrow\infty}
Q_n(q^{-x};b,-b^{-1}c^{-1}q^{-N-1},N|q)=M_n(q^{-x};b,c;q).
\end{equation}

\subsubsection*{$q$-Hahn $\rightarrow$ Quantum $q$-Krawtchouk}
The quantum $q$-Krawtchouk polynomials given by (\ref{DefQuantumqKrawtchouk})
simply follow from the $q$-Hahn polynomials by setting $\beta=p$ in the definition (\ref{DefqHahn}) of
the $q$-Hahn polynomials and taking the limit $\alpha\rightarrow\infty$:
\begin{equation}
\lim_{\alpha\rightarrow\infty}Q_n(q^{-x};\alpha,p,N|q)=K_n^{qtm}(q^{-x};p,N;q).
\end{equation}

\subsubsection*{$q$-Hahn $\rightarrow$ $q$-Krawtchouk}
If we set $\beta=-\alpha^{-1}q^{-1}p$ in the definition (\ref{DefqHahn}) of the $q$-Hahn polynomials
and then let $\alpha\rightarrow 0$ we obtain the $q$-Krawtchouk polynomials given by
(\ref{DefqKrawtchouk}):
\begin{equation}
\lim_{\alpha\rightarrow 0}
Q_n(q^{-x};\alpha,-\alpha^{-1}q^{-1}p,N|q)=K_n(q^{-x};p,N;q).
\end{equation}

\subsubsection*{$q$-Hahn $\rightarrow$ Affine $q$-Krawtchouk}
The affine $q$-Krawtchouk polynomials given by (\ref{DefAffqKrawtchouk})
can be obtained from the $q$-Hahn polynomials by the substitution $\alpha=p$ and
$\beta=0$ in (\ref{DefqHahn}):
\begin{equation}
Q_n(q^{-x};p,0,N|q)=K_n^{Aff}(q^{-x};p,N;q).
\end{equation}

\subsubsection*{$q$-Hahn $\rightarrow$ Hahn}
The Hahn polynomials given by (\ref{DefHahn}) simply follow from the $q$-Hahn
polynomials given by (\ref{DefqHahn}), after setting $\alpha\rightarrow q^{\alpha}$
and $\beta\rightarrow q^{\beta}$, in the following way:
\begin{equation}
\lim_{q\rightarrow 1}Q_n(q^{-x};q^{\alpha},q^{\beta},N|q)=Q_n(x;\alpha,\beta,N).
\end{equation}

\subsection*{Remark}
The $q$-Hahn polynomials given by (\ref{DefqHahn}) and the dual $q$-Hahn
polynomials given by (\ref{DefDualqHahn}) are related in the following way:
$$Q_n(q^{-x};\alpha,\beta,N|q)=R_x(\mu(n);\alpha,\beta,N|q),$$
with
$$\mu(n)=q^{-n}+\alpha\beta q^{n+1}$$
or
$$R_n(\mu(x);\gamma,\delta,N|q)=Q_x(q^{-n};\gamma,\delta,N|q),$$
where
$$\mu(x)=q^{-x}+\gamma\delta q^{x+1}.$$

\subsection*{References}
\cite{AlSalam90}, \cite{AlvarezRonveaux}, \cite{AndrewsAskey85},
\cite{AskeyWilson79}, \cite{AskeyWilson85}, \cite{AtakRahmanSuslov},
\cite{Dunkl78II}, \cite{Fischer95}, \cite{GasperRahman84},
\cite{GasperRahman90}, \cite{Hahn}, \cite{KalninsMiller88},
\cite{Koelink96I}, \cite{KoelinkKoorn}, \cite{Koorn89III}, \cite{Koorn90II},
\cite{Nikiforov+}, \cite{NoumiMimachi90II}, \cite{Rahman82},
\cite{Stanton80II}, \cite{Stanton84}, \cite{Stanton90}.


\section{Dual $q$-Hahn}\index{Dual q-Hahn polynomials@Dual $q$-Hahn polynomials}
\index{q-Hahn polynomials@$q$-Hahn polynomials!Dual}
\par\setcounter{equation}{0}

\subsection*{Basic hypergeometric representation}
\begin{equation}
\label{DefDualqHahn}
R_n(\mu(x);\gamma,\delta,N|q)=
\qhyp{3}{2}{q^{-n},q^{-x},\gamma\delta q^{x+1}}{\gamma q,q^{-N}}{q},\quad n=0,1,2,\ldots,N,
\end{equation}
where
$$\mu(x):=q^{-x}+\gamma\delta q^{x+1}.$$

\newpage

\subsection*{Orthogonality relation}
For $0<\gamma q<1$ and $0<\delta q<1$, or for $\gamma>q^{-N}$ and $\delta>q^{-N}$, we have
\begin{eqnarray}
\label{OrtDualqHahn}
& &\sum_{x=0}^N\frac{(\gamma q,\gamma\delta q,q^{-N};q)_x}{(q,\gamma\delta q^{N+2},\delta q;q)_x}
\frac{(1-\gamma\delta q^{2x+1})}{(1-\gamma\delta q)(-\gamma q)^x}q^{Nx-\binom{x}{2}}\nonumber\\
& &{}\mathindent{}\times R_m(\mu(x);\gamma,\delta,N|q)R_n(\mu(x);\gamma,\delta,N|q)\nonumber\\
& &{}=\frac{(\gamma\delta q^2;q)_N}{(\delta q;q)_N}(\gamma q)^{-N}
\frac{(q,\delta^{-1}q^{-N};q)_n}{(\gamma q,q^{-N};q)_n}(\gamma\delta q)^n\,\delta_{mn}.
\end{eqnarray}

\subsection*{Recurrence relation}
\begin{eqnarray}
\label{RecDualqHahn}
& &-\left(1-q^{-x}\right)\left(1-\gamma\delta q^{x+1}\right)R_n(\mu(x))\nonumber\\
& &{}=A_nR_{n+1}(\mu(x))-\left(A_n+C_n\right)R_n(\mu(x))+C_nR_{n-1}(\mu(x)),
\end{eqnarray}
where
$$R_n(\mu(x)):=R_n(\mu(x);\gamma,\delta,N|q)$$
and
$$\left\{\begin{array}{l}
\displaystyle A_n=\left(1-q^{n-N}\right)\left(1-\gamma q^{n+1}\right)\\
\\
\displaystyle C_n=\gamma q\left(1-q^n\right)\left(\delta -q^{n-N-1}\right).
\end{array}\right.$$

\subsection*{Normalized recurrence relation}
\begin{eqnarray}
\label{NormRecDualqHahn}
xp_n(x)&=&p_{n+1}(x)+\left[1+\gamma\delta q-(A_n+C_n)\right]p_n(x)\nonumber\\
& &{}\mathindent{}+\gamma q(1-q^n)(1-\gamma q^n)\nonumber\\
& &{}\mathindent\mathindent{}\times(1-q^{n-N-1})(\delta-q^{n-N-1})p_{n-1}(x),
\end{eqnarray}
where
$$R_n(\mu(x);\gamma,\delta,N|q)=\frac{1}{(\gamma q,q^{-N};q)_n}p_n(\mu(x)).$$

\subsection*{$q$-Difference equation}
\begin{equation}
\label{dvDualqHahn}
q^{-n}(1-q^n)y(x)=B(x)y(x+1)-\left[B(x)+D(x)\right]y(x)
+D(x)y(x-1),
\end{equation}
where
$$y(x)=R_n(\mu(x);\gamma,\delta,N|q)$$
and
$$\left\{\begin{array}{l}\displaystyle B(x)=\frac{(1-q^{x-N})(1-\gamma q^{x+1})(1-\gamma\delta q^{x+1})}
{(1-\gamma\delta q^{2x+1})(1-\gamma\delta q^{2x+2})}\\
\\
\displaystyle D(x)=-\frac{\gamma q^{x-N}(1-q^x)(1-\gamma\delta q^{x+N+1})(1-\delta q^x)}
{(1-\gamma\delta q^{2x})(1-\gamma\delta q^{2x+1})}.\end{array}\right.$$

\subsection*{Forward shift operator}
\begin{eqnarray}
\label{shift1DualqHahnI}
& &R_n(\mu(x+1);\gamma,\delta,N|q)-R_n(\mu(x);\gamma,\delta,N|q)\nonumber\\
& &{}=\frac{q^{-n-x}(1-q^n)(1-\gamma\delta q^{2x+2})}{(1-\gamma q)(1-q^{-N})}
R_{n-1}(\mu(x);\gamma q,\delta,N-1|q)
\end{eqnarray}
or equivalently
\begin{eqnarray}
\label{shift1DualqHahnII}
& &\frac{\Delta R_n(\mu(x);\gamma,\delta,N|q)}{\Delta\mu(x)}\nonumber\\
& &{}=\frac{q^{-n+1}(1-q^n)}{(1-q)(1-\gamma q)(1-q^{-N})}R_{n-1}(\mu(x);\gamma q,\delta,N-1|q).
\end{eqnarray}

\subsection*{Backward shift operator}
\begin{eqnarray}
\label{shift2DualqHahnI}
& &(1-\gamma q^x)(1-\gamma\delta q^x)(1-q^{x-N-1})R_n(\mu(x);\gamma,\delta,N|q)\nonumber\\
& &{}\mathindent{}+\gamma q^{x-N-1}(1-q^x)(1-\gamma\delta q^{x+N+1})(1-\delta q^x)R_n(\mu(x-1);\gamma,\delta,N|q)\nonumber\\
& &{}=q^x(1-\gamma)(1-q^{-N-1})(1-\gamma\delta q^{2x})R_{n+1}(\mu(x);\gamma q^{-1},\delta,N+1|q)
\end{eqnarray}
or equivalently
\begin{eqnarray}
\label{shift2DualqHahnII}
& &\frac{\nabla\left[w(x;\gamma,\delta,N|q)R_n(\mu(x);\gamma,\delta,N|q)\right]}{\nabla\mu(x)}\nonumber\\
& &{}=\frac{1}{(1-q)(1-\gamma\delta)}w(x;\gamma q^{-1},\delta,N+1|q)\nonumber\\
& &{}\mathindent{}\times R_{n+1}(\mu(x);\gamma q^{-1},\delta,N+1|q),
\end{eqnarray}
where
$$w(x;\gamma,\delta,N|q)=\frac{(\gamma q,\gamma\delta q,q^{-N};q)_x}
{(q,\gamma\delta q^{N+2},\delta q;q)_x}(-\gamma^{-1})^x q^{Nx-\binom{x}{2}}.$$

\subsection*{Rodrigues-type formula}
\begin{eqnarray}
\label{RodDualqHahn}
& &w(x;\gamma,\delta,N|q)R_n(\mu(x);\gamma,\delta,N|q)\nonumber\\
& &{}=(1-q)^n(\gamma\delta q;q)_n
\left(\nabla_{\mu}\right)^n\left[w(x;\gamma q^n,\delta,N-n|q)\right],
\end{eqnarray}
where
$$\nabla_{\mu}:=\frac{\nabla}{\nabla\mu(x)}.$$

\subsection*{Generating functions} For $x=0,1,2,\ldots,N$ we have
\begin{eqnarray}
\label{GenDualqHahn1}
& &(q^{-N}t;q)_{N-x}\cdot\qhyp{2}{1}{q^{-x},\delta^{-1}q^{-x}}{\gamma q}{\gamma\delta q^{x+1}t}\nonumber\\
& &{}=\sum_{n=0}^N\frac{(q^{-N};q)_n}{(q;q)_n}R_n(\mu(x);\gamma,\delta,N|q)t^n.
\end{eqnarray}

\begin{eqnarray}
\label{GenDualqHahn2}
& &(\gamma\delta qt;q)_x\cdot\qhyp{2}{1}{q^{x-N},\gamma q^{x+1}}{\delta^{-1}q^{-N}}{q^{-x}t}\nonumber\\
& &{}=\sum_{n=0}^N
\frac{(q^{-N},\gamma q;q)_n}{(\delta^{-1}q^{-N},q;q)_n}R_n(\mu(x);\gamma,\delta,N|q)t^n.
\end{eqnarray}

\subsection*{Limit relations}

\subsubsection*{$q$-Racah $\rightarrow$ Dual $q$-Hahn}
To obtain the dual $q$-Hahn polynomials from the $q$-Racah polynomials we have to
take $\beta=0$ and $\alpha q=q^{-N}$ in (\ref{DefqRacah}):
$$R_n(\mu(x);q^{-N-1},0,\gamma,\delta|q)=R_n(\mu(x);\gamma,\delta,N|q),$$
with
$$\mu(x)=q^{-x}+\gamma\delta q^{x+1}.$$
We may also take $\alpha=0$ and $\beta=\delta^{-1}q^{-N-1}$ in (\ref{DefqRacah}) to obtain
the dual $q$-Hahn polynomials from the $q$-Racah polynomials:
$$R_n(\mu(x);0,\delta^{-1}q^{-N-1},\gamma,\delta|q)=R_n(\mu(x);\gamma,\delta,N|q).$$
And if we take $\gamma q=q^{-N}$, $\delta\rightarrow\alpha\delta q^{N+1}$ and $\beta=0$ in the
definition (\ref{DefqRacah}) of the $q$-Racah polynomials we find the dual
$q$-Hahn polynomials given by (\ref{DefDualqHahn}) in the following way:
$$R_n(\mu(x);\alpha,0,q^{-N-1},\alpha\delta q^{N+1}|q)=R_n({\tilde \mu}(x);\alpha,\delta,N|q),$$
with
$${\tilde \mu}(x)=q^{-x}+\alpha\delta q^{x+1}.$$

\subsubsection*{Dual $q$-Hahn $\rightarrow$ Affine $q$-Krawtchouk}
The affine $q$-Krawtchouk polynomials given by (\ref{DefAffqKrawtchouk})
can be obtained from the dual $q$-Hahn polynomials by the substitution $\gamma=p$
and $\delta=0$ in (\ref{DefDualqHahn}):
\begin{equation}
R_n(\mu(x);p,0,N|q)=K_n^{Aff}(q^{-x};p,N;q).
\end{equation}
Note that $\mu(x)=q^{-x}$ in this case.

\subsubsection*{Dual $q$-Hahn $\rightarrow$ Dual $q$-Krawtchouk}
The dual $q$-Krawtchouk polynomials given by (\ref{DefDualqKrawtchouk}) can
be obtained from the dual $q$-Hahn polynomials by setting $\delta=c\gamma^{-1}q^{-N-1}$
in (\ref{DefDualqHahn}) and letting $\gamma\rightarrow 0$:
\begin{equation}
\lim_{\gamma\rightarrow 0}
R_n(\mu(x);\gamma,c\gamma^{-1}q^{-N-1},N|q)=K_n(\lambda(x);c,N|q).
\end{equation}

\subsubsection*{Dual $q$-Hahn $\rightarrow$ Dual Hahn}
The dual Hahn polynomials given by (\ref{DefDualHahn}) follow from the dual $q$-Hahn
polynomials by simply taking the limit $q\rightarrow 1$ in the definition (\ref{DefDualqHahn})
of the dual $q$-Hahn polynomials after applying the substitution
$\gamma\rightarrow q^{\gamma}$ and $\delta\rightarrow q^{\delta}$:
\begin{equation}
\lim_{q\rightarrow 1}R_n(\mu(x);q^{\gamma},q^{\delta},N|q)=R_n(\lambda(x);\gamma,\delta,N),
\end{equation}
where
$$\left\{\begin{array}{l}
\displaystyle\mu(x)=q^{-x}+q^{x+\gamma+\delta+1}\\[5mm]
\displaystyle\lambda(x)=x(x+\gamma+\delta+1).
\end{array}\right.$$

\subsection*{Remark}
The dual $q$-Hahn polynomials given by (\ref{DefDualqHahn}) and the
$q$-Hahn polynomials given by (\ref{DefqHahn}) are related in the following way:
$$Q_n(q^{-x};\alpha,\beta,N|q)=R_x(\mu(n);\alpha,\beta,N|q),$$
with
$$\mu(n)=q^{-n}+\alpha\beta q^{n+1}$$
or
$$R_n(\mu(x);\gamma,\delta,N|q)=Q_x(q^{-n};\gamma,\delta,N|q),$$
where
$$\mu(x)=q^{-x}+\gamma\delta q^{x+1}.$$

\subsection*{References}
\cite{AlvarezSmirnov}, \cite{AndrewsAskey85}, \cite{AskeyWilson79},
\cite{AskeyWilson85}, \cite{AtakRahmanSuslov}, \cite{GasperRahman90},
\cite{KoelinkKoorn}, \cite{Nikiforov+}, \cite{Stanton84}.


\section{Al-Salam-Chihara}\index{Al-Salam-Chihara polynomials}
\par\setcounter{equation}{0}

\subsection*{Basic hypergeometric representation}
\begin{eqnarray}
\label{DefAlSalamChihara}
Q_n(x;a,b|q)&=&\frac{(ab;q)_n}{a^n}\,
\qhyp{3}{2}{q^{-n},a\e^{i\theta},a\e^{-i\theta}}{ab,0}{q}\\
&=&(a\e^{i\theta};q)_n\e^{-in\theta}\,\qhyp{2}{1}{q^{-n},b\e^{-i\theta}}
{a^{-1}q^{-n+1}\e^{-i\theta}}{a^{-1}q\e^{i\theta}}\nonumber\\
&=&(b\e^{-i\theta};q)_n\e^{in\theta}\,\qhyp{2}{1}{q^{-n},a\e^{i\theta}}
{b^{-1}q^{-n+1}\e^{i\theta}}{b^{-1}q\e^{-i\theta}},\quad x=\cos\theta.\nonumber
\end{eqnarray}

\subsection*{Orthogonality relation}
If $a$ and $b$ are real or complex conjugates and $\max(|a|,|b|)<1$, then we have the following orthogonality relation
\begin{equation}
\label{OrtAlSalamChiharaI}
\frac{1}{2\pi}\int_{-1}^1\frac{w(x)}{\sqrt{1-x^2}}Q_m(x;a,b|q)Q_n(x;a,b|q)\,dx
=\frac{\,\delta_{mn}}{(q^{n+1},abq^n;q)_{\infty}},
\end{equation}
where
$$w(x):=w(x;a,b|q)=\left|\frac{(\e^{2i\theta};q)_{\infty}}
{(a\e^{i\theta},b\e^{i\theta};q)_{\infty}}\right|^2
=\frac{h(x,1)h(x,-1)h(x,q^{\frac{1}{2}})h(x,-q^{\frac{1}{2}})}{h(x,a)h(x,b)},$$
with
$$h(x,\alpha):=\prod_{k=0}^{\infty}\left(1-2\alpha xq^k+\alpha^2q^{2k}\right)
=\left(\alpha\e^{i\theta},\alpha\e^{-i\theta};q\right)_{\infty},\quad x=\cos\theta.$$
If $a>1$ and $|ab|<1$, then we have another orthogonality relation given by:
\begin{eqnarray}
\label{OrtAlSalamChiharaII}
& &\frac{1}{2\pi}\int_{-1}^1\frac{w(x)}{\sqrt{1-x^2}}Q_m(x;a,b|q)Q_n(x;a,b|q)\,dx\nonumber\\
& &{}\mathindent{}+\sum_{\begin{array}{c}\scriptstyle k\\ \scriptstyle 1<aq^k\leq a\end{array}}
w_kQ_m(x_k;a,b|q)Q_n(x_k;a,b|q)=\frac{\,\delta_{mn}}{(q^{n+1},abq^n;q)_{\infty}},
\end{eqnarray}
where $w(x)$ is as before,
$$x_k=\frac{aq^k+\left(aq^k\right)^{-1}}{2}$$
and
$$w_k=\frac{(a^{-2};q)_{\infty}}{(q,ab,a^{-1}b;q)_{\infty}}
\frac{(1-a^2q^{2k})(a^2,ab;q)_k}{(1-a^2)(q,ab^{-1}q;q)_k}
q^{-k^2}\left(\frac{1}{a^3b}\right)^k.$$

\subsection*{Recurrence relation}
\begin{equation}
\label{RecAlSalamChihara}
2xQ_n(x)=Q_{n+1}(x)+(a+b)q^nQ_n(x)+(1-q^n)(1-abq^{n-1})Q_{n-1}(x).
\end{equation}
%where
%$$Q_n(x):=\AlSalamChihara{n}@{x}{a}{b}{q}$$

\subsection*{Normalized recurrence relation}
\begin{equation}
\label{NormRecAlSalamChihara}
xp_n(x)=p_{n+1}(x)+\frac{1}{2}(a+b)q^np_n(x)+\frac{1}{4}(1-q^n)(1-abq^{n-1})p_{n-1}(x),
\end{equation}
where
$$Q_n(x;a,b|q)=2^np_n(x).$$

\newpage

\subsection*{$q$-Difference equation}
\begin{eqnarray}
\label{dvAlSalamChihara1}
& &(1-q)^2D_q\left[{\tilde w}(x;aq^{\frac{1}{2}},bq^{\frac{1}{2}}|q)D_qy(x)\right]\nonumber\\
& &{}\mathindent{}+4q^{-n+1}(1-q^n){\tilde w}(x;a,b|q)y(x)=0,
\end{eqnarray}
where
$$y(x)=Q_n(x;a,b|q)$$
and
$${\tilde w}(x;a,b|q):=\frac{w(x;a,b|q)}{\sqrt{1-x^2}}.$$
If we define
$$P_n(z):=\frac{(ab;q)_n}{a^n}\,\qhyp{3}{2}{q^{-n},az,az^{-1}}{ab,0}{q}$$
then the $q$-difference equation can also be written in the form
\begin{equation}
\label{dvAlSalamChihara2}
q^{-n}(1-q^n)P_n(z)=A(z)P_n(qz)-\left[A(z)+A(z^{-1})\right]P_n(z)
+A(z^{-1})P_n(q^{-1}z),
\end{equation}
where
$$A(z)=\frac{(1-az)(1-bz)}{(1-z^2)(1-qz^2)}.$$

\subsection*{Forward shift operator}
\begin{eqnarray}
\label{shift1AlSalamChiharaI}
& &\delta_qQ_n(x;a,b|q)\nonumber\\
& &{}=-q^{-\frac{1}{2}n}(1-q^n)(\e^{i\theta}-\e^{-i\theta})
Q_{n-1}(x;aq^{\frac{1}{2}},bq^{\frac{1}{2}}|q),\quad x=\cos\theta
\end{eqnarray}
or equivalently
\begin{equation}
\label{shift1AlSalamChiharaII}
D_qQ_n(x;a,b|q)=2q^{-\frac{1}{2}(n-1)}
\frac{1-q^n}{1-q}Q_{n-1}(x;aq^{\frac{1}{2}},bq^{\frac{1}{2}}|q).
\end{equation}

\subsection*{Backward shift operator}
\begin{eqnarray}
\label{shift2AlSalamChiharaI}
& &\delta_q\left[{\tilde w}(x;a,b|q)Q_n(x;a,b|q)\right]\nonumber\\
& &{}=q^{-\frac{1}{2}(n+1)}(\e^{i\theta}-\e^{-i\theta})
{\tilde w}(x;aq^{-\frac{1}{2}},bq^{-\frac{1}{2}}|q)\nonumber\\
& &{}\mathindent{}\times Q_{n+1}(x;aq^{-\frac{1}{2}},bq^{-\frac{1}{2}}|q),
\quad x=\cos\theta
\end{eqnarray}
or equivalently
\begin{eqnarray}
\label{shift2AlSalamChiharaII}
& &D_q\left[{\tilde w}(x;a,b|q)Q_n(x;a,b|q)\right]\nonumber\\
& &{}=-\frac{2q^{-\frac{1}{2}n}}{1-q}{\tilde w}(x;aq^{-\frac{1}{2}},bq^{-\frac{1}{2}}|q)
Q_{n+1}(x;aq^{-\frac{1}{2}},bq^{-\frac{1}{2}}|q).
\end{eqnarray}

\subsection*{Rodrigues-type formula}
\begin{eqnarray}
\label{RodAlSalamChihara}
& &{\tilde w}(x;a,b|q)Q_n(x;a,b|q)\nonumber\\
& &{}=\left(\frac{q-1}{2}\right)^nq^{\frac{1}{4}n(n-1)}
\left(D_q\right)^n\left[{\tilde w}(x;aq^{\frac{1}{2}n},bq^{\frac{1}{2}n}|q)\right].
\end{eqnarray}

\subsection*{Generating functions}
\begin{equation}
\label{GenAlSalamChihara1}
\frac{(at,bt;q)_{\infty}}{(\e^{i\theta}t,\e^{-i\theta}t;q)_{\infty}}
=\sum_{n=0}^{\infty}\frac{Q_n(x;a,b|q)}{(q;q)_n}t^n,\quad x=\cos\theta.
\end{equation}

\begin{equation}
\label{GenAlSalamChihara2}
\frac{1}{(\e^{i\theta}t;q)_{\infty}}\,
\qhyp{2}{1}{a\e^{i\theta},b\e^{i\theta}}{ab}{\e^{-i\theta}t}
=\sum_{n=0}^{\infty}\frac{Q_n(x;a,b|q)}{(ab,q;q)_n}t^n,\quad x=\cos\theta.
\end{equation}

\begin{eqnarray}
\label{GenAlSalamChihara3}
& &(t;q)_{\infty}\cdot\qhyp{2}{1}{a\e^{i\theta},a\e^{-i\theta}}{ab}{t}\nonumber\\
& &{}=\sum_{n=0}^{\infty}\frac{(-1)^na^nq^{\binom{n}{2}}}{(ab,q;q)_n}Q_n(x;a,b|q)t^n,
\quad x=\cos\theta.
\end{eqnarray}

\begin{eqnarray}
\label{GenAlSalamChihara4}
& &\frac{(\gamma \e^{i\theta}t;q)_{\infty}}{(\e^{i\theta}t;q)_{\infty}}\,
\qhyp{3}{2}{\gamma,a\e^{i\theta},b\e^{i\theta}}{ab,\gamma \e^{i\theta}t}{\e^{-i\theta}t}\nonumber\\
& &{}=\sum_{n=0}^{\infty}\frac{(\gamma;q)_n}{(ab,q;q)_n}Q_n(x;a,b|q)t^n,
\quad x=\cos\theta,\quad\textrm{$\gamma$ arbitrary}.
\end{eqnarray}

\subsection*{Limit relations}

\subsubsection*{Continuous dual $q$-Hahn $\rightarrow$ Al-Salam-Chihara}
The Al-Salam-Chihara polynomials given by (\ref{DefAlSalamChihara}) simply follow
from the continuous dual $q$-Hahn polynomials by taking $c=0$ in the
definition (\ref{DefContDualqHahn}) of the continuous dual $q$-Hahn polynomials:
$$p_n(x;a,b,0|q)=Q_n(x;a,b|q).$$

\subsubsection*{Al-Salam-Chihara $\rightarrow$ Continuous big $q$-Hermite}
If we take $b=0$ in the definition (\ref{DefAlSalamChihara}) of the Al-Salam-Chihara
polynomials we simply obtain the continuous big $q$-Hermite polynomials given by
(\ref{DefContBigqHermite}):
\begin{equation}
Q_n(x;a,0|q)=H_n(x;a|q).
\end{equation}

\subsubsection*{Al-Salam-Chihara $\rightarrow$ Continuous $q$-Laguerre}
The continuous $q$-Laguerre polynomials given by (\ref{DefContqLaguerre})
can be obtained from the Al-Salam-Chihara polynomials given by
(\ref{DefAlSalamChihara}) by taking $a=q^{\frac{1}{2}\alpha+\frac{1}{4}}$ and
$b=q^{\frac{1}{2}\alpha+\frac{3}{4}}$:
\begin{equation}
Q_n(x;q^{\frac{1}{2}\alpha+\frac{1}{4}},q^{\frac{1}{2}\alpha+\frac{3}{4}}|q)
=\frac{(q;q)_n}{q^{(\frac{1}{2}\alpha+\frac{1}{4})n}}P_n^{(\alpha)}(x|q).
\end{equation}

\subsubsection*{Al-Salam-Chihara $\rightarrow$ Meixner-Pollaczek}
If we set $a=q^{\lambda}\e^{-i\phi}$, $b=q^{\lambda}\e^{i\phi}$ and
$\e^{i\theta}=q^{ix}\e^{i\phi}$ in the definition (\ref{DefAlSalamChihara}) of
the Al-Salam-Chihara polynomials and take the limit $q\rightarrow 1$ we obtain
the Meixner-Pollaczek polynomials given by (\ref{DefMP}) in the following
way:
\begin{equation}
\lim_{q\rightarrow 1}\frac{Q_n(\cos(\ln q^x+\phi);
q^{\lambda}\e^{i\phi},q^{\lambda}\e^{-i\phi}|q)}{(q;q)_n}=P_n^{(\lambda)}(x;\phi).
\end{equation}

\subsection*{References}
\cite{AlSalam90}, \cite{AlSalamChihara76}, \cite{AlSalamChihara87}, \cite{AskeyIsmail84},
\cite{AskeyRahmanSuslov}, \cite{AtakAtakIII}, \cite{Bryc}, \cite{ChiharaIsmail},
\cite{ChrisIsmail}, \cite{Dehesa}, \cite{Floreanini+97}, \cite{IsmailRahmanSuslov},
\cite{IsmailStanton97}, \cite{KoelinkE}, \cite{Koelink96III}.


\section{$q$-Meixner-Pollaczek}
\index{q-Meixner-Pollaczek polynomials@$q$-Meixner-Pollaczek polynomials}
\par\setcounter{equation}{0}

\subsection*{Basic hypergeometric representation}
\begin{eqnarray}
\label{DefqMP}
P_n(x;a|q)&=&a^{-n}\e^{-in\phi}\frac{(a^2;q)_n}{(q;q)_n}\,
\qhyp{3}{2}{q^{-n},a\e^{i(\theta+2\phi)},a\e^{-i\theta}}{a^2,0}{q}\\
&=&\frac{(a\e^{-i\theta};q)_n}{(q;q)_n}\e^{in(\theta+\phi)}\,
\qhyp{2}{1}{q^{-n},a\e^{i\theta}}{a^{-1}q^{-n+1}\e^{i\theta}}
{qa^{-1}\e^{-i(\theta+2\phi)}}\nonumber
\end{eqnarray}
with $x=\cos(\theta+\phi)$.

\subsection*{Orthogonality relation}
\begin{eqnarray}
\label{OrtqMP}
& &\frac{1}{2\pi}\int_{-\pi}^{\pi}w(\cos(\theta+\phi);a|q)
P_m(\cos(\theta+\phi);a|q)P_n(\cos(\theta+\phi);a|q)\,d\theta\nonumber\\
& &{}=\frac{\,\delta_{mn}}{(q;q)_n(q,a^2q^n;q)_{\infty}},\quad 0<a<1,
\end{eqnarray}
where
$$w(x;a|q)=\left|\frac{(\e^{2i(\theta+\phi)};q)_{\infty}}
{(a\e^{i(\theta+2\phi)},a\e^{i\theta};q)_{\infty}}\right|^2=
\frac{h(x,1)h(x,-1)h(x,q^{\frac{1}{2}})h(x,-q^{\frac{1}{2}})}
{h(x,a\e^{i\phi})h(x,a\e^{-i\phi})},$$
with
$$h(x,\alpha):=\prod_{k=0}^{\infty}\left(1-2\alpha xq^k+\alpha^2q^{2k}\right)
=\left(\alpha\e^{i(\theta+\phi)},\alpha\e^{-i(\theta+\phi)};q\right)_{\infty},
\quad x=\cos(\theta+\phi).$$

\subsection*{Recurrence relation}
\begin{eqnarray}
\label{RecqMP}
2xP_n(x;a|q)&=&(1-q^{n+1})P_{n+1}(x;a|q)+2aq^n\cos\phi P_n(x;a|q)\nonumber\\
& &{}\mathindent{}+(1-a^2q^{n-1})P_{n-1}(x;a|q).
\end{eqnarray}

\newpage

\subsection*{Normalized recurrence relation}
\begin{equation}
\label{NormRecqMP}
xp_n(x)=p_{n+1}(x)+aq^n\cos\phi\,p_n(x)+\frac{1}{4}(1-q^n)(1-a^2q^{n-1})p_{n-1}(x),
\end{equation}
where
$$P_n(x;a|q)=\frac{2^n}{(q;q)_n}p_n(x).$$

\subsection*{$q$-Difference equation}
\begin{equation}
\label{dvqMP}
(1-q)^2D_q\left[{\tilde w}(x;aq^{\frac{1}{2}}|q)D_qy(x)\right]
+4q^{-n+1}(1-q^n){\tilde w}(x;a|q)y(x)=0,
\end{equation}
where
$$y(x)=P_n(x;a|q)$$
and
$${\tilde w}(x;a|q):=\frac{w(x;a|q)}{\sqrt{1-x^2}}.$$

\subsection*{Forward shift operator}
\begin{equation}
\label{shift1qMPI}
\delta_qP_n(x;a|q)=-q^{-\frac{1}{2}n}(\e^{i(\theta+\phi)}-\e^{-i(\theta+\phi)})
P_{n-1}(x;aq^{\frac{1}{2}}|q),\quad x=\cos(\theta+\phi)
\end{equation}
or equivalently
\begin{equation}
\label{shift1qMPII}
D_qP_n(x;a|q)=\frac{2q^{-\frac{1}{2}(n-1)}}{1-q}P_{n-1}(x;aq^{\frac{1}{2}}|q).
\end{equation}

\subsection*{Backward shift operator}
\begin{eqnarray}
\label{shift2qMPI}
& &\delta_q\left[{\tilde w}(x;a|q)P_n(x;a|q)\right]\nonumber\\
& &{}=q^{-\frac{1}{2}(n+1)}(1-q^{n+1})(\e^{i\theta}-\e^{-i\theta})\nonumber\\
& &{}\mathindent{}\times
{\tilde w}(x;aq^{-\frac{1}{2}}|q)P_{n+1}(x;aq^{-\frac{1}{2}}|q),\quad x=\cos(\theta+\phi)
\end{eqnarray}
or equivalently
\begin{equation}
\label{shift2qMPII}
D_q\left[{\tilde w}(x;a|q)P_n(x;a|q)\right]=
-2q^{-\frac{1}{2}n}\frac{1-q^{n+1}}{1-q}{\tilde w}(x;aq^{-\frac{1}{2}}|q)P_{n+1}(x;aq^{-\frac{1}{2}}|q).
\end{equation}

\subsection*{Rodrigues-type formula}
\begin{equation}
\label{RodqMP}
{\tilde w}(x;a|q)P_n(x;a|q)=\left(\frac{q-1}{2}\right)^n
q^{\frac{1}{4}n(n-1)}\frac{1}{(q;q)_n}\left(D_q\right)^n\left[{\tilde w}(x;aq^{\frac{1}{2}n}|q)\right].
\end{equation}

\subsection*{Generating functions}
\begin{eqnarray}
\label{GenqMP1}
& &\left|\frac{(a\e^{i\phi}t;q)_{\infty}}{(\e^{i(\theta+\phi)}t;q)_{\infty}}\right|^2\nonumber\\
& &{}=\frac{(a\e^{i\phi}t,a\e^{-i\phi}t;q)_{\infty}}{(\e^{i(\theta+\phi)}t,\e^{-i(\theta+\phi)}t;q)_{\infty}}
=\sum_{n=0}^{\infty}P_n(x;a|q)t^n,\quad x=\cos(\theta+\phi).
\end{eqnarray}

\begin{eqnarray}
\label{GenqMP2}
& &\frac{1}{(\e^{i(\theta+\phi)}t;q)_{\infty}}\,
\qhyp{2}{1}{a\e^{i(\theta+2\phi)},a\e^{i\theta}}{a^2}{\e^{-i(\theta+\phi)}t}\nonumber\\
& &{}=\sum_{n=0}^{\infty}\frac{P_n(x;a|q)}{(a^2;q)_n}t^n,\quad x=\cos(\theta+\phi).
\end{eqnarray}

\subsection*{Limit relations}

\subsubsection*{Continuous $q$-Hahn $\rightarrow$ $q$-Meixner-Pollaczek}
The $q$-Meixner-Pollaczek polynomials given by (\ref{DefqMP}) simply follow
from the continuous $q$-Hahn polynomials if we set $d=a$ and $b=c=0$ in the
definition (\ref{DefContqHahn}) of the continuous $q$-Hahn polynomials:
$$\frac{p_n(\cos(\theta+\phi);a,0,0,a;q)}{(q;q)_n}=P_n(\cos(\theta+\phi);a|q).$$

\subsubsection*{$q$-Meixner-Pollaczek $\rightarrow$ Continuous $q$-ultraspherical /
Rogers}
If we take $\theta=0$ and $a=\beta$ in the definition (\ref{DefqMP}) of the
$q$-Meixner-Pollaczek polynomials we obtain the continuous
$q$-ultraspherical (or Rogers) polynomials given by (\ref{DefContqUltra}):
\begin{equation}
P_n(\cos\phi;\beta|q)=C_n(\cos\phi;\beta|q).
\end{equation}

\subsubsection*{$q$-Meixner-Pollaczek $\rightarrow$ Continuous
$q$-Laguerre}
If we take $\e^{i\phi}=q^{-\frac{1}{4}}$, $a=q^{\frac{1}{2}\alpha+\frac{1}{2}}$
and $\e^{i\theta}\rightarrow q^{\frac{1}{4}}\e^{i\theta}$ in the definition
(\ref{DefqMP}) of the $q$-Meixner-Pollaczek polynomials we obtain the
continuous $q$-Laguerre polynomials given by (\ref{DefContqLaguerre}):
\begin{equation}
P_n(\cos(\theta+\phi);q^{\frac{1}{2}\alpha+\frac{1}{2}}|q)=
q^{-(\frac{1}{2}\alpha+\frac{1}{4})n}P_n^{(\alpha)}(\cos\theta|q).
\end{equation}

\subsubsection*{$q$-Meixner-Pollaczek $\rightarrow$ Meixner-Pollaczek}
To find the Meixner-Pollaczek polynomials given by (\ref{DefMP}) from the $q$-Meixner-Pollaczek
polynomials we substitute $a=q^{\lambda}$ and $\e^{i\theta}=q^{-ix}$(or $\theta=\ln q^{-x}$) in the definition
(\ref{DefqMP}) of the $q$-Meixner-Pollaczek polynomials and take the limit $q\rightarrow 1$ to find:
\begin{equation}
\lim_{q\rightarrow 1}P_n(\cos(\ln q^{-x}+\phi);q^{\lambda}|q)
=P_n^{(\lambda)}(x;-\phi).
\end{equation}

\subsection*{References}
\cite{AlSalam90}, \cite{AlSalamChihara87}, \cite{AskeyIsmail84},
\cite{AskeyWilson85}, \cite{CharrisIsmail}, \cite{Ismail85II}.


\section{Continuous $q$-Jacobi}
\index{Continuous q-Jacobi polynomials@Continuous $q$-Jacobi polynomials}
\index{q-Jacobi polynomials@$q$-Jacobi polynomials!Continuous}
\par\setcounter{equation}{0}

\subsection*{Basic hypergeometric representation} If we take $a=q^{\frac{1}{2}\alpha+\frac{1}{4}}$, $b=q^{\frac{1}{2}\alpha+\frac{3}{4}}$,
$c=-q^{\frac{1}{2}\beta+\frac{1}{4}}$ and $d=-q^{\frac{1}{2}\beta+\frac{3}{4}}$ in
the definition (\ref{DefAskeyWilson}) of the Askey-Wilson polynomials we find
after renormalizing
\begin{eqnarray}
\label{DefContqJacobi}
& &P_n^{(\alpha,\beta)}(x|q)\nonumber\\
& &{}=\frac{(q^{\alpha+1};q)_n}{(q;q)_n}\,
\qhyp{4}{3}{q^{-n},q^{n+\alpha+\beta+1},q^{\frac{1}{2}\alpha+\frac{1}{4}}\e^{i\theta},q^{\frac{1}{2}\alpha+\frac{1}{4}}\e^{-i\theta}}
{q^{\alpha+1},-q^{\frac{1}{2}(\alpha+\beta+1)},-q^{\frac{1}{2}(\alpha+\beta+2)}}{q}
\end{eqnarray}
with $x=\cos\theta$.

\newpage

\subsection*{Orthogonality relation}
For $\alpha\geq -\frac{1}{2}$ and $\beta\geq -\frac{1}{2}$ we have
\begin{eqnarray}
\label{OrtContqJacobi}
& &\frac{1}{2\pi}\int_{-1}^1\frac{w(x)}{\sqrt{1-x^2}}
P_m^{(\alpha,\beta)}(x|q)P_n^{(\alpha,\beta)}(x|q)\,dx\nonumber\\
& &{}=\frac{(q^{\frac{1}{2}(\alpha+\beta+2)},q^{\frac{1}{2}(\alpha+\beta+3)};q)_{\infty}}{(q,q^{\alpha+1},q^{\beta+1},-q^{\frac{1}{2}(\alpha+\beta+1)},
-q^{\frac{1}{2}(\alpha+\beta+2)};q)_{\infty}}\,\frac{1-q^{\alpha+\beta+1}}{1-q^{2n+\alpha+\beta+1}}\nonumber\\
& &{}\mathindent{}\times\frac{(q^{\alpha+1},q^{\beta+1},-q^{\frac{1}{2}(\alpha+\beta+3)};q)_n}
{(q,q^{\alpha+\beta+1},-q^{\frac{1}{2}(\alpha+\beta+1)};q)_n}q^{(\alpha+\frac{1}{2})n}\,\delta_{mn},
\end{eqnarray}
where
\begin{eqnarray*} w(x):=w(x;q^{\alpha},q^{\beta}|q)&=&\left|\frac{(\e^{2i\theta};q)_{\infty}}
{(q^{\frac{1}{2}\alpha+\frac{1}{4}}\e^{i\theta},q^{\frac{1}{2}\alpha+\frac{3}{4}}\e^{i\theta},
-q^{\frac{1}{2}\beta+\frac{1}{4}}\e^{i\theta},-q^{\frac{1}{2}\beta+\frac{3}{4}}\e^{i\theta};q)_{\infty}}\right|^2\\
&=&\left|\frac{(\e^{i\theta},-\e^{i\theta};q^{\frac{1}{2}})_{\infty}}{(q^{\frac{1}{2}\alpha+\frac{1}{4}}\e^{i\theta},
-q^{\frac{1}{2}\beta+\frac{1}{4}}\e^{i\theta};q^{\frac{1}{2}})_{\infty}}\right|^2\\
&=&\frac{h(x,1)h(x,-1)h(x,q^{\frac{1}{2}})h(x,-q^{\frac{1}{2}})}
{h(x,q^{\frac{1}{2}\alpha+\frac{1}{4}})h(x,q^{\frac{1}{2}\alpha+\frac{3}{4}})
h(x,-q^{\frac{1}{2}\beta+\frac{1}{4}})h(x,-q^{\frac{1}{2}\beta+\frac{3}{4}})},
\end{eqnarray*}
with
$$h(x,\alpha):=\prod_{k=0}^{\infty}\left(1-2\alpha xq^k+\alpha^2q^{2k}\right)
=\left(\alpha \e^{i\theta},\alpha \e^{-i\theta};q\right)_{\infty},\quad x=\cos\theta.$$

\subsection*{Recurrence relation}
\begin{eqnarray}
\label{RecContqJacobi}
2x{\tilde P}_n(x|q)&=&A_n{\tilde P}_{n+1}(x|q)+\left[q^{\frac{1}{2}\alpha+\frac{1}{4}}+
q^{-\frac{1}{2}\alpha-\frac{1}{4}}-\left(A_n+C_n\right)\right]{\tilde P}_n(x|q)\nonumber\\
& &{}\mathindent{}+C_n{\tilde P}_{n-1}(x|q),
\end{eqnarray}
where
$$
{\tilde P}_n(x|q):={\tilde P}_n^{(\alpha,\beta)}(x|q)=\frac{(q;q)_n}{(q^{\alpha+1};q)_n}P_n^{(\alpha,\beta)}(x|q)$$
and
$$\left\{\begin{array}{l}
\displaystyle A_n=\frac{(1-q^{n+\alpha+1})(1-q^{n+\alpha+\beta+1})(1+q^{n+\frac{1}{2}(\alpha+\beta+1)})(1+q^{n+\frac{1}{2}(\alpha+\beta+2)})}
{q^{\frac{1}{2}\alpha+\frac{1}{4}}(1-q^{2n+\alpha+\beta+1})(1-q^{2n+\alpha+\beta+2})}\\
\\
\displaystyle C_n=\frac{q^{\frac{1}{2}\alpha+\frac{1}{4}}(1-q^n)(1-q^{n+\beta})(1+q^{n+\frac{1}{2}(\alpha+\beta)})(1+q^{n+\frac{1}{2}(\alpha+\beta+1)})}
{(1-q^{2n+\alpha+\beta})(1-q^{2n+\alpha+\beta+1})}.
\end{array}\right.$$

\subsection*{Normalized recurrence relation}
\begin{eqnarray}
\label{NormRecContqJacobi}
xp_n(x)&=&p_{n+1}(x)+\frac{1}{2}\left[q^{\frac{1}{2}\alpha+\frac{1}{4}}+
q^{-\frac{1}{2}\alpha-\frac{1}{4}}-(A_n+C_n)\right]p_n(x)\nonumber\\
& &{}\mathindent{}+\frac{1}{4}A_{n-1}C_np_{n-1}(x),
\end{eqnarray}
where
$$P_n^{(\alpha,\beta)}(x|q)=\frac{2^nq^{(\frac{1}{2}\alpha+\frac{1}{4})n}(q^{n+\alpha+\beta+1};q)_n}
{(q,-q^{\frac{1}{2}(\alpha+\beta+1)},-q^{\frac{1}{2}(\alpha+\beta+2)};q)_n}p_n(x).$$

\subsection*{$q$-Difference equation}
\begin{equation}
\label{dvContqJacobi}
(1-q)^2D_q\left[{\tilde w}(x;q^{\alpha+1},q^{\beta+1}|q)D_qy(x)\right]+
\lambda_n{\tilde w}(x;q^{\alpha},q^{\beta}|q)y(x)=0,
\end{equation}
where
$$y(x)=P_n^{(\alpha,\beta)}(x|q)$$
and
$${\tilde w}(x;q^{\alpha},q^{\beta}|q):=\frac{w(x;q^{\alpha},q^{\beta}|q)}{\sqrt{1-x^2}},$$
$$\lambda_n=4q^{-n+1}(1-q^n)(1-q^{n+\alpha+\beta+1}).$$

\subsection*{Forward shift operator}
\begin{eqnarray}
\label{shift1ContqJacobiI}
\delta_qP_n^{(\alpha,\beta)}(x|q)&=&-\frac{q^{-n+\frac{1}{2}\alpha+\frac{3}{4}}
(1-q^{n+\alpha+\beta+1})(\e^{i\theta}-\e^{-i\theta})}{(1+q^{\frac{1}{2}(\alpha+\beta+1)})
(1+q^{\frac{1}{2}(\alpha+\beta+2)})}\nonumber\\
& &{}\mathindent{}\times P_{n-1}^{(\alpha+1,\beta+1)}(x|q),\quad x=\cos\theta
\end{eqnarray}
or equivalently
\begin{eqnarray}
\label{shift1ContqJacobiII}
D_qP_n^{(\alpha,\beta)}(x|q)&=&\frac{2q^{-n+\frac{1}{2}\alpha+\frac{5}{4}}
(1-q^{n+\alpha+\beta+1})}{(1-q)(1+q^{\frac{1}{2}(\alpha+\beta+1)})
(1+q^{\frac{1}{2}(\alpha+\beta+2)})}\nonumber\\
& &{}\mathindent{}\times P_{n-1}^{(\alpha+1,\beta+1)}(x|q).
\end{eqnarray}

\subsection*{Backward shift operator}
\begin{eqnarray}
\label{shift2ContqJacobiI}
& &\delta_q\left[{\tilde w}(x;q^{\alpha},q^{\beta}|q)P_n^{(\alpha,\beta)}(x|q)\right]\nonumber\\
& &{}=q^{-\frac{1}{2}\alpha-\frac{1}{4}}(1-q^{n+1})(1+q^{\frac{1}{2}(\alpha+\beta-1)})
(1+q^{\frac{1}{2}(\alpha+\beta)})(\e^{i\theta}-\e^{-i\theta})\nonumber\\
& &{}\mathindent{}\times{\tilde w}(x;q^{\alpha-1},q^{\beta-1}|q)
P_{n+1}^{(\alpha-1,\beta-1)}(x|q),\quad x=\cos\theta
\end{eqnarray}
or equivalently
\begin{eqnarray}
\label{shift2ContqJacobiII}
& &D_q\left[{\tilde w}(x;q^{\alpha},q^{\beta}|q)P_n^{(\alpha,\beta)}(x|q)\right]\nonumber\\
& &{}=-2q^{-\frac{1}{2}\alpha+\frac{1}{4}}
\frac{(1-q^{n+1})(1+q^{\frac{1}{2}(\alpha+\beta-1)})(1+q^{\frac{1}{2}(\alpha+\beta)})}{1-q}\nonumber\\
& &{}\mathindent{}\times{\tilde w}(x;q^{\alpha-1},q^{\beta-1}|q)P_{n+1}^{(\alpha-1,\beta-1)}(x|q).
\end{eqnarray}

\subsection*{Rodrigues-type formula}
\begin{eqnarray}
\label{RodContqJacobi}
& &{\tilde w}(x;q^\alpha,q^\beta|q)P_n^{(\alpha,\beta)}(x|q)\nonumber\\
& &{}=\left(\frac{q-1}{2}\right)^n\frac{q^{\frac{1}{4}n^2+\frac{1}{2}n\alpha}}
{(q,-q^{\frac{1}{2}(\alpha+\beta+1)},-q^{\frac{1}{2}(\alpha+\beta+2)};q)_n}\nonumber\\
& &{}\mathindent{}\times\left(D_q\right)^n\left[{\tilde w}(x;q^{\alpha+n},q^{\beta+n}|q)\right].
\end{eqnarray}

\subsection*{Generating functions}
\begin{eqnarray}
\label{GenContqJacobi1}
& &\qhyp{2}{1}{q^{\frac{1}{2}\alpha+\frac{1}{4}}\e^{i\theta},
q^{\frac{1}{2}\alpha+\frac{3}{4}}\e^{i\theta}}{q^{\alpha+1}}{\e^{-i\theta}t}\,
\qhyp{2}{1}{-q^{\frac{1}{2}\beta+\frac{1}{4}}\e^{-i\theta},
-q^{\frac{1}{2}\beta+\frac{3}{4}}\e^{-i\theta}}{q^{\beta+1}}{\e^{i\theta}t}\nonumber\\
& &{}=\sum_{n=0}^{\infty}\frac{(-q^{\frac{1}{2}(\alpha+\beta+1)},-q^{\frac{1}{2}(\alpha+\beta+2)};q)_n}
{(q^{\alpha+1},q^{\beta+1};q)_n}\frac{P_n^{(\alpha,\beta)}(x|q)}{q^{(\frac{1}{2}\alpha+\frac{1}{4})n}}t^n,
\quad x=\cos\theta.
\end{eqnarray}

\begin{eqnarray}
\label{GenContqJacobi2}
& &\qhyp{2}{1}{q^{\frac{1}{2}\alpha+\frac{1}{4}}\e^{i\theta},
-q^{\frac{1}{2}\beta+\frac{1}{4}}\e^{i\theta}}{-q^{\frac{1}{2}(\alpha+\beta+1)}}{\e^{-i\theta}t}\,
\qhyp{2}{1}{q^{\frac{1}{2}\alpha+\frac{3}{4}}\e^{-i\theta},
-q^{\frac{1}{2}\beta+\frac{3}{4}}\e^{-i\theta}}{-q^{\frac{1}{2}(\alpha+\beta+3)}}{\e^{i\theta}t}\nonumber\\
& &{}=\sum_{n=0}^{\infty}\frac{(-q^{\frac{1}{2}(\alpha+\beta+2)};q)_n}
{(-q^{\frac{1}{2}(\alpha+\beta+3)};q)_n}\frac{P_n^{(\alpha,\beta)}(x|q)}{q^{(\frac{1}{2}\alpha+\frac{1}{4})n}}t^n,
\quad x=\cos\theta.
\end{eqnarray}

\begin{eqnarray}
\label{GenContqJacobi3}
& &\qhyp{2}{1}{q^{\frac{1}{2}\alpha+\frac{1}{4}}\e^{i\theta},
-q^{\frac{1}{2}\beta+\frac{3}{4}}\e^{i\theta}}{-q^{\frac{1}{2}(\alpha+\beta+2)}}{\e^{-i\theta}t}\,
\qhyp{2}{1}{q^{\frac{1}{2}\alpha+\frac{3}{4}}\e^{-i\theta},
-q^{\frac{1}{2}\beta+\frac{1}{4}}\e^{-i\theta}}{-q^{\frac{1}{2}(\alpha+\beta+2)}}{\e^{i\theta}t}\nonumber\\
& &{}=\sum_{n=0}^{\infty}\frac{(-q^{\frac{1}{2}(\alpha+\beta+1)};q)_n}
{(-q^{\frac{1}{2}(\alpha+\beta+2)};q)_n}\frac{P_n^{(\alpha,\beta)}(x|q)}{q^{(\frac{1}{2}\alpha+\frac{1}{4})n}}t^n,
\quad x=\cos\theta.
\end{eqnarray}

\subsection*{Limit relations}

\subsubsection*{Askey-Wilson $\rightarrow$ Continuous $q$-Jacobi}
If we take $a=q^{\frac{1}{2}\alpha+\frac{1}{4}}$, $b=q^{\frac{1}{2}\alpha+\frac{3}{4}}$,
$c=-q^{\frac{1}{2}\beta+\frac{1}{4}}$ and $d=-q^{\frac{1}{2}\beta+\frac{3}{4}}$ in
the definition (\ref{DefAskeyWilson}) of the Askey-Wilson polynomials and
change the normalization we find the continuous $q$-Jacobi polynomials given
by (\ref{DefContqJacobi}):
$$\frac{q^{(\frac{1}{2}\alpha+\frac{1}{4})n}p_n(x;q^{\frac{1}{2}\alpha+\frac{1}{4}},q^{\frac{1}{2}\alpha+\frac{3}{4}},
-q^{\frac{1}{2}\beta+\frac{1}{4}},-q^{\frac{1}{2}\beta+\frac{3}{4}}|q)}
{(q,-q^{\frac{1}{2}(\alpha+\beta+1)},-q^{\frac{1}{2}(\alpha+\beta+2)};q)_n}
=P_n^{(\alpha,\beta)}(x|q).$$

\subsubsection*{Continuous $q$-Jacobi $\rightarrow$ Continuous $q$-Laguerre}
The continuous $q$-Laguerre polynomials given by (\ref{DefContqLaguerre})
follow simply from the continuous $q$-Jacobi polynomials given by (\ref{DefContqJacobi})
by taking the limit $\beta\rightarrow\infty$:
\begin{equation}
\lim _{\beta\rightarrow\infty} P_n^{(\alpha,\beta)}(x|q)=P_n^{(\alpha)}(x|q).
\end{equation}

\subsubsection*{Continuous $q$-Jacobi $\rightarrow$ Jacobi}
If we take the limit $q\rightarrow 1$ in the definitions (\ref{DefContqJacobi}) of the continuous
$q$-Jacobi polynomials we simply find the Jacobi polynomials given by (\ref{DefJacobi}):
\begin{equation}
\lim_{q\rightarrow 1}P_n^{(\alpha,\beta)}(x|q)=P_n^{(\alpha,\beta)}(x).
\end{equation}

\subsection*{Remarks} 
In \cite{Rahman81IV} M. Rahman takes
$a=q^{\frac{1}{2}}$, $b=q^{\alpha+\frac{1}{2}}$, $c=-q^{\beta+\frac{1}{2}}$ and
$d=-q^{\frac{1}{2}}$ in the definition (\ref{DefAskeyWilson}) of the
Askey-Wilson polynomials to obtain after renormalizing
\begin{eqnarray}
\label{DefContqJacobi2}
P_n^{(\alpha,\beta)}(x;q)&=&\frac{(q^{\alpha+1},-q^{\beta+1};q)_n}{(q,-q;q)_n}\nonumber\\
& &{}\mathindent{}\times
\qhyp{4}{3}{q^{-n},q^{n+\alpha+\beta+1},q^{\frac{1}{2}}\e^{i\theta},q^{\frac{1}{2}}\e^{-i\theta}}
{q^{\alpha+1},-q^{\beta+1},-q}{q}
\end{eqnarray}
with $x=\cos\theta$. These two $q$-analogues of the Jacobi polynomials are not really
different, since they are connected by the quadratic transformation:
$$P_n^{(\alpha,\beta)}(x|q^2)=\frac{(-q;q)_n}{(-q^{\alpha+\beta+1};q)_n}q^{n\alpha}P_n^{(\alpha,\beta)}(x;q).$$
The continuous $q$-Jacobi polynomials given by (\ref{DefContqJacobi2}) and
the continuous $q$-ultra\-spher\-ical (or Rogers) polynomials given by
(\ref{DefContqUltra}) are connected by the quadratic transformations:
$$C_{2n}(x;q^{\lambda}|q)=\frac{(q^{\lambda},-q;q)_n}
{(q^{\frac{1}{2}},-q^{\frac{1}{2}};q)_n}q^{-\frac{1}{2}n}
P_n^{(\lambda-\frac{1}{2},-\frac{1}{2})}(2x^2-1;q)$$
and
$$C_{2n+1}(x;q^{\lambda}|q)=\frac{(q^{\lambda},-1;q)_{n+1}}
{(q^{\frac{1}{2}},-q^{\frac{1}{2}};q)_{n+1}}q^{-\frac{1}{2}n}
xP_n^{(\lambda-\frac{1}{2},\frac{1}{2})}(2x^2-1;q).$$

\noindent
If we replace $q$ by $q^{-1}$ we find
$$P_n^{(\alpha,\beta)}(x|q^{-1})=q^{-n\alpha}P_n^{(\alpha,\beta)}(x|q)\quad\textrm{and}\quad
P_n^{(\alpha,\beta)}(x;q^{-1})=q^{-n(\alpha+\beta)}P_n^{(\alpha,\beta)}(x;q).$$

\subsection*{References}
\cite{AskeyWilson85}, \cite{Floreanini+96}, \cite{GasperRahman86},
\cite{GasperRahman90}, \cite{IsmailMassonSuslov}, \cite{IsmailRahman91},
\cite{IsmailRahmanZhang}, \cite{Koorn2007}, \cite{NassrallahRahman}, \cite{Nikiforov+},
\cite{Rahman81IV}, \cite{Rahman85}, \cite{Rahman86I}, \cite{Rahman88},
\cite{Spiridonov97}, \cite{Szwarc}.


\section*{Special cases}

\subsection{Continuous $q$-ultraspherical / Rogers}\index{Rogers polynomials}
\index{Continuous q-ultraspherical polynomials@Continuous $q$-ultraspherical polynomials}
\index{q-Ultraspherical polynomials@$q$-Ultraspherical polynomials!Continuous}
\par

\subsection*{Basic hypergeometric representation} If we set $a=\beta^{\frac{1}{2}}$, $b=\beta^{\frac{1}{2}}q^{\frac{1}{2}}$,
$c=-\beta^{\frac{1}{2}}$ and $d=-\beta^{\frac{1}{2}}q^{\frac{1}{2}}$ in the definition
(\ref{DefAskeyWilson}) of the Askey-Wilson polynomials and change the
normalization we obtain the continuous $q$-ultraspherical (or Rogers) polynomials:
\begin{eqnarray}
\label{DefContqUltra}
C_n(x;\beta|q)&=&\frac{(\beta^2;q)_n}{(q;q)_n}\beta^{-\frac{1}{2}n}\,\qhyp{4}{3}{q^{-n},\beta^2q^n,
\beta^{\frac{1}{2}}\e^{i\theta},\beta^{\frac{1}{2}}\e^{-i\theta}}{\beta q^{\frac{1}{2}},
-\beta,-\beta q^{\frac{1}{2}}}{q}\\
&=&\frac{(\beta^2;q)_n}{(q;q)_n}\beta^{-n}\e^{-in\theta}\,\qhyp{3}{2}{q^{-n},\beta,\beta\e^{2i\theta}}
{\beta^2,0}{q}\nonumber\\
&=&\frac{(\beta;q)_n}{(q;q)_n}\e^{in\theta}\,\qhyp{2}{1}{q^{-n},\beta}{\beta^{-1}q^{-n+1}}{\beta^{-1}q\e^{-2i\theta}},
\quad x=\cos\theta.\nonumber
\end{eqnarray}

\subsection*{Orthogonality relation}
\begin{eqnarray}
\label{OrtContqUltra}
& &\frac{1}{2\pi}\int_{-1}^1\frac{w(x)}{\sqrt{1-x^2}}C_m(x;\beta|q)C_n(x;\beta|q)\,dx\nonumber\\
& &{}=\frac{(\beta,\beta q;q)_{\infty}}{(\beta^2,q;q)_{\infty}}
\frac{(\beta^2;q)_n}{(q;q)_n}\frac{(1-\beta)}{(1-\beta q^n)}\,\delta_{mn},\quad|\beta|<1,
\end{eqnarray}
where
\begin{eqnarray*}
w(x):=w(x;\beta|q)&=&\left|\frac{(\e^{2i\theta};q)_{\infty}}
{(\beta^{\frac{1}{2}}\e^{i\theta},\beta^{\frac{1}{2}}q^{\frac{1}{2}}\e^{i\theta},
-\beta^{\frac{1}{2}}\e^{i\theta},-\beta^{\frac{1}{2}}q^{\frac{1}{2}}\e^{i\theta};q)_{\infty}}\right|^2\\
&=&\left|\frac{(\e^{2i\theta};q)_{\infty}}{(\beta\e^{2i\theta};q)_{\infty}}\right|^2
=\frac{h(x,1)h(x,-1)h(x,q^{\frac{1}{2}})h(x,-q^{\frac{1}{2}})}
{h(x,\beta^{\frac{1}{2}})h(x,\beta^{\frac{1}{2}}q^{\frac{1}{2}})
h(x,-\beta^{\frac{1}{2}})h(x,-\beta^{\frac{1}{2}}q^{\frac{1}{2}})},
\end{eqnarray*}
with
$$h(x,\alpha):=\prod_{k=0}^{\infty}\left(1-2\alpha xq^k+\alpha^2q^{2k}\right)
=\left(\alpha\e^{i\theta},\alpha\e^{-i\theta};q\right)_{\infty},\quad x=\cos\theta.$$

\subsection*{Recurrence relation}
\begin{eqnarray}
\label{RecContqUltra}
2(1-\beta q^n)xC_n(x;\beta|q)&=&(1-q^{n+1})C_{n+1}(x;\beta|q)\nonumber\\
& &{}\mathindent{}+(1-\beta^2 q^{n-1})C_{n-1}(x;\beta|q).
\end{eqnarray}

\subsection*{Normalized recurrence relation}
\begin{equation}
\label{NormRecContqUltra}
xp_n(x)=p_{n+1}(x)+
\frac{(1-q^n)(1-\beta^2q^{n-1})}{4(1-\beta q^{n-1})(1-\beta q^n)}p_{n-1}(x),
\end{equation}
where
$$C_n(x;\beta|q)=\frac{2^n(\beta;q)_n}{(q;q)_n}p_n(x).$$

\subsection*{$q$-Difference equation}
\begin{equation}
\label{dvContqUltra}
(1-q)^2D_q\left[{\tilde w}(x;\beta q|q)D_qy(x)\right]+
\lambda_n{\tilde w}(x;\beta|q)y(x)=0,
\end{equation}
where
$$y(x)=C_n(x;\beta|q),$$
$${\tilde w}(x;\beta|q):=\frac{w(x;\beta|q)}{\sqrt{1-x^2}}$$
and
$$\lambda_n=4q^{-n+1}(1-q^n)(1-\beta^2q^n).$$

\subsection*{Forward shift operator}
\begin{equation}
\label{shift1ContqUltraI}
\delta_qC_n(x;\beta|q)=-q^{-\frac{1}{2}n}(1-\beta)(\e^{i\theta}-\e^{-i\theta})
C_{n-1}(x;\beta q|q),\quad x=\cos\theta
\end{equation}
or equivalently
\begin{equation}
\label{shift1ContqUltraII}
D_qC_n(x;\beta|q)=2q^{-\frac{1}{2}(n-1)}\frac{1-\beta}{1-q}C_{n-1}(x;\beta q|q).
\end{equation}

\newpage

\subsection*{Backward shift operator}
\begin{eqnarray}
\label{shift2ContqUltraI}
& &\delta_q\left[{\tilde w}(x;\beta|q)C_n(x;\beta|q)\right]\nonumber\\
& &{}=q^{-\frac{1}{2}(n+1)}\frac{(1-q^{n+1})(1-\beta^2q^{n-1})}{(1-\beta q^{-1})}
(\e^{i\theta}-\e^{-i\theta})\nonumber\\
& &{}\mathindent{}\times{\tilde w}(x;\beta q^{-1}|q)C_{n+1}(x;\beta q^{-1}|q),\quad x=\cos\theta
\end{eqnarray}
or equivalently
\begin{eqnarray}
\label{shift2ContqUltraII}
& &D_q\left[{\tilde w}(x;\beta|q)C_n(x;\beta|q)\right]\nonumber\\
& &{}=-\frac{2q^{-\frac{1}{2}n}(1-q^{n+1})(1-\beta^2q^{n-1})}{(1-q)(1-\beta q^{-1})}\nonumber\\
& &{}\mathindent{}\times{\tilde w}(x;\beta q^{-1}|q)C_{n+1}(x;\beta q^{-1}|q).
\end{eqnarray}

\subsection*{Rodrigues-type formula}
\begin{eqnarray}
\label{RodContqUltra}
& &{\tilde w}(x;\beta|q)C_n(x;\beta|q)\nonumber\\
& &{}=\left(\frac{q-1}{2}\right)^n
q^{\frac{1}{4}n(n-1)}\frac{(\beta;q)_n}{(q,\beta^2q^n;q)_n}
\left(D_q\right)^n\left[{\tilde w}(x;\beta q^n|q)\right].
\end{eqnarray}

\subsection*{Generating functions}
\begin{equation}
\label{GenContqUltra1}
\left|\frac{(\beta\e^{i\theta}t;q)_{\infty}}{(\e^{i\theta}t;q)_{\infty}}\right|^2
=\frac{(\beta\e^{i\theta}t,\beta\e^{-i\theta}t;q)_{\infty}}
{(\e^{i\theta}t,\e^{-i\theta}t;q)_{\infty}}
=\sum_{n=0}^{\infty}C_n(x;\beta|q)t^n,\quad x=\cos\theta.
\end{equation}

\begin{equation}
\label{GenContqUltra2}
\frac{1}{(\e^{i\theta}t;q)_{\infty}}\,
\qhyp{2}{1}{\beta,\beta\e^{2i\theta}}{\beta^2}{\e^{-i\theta}t}
=\sum_{n=0}^{\infty}\frac{C_n(x;\beta|q)}{(\beta^2;q)_n}t^n,\quad x=\cos\theta.
\end{equation}

\begin{eqnarray}
\label{GenContqUltra3}
& &(\e^{-i\theta}t;q)_{\infty}\cdot
\qhyp{2}{1}{\beta,\beta\e^{2i\theta}}{\beta^2}{\e^{-i\theta}t}\nonumber\\
& &{}=\sum_{n=0}^{\infty}\frac{(-1)^n\beta^nq^{\binom{n}{2}}}{(\beta^2;q)_n}C_n(x;\beta|q)t^n,
\quad x=\cos\theta.
\end{eqnarray}

\begin{eqnarray}
\label{GenContqUltra4}
& &\qhyp{2}{1}{\beta^{\frac{1}{2}}\e^{i\theta},
\beta^{\frac{1}{2}}q^{\frac{1}{2}}\e^{i\theta}}{\beta q^{\frac{1}{2}}}{\e^{-i\theta}t}\,
\qhyp{2}{1}{-\beta^{\frac{1}{2}}\e^{-i\theta},
-\beta^{\frac{1}{2}}q^{\frac{1}{2}}\e^{-i\theta}}{\beta q^{\frac{1}{2}}}{\e^{i\theta}t}\nonumber\\
& &{}=\sum_{n=0}^{\infty}\frac{(-\beta,-\beta q^{\frac{1}{2}};q)_n}
{(\beta^2,\beta q^{\frac{1}{2}};q)_n}C_n(x;\beta|q)t^n,\quad x=\cos\theta.
\end{eqnarray}

\begin{eqnarray}
\label{GenContqUltra5}
& &\qhyp{2}{1}{\beta^{\frac{1}{2}}\e^{i\theta},
-\beta^{\frac{1}{2}}\e^{i\theta}}{-\beta}{\e^{-i\theta}t}\,
\qhyp{2}{1}{\beta^{\frac{1}{2}}q^{\frac{1}{2}}\e^{-i\theta},
-\beta^{\frac{1}{2}}q^{\frac{1}{2}}\e^{-i\theta}}{-\beta q}{\e^{i\theta}t}\nonumber\\
& &{}=\sum_{n=0}^{\infty}\frac{(\beta q^{\frac{1}{2}},-\beta q^{\frac{1}{2}};q)_n}
{(\beta^2,-\beta q;q)_n}C_n(x;\beta|q)t^n,\quad x=\cos\theta.
\end{eqnarray}

\begin{eqnarray}
\label{GenContqUltra6}
& &\qhyp{2}{1}{\beta^{\frac{1}{2}}\e^{i\theta},
-\beta^{\frac{1}{2}}q^{\frac{1}{2}}\e^{i\theta}}{-\beta q^{\frac{1}{2}}}{\e^{-i\theta}t}\,
\qhyp{2}{1}{\beta^{\frac{1}{2}}q^{\frac{1}{2}}\e^{-i\theta},
-\beta^{\frac{1}{2}}\e^{-i\theta}}{-\beta q^{\frac{1}{2}}}{\e^{i\theta}t}\nonumber\\
& &{}=\sum_{n=0}^{\infty}\frac{(-\beta,\beta q^{\frac{1}{2}};q)_n}
{(\beta^2,-\beta q^{\frac{1}{2}};q)_n}C_n(x;\beta|q)t^n,\quad x=\cos\theta.
\end{eqnarray}

\begin{eqnarray}
\label{GenContqUltra7}
& &\frac{(\gamma\e^{i\theta}t;q)_{\infty}}{(\e^{i\theta}t;q)_{\infty}}\,
\qhyp{3}{2}{\gamma,\beta,\beta\e^{2i\theta}}{\beta^2,\gamma\e^{i\theta}t}{\e^{-i\theta}t}\nonumber\\
& &{}=\sum_{n=0}^{\infty}\frac{(\gamma;q)_n}{(\beta^2;q)_n}C_n(x;\beta|q)t^n,
\quad x=\cos\theta,\quad\textrm{$\gamma$ arbitrary}.
\end{eqnarray}

\subsection*{Limit relations}

\subsubsection*{Askey-Wilson $\rightarrow$ Continuous $q$-ultraspherical / Rogers}
If we set $a=\beta^{\frac{1}{2}}$, $b=\beta^{\frac{1}{2}}q^{\frac{1}{2}}$,
$c=-\beta^{\frac{1}{2}}$ and $d=-\beta^{\frac{1}{2}}q^{\frac{1}{2}}$ in the definition
(\ref{DefAskeyWilson}) of the Askey-Wilson polynomials and change the
normalization we obtain the continuous $q$-ultraspherical (or Rogers)
polynomials given by (\ref{DefContqUltra}). In fact we have:
$$\frac{(\beta^2;q)_np_n(x;\beta^{\frac{1}{2}},\beta^{\frac{1}{2}}q^{\frac{1}{2}},
-\beta^{\frac{1}{2}},-\beta^{\frac{1}{2}}q^{\frac{1}{2}}|q)}
{(\beta q^{\frac{1}{2}},-\beta,-\beta q^{\frac{1}{2}},q;q)_n}=C_n(x;\beta|q).$$

\subsubsection*{$q$-Meixner-Pollaczek $\rightarrow$ Continuous $q$-ultraspherical /
Rogers}
If we take $\theta=0$ and $a=\beta$ in the definition (\ref{DefqMP}) of the
$q$-Meixner-Pollaczek polynomials we obtain the continuous
$q$-ultraspherical (or Rogers) polynomials given by (\ref{DefContqUltra}):
$$P_n(\cos\phi;\beta|q)=C_n(\cos\phi;\beta|q).$$

\subsubsection*{Continuous $q$-ultraspherical / Rogers $\rightarrow$ Continuous $q$-Hermite}
The continuous $q$-Hermite polynomials given by (\ref{DefContqHermite})
can be obtained from the continuous $q$-ultraspherical (or Rogers)
polynomials given by (\ref{DefContqUltra}) by taking the limit
$\beta\rightarrow 0$. In fact we have
\begin{equation}
\lim_{\beta\rightarrow 0}C_n(x;\beta|q)=\frac{H_n(x|q)}{(q;q)_n}.
\end{equation}

\subsubsection*{Continuous $q$-ultraspherical / Rogers $\rightarrow$ Gegenbauer /
Ultra\-spherical}
If we set $\beta=q^{\lambda}$ in the definition (\ref{DefContqUltra}) of the
continuous $q$-ultraspherical (or Rogers) polynomials and let $q$ tend to $1$ we obtain
the Gegenbauer (or ultraspherical) polynomials given by (\ref{DefGegenbauer}):
\begin{equation}
\lim_{q\rightarrow 1}C_n(x;q^{\lambda}|q)=C_n^{(\lambda)}(x).
\end{equation}

\subsection*{Remarks} 
The continuous $q$-ultraspherical (or Rogers) polynomials
can also be written as:
$$C_n(x;\beta|q)=\sum_{k=0}^n\frac{(\beta;q)_k(\beta;q)_{n-k}}{(q;q)_k(q;q)_{n-k}}
\e^{i(n-2k)\theta},\quad x=\cos\theta.$$

\noindent
They can be obtained from the continuous $q$-Jacobi polynomials given by
(\ref{DefContqJacobi}) in the following way. Set $\beta=\alpha$ in the definition
(\ref{DefContqJacobi}) and replace $q^{\alpha+\frac{1}{2}}$ by $\beta$ and we find
the continuous $q$-ultraspherical (or Rogers) polynomials with a different
normalization. We have
$$P_n^{(\alpha,\alpha)}(x|q)\stackrel{q^{\alpha+\frac{1}{2}}\rightarrow\beta}{\longrightarrow}
\frac{(\beta q^{\frac{1}{2}};q)_n}{(\beta^2;q)_n}\beta^{\frac{1}{2}n}C_n(x;\beta|q).$$

\noindent
If we set $\beta=q^{\alpha+\frac{1}{2}}$ in the definition (\ref{DefContqUltra}) of
the $q$-ultraspherical (or Rogers) polynomials we find the continuous
$q$-Jacobi polynomials given by (\ref{DefContqJacobi}) with $\beta=\alpha$. In fact we have
$$C_n(x;q^{\alpha+\frac{1}{2}}|q)=
\frac{(q^{2\alpha+1};q)_n}{(q^{\alpha+1};q)_nq^{(\frac{1}{2}\alpha+\frac{1}{4})n}}P_n^{(\alpha,\alpha)}(x|q).$$

\noindent
If we replace $q$ by $q^{-1}$ we find
$$C_n(x;\beta|q^{-1})=(\beta q)^nC_n(x;\beta^{-1}|q).$$

\noindent
The special case $\beta=q$ of the continuous $q$-ultraspherical (or Rogers)
polynomials equals the Chebyshev polynomials of the second kind given by
(\ref{DefChebyshevII}). In fact we have
$$C_n(x;q|q)=\frac{\sin(n+1)\theta}{\sin\theta}=U_n(x),\quad x=\cos\theta.$$
The limit case $\beta\rightarrow 1$ leads to the Chebyshev polynomials of the
first kind given by (\ref{DefChebyshevI}) in the following way:
$$\lim_{\beta\rightarrow 1}\frac{1-q^n}{2(1-\beta)}C_n(x;\beta|q)=\cos n\theta
=T_n(x),\quad x=\cos\theta,\quad n=1,2,3,\ldots.$$

\noindent
The continuous $q$-Jacobi polynomials given by (\ref{DefContqJacobi2}) and
the continuous $q$-ultraspherical (or Rogers) polynomials given by
(\ref{DefContqUltra}) are connected by the quadratic transformations:
$$C_{2n}(x;q^{\lambda}|q)=\frac{(q^{\lambda},-q;q)_n}
{(q^{\frac{1}{2}},-q^{\frac{1}{2}};q)_n}q^{-\frac{1}{2}n}
P_n^{(\lambda-\frac{1}{2},-\frac{1}{2})}(2x^2-1;q)$$
and
$$C_{2n+1}(x;q^{\lambda}|q)=\frac{(q^{\lambda},-1;q)_{n+1}}
{(q^{\frac{1}{2}},-q^{\frac{1}{2}};q)_{n+1}}q^{-\frac{1}{2}n}
xP_n^{(\lambda-\frac{1}{2},\frac{1}{2})}(2x^2-1;q).$$

\noindent
Finally we remark that the continuous $q$-ultraspherical (or Rogers)
polynomials are related to the continuous $q$-Legendre polynomials given by (\ref{DefContqLegendre}) in the following way:
$$C_n(x;q^{\frac{1}{2}}|q)=q^{-\frac{1}{4}n}P_n(x|q).$$

\subsection*{References}
\cite{AlSalamAllaway84I}, \cite{AlSalamAllaway84II}, \cite{AlSalam90}, \cite{AndrewsAskey85},
\cite{Askey89I}, \cite{Askey89II}, \cite{Askey89III}, \cite{AskeyIsmail80},
\cite{AskeyIsmail83}, \cite{AskeyIsmail84}, \cite{AskeyKoornRahman}, \cite{AskeyWilson85},
\cite{AtakRahmanSuslov}, \cite{Bressoud81}, \cite{BustozIsmail82}, \cite{BustozIsmail83},
\cite{FloreaniniVinetI}, \cite{Gasper81}, \cite{Gasper85}, \cite{Gasper89},
\cite{GasperRahman83II}, \cite{GasperRahman86}, \cite{GasperRahman89}, \cite{GasperRahman90},
\cite{Ismail86I}, \cite{IsmailMassonSuslov}, \cite{IsmailStanton88}, \cite{IsmailStanton97},
\cite{IsmailZhang}, \cite{Koelink95II}, \cite{Koelink96I}, \cite{Koorn90I}, \cite{Koorn2005},
\cite{Koorn2007}, \cite{Liu}, \cite{NassrallahRahman}, \cite{Nikiforov+}, \cite{NoumiMimachi91},
\cite{Rahman88}, \cite{Rahman92}, \cite{RahmanVerma86I}, \cite{RahmanVerma86II},
\cite{RahmanVerma87}, \cite{Rogers93}, \cite{Rogers94}, \cite{Rogers95}, \cite{Spiridonov96}.


\subsection{Continuous $q$-Legendre}
\index{Continuous q-Legendre polynomials@Continuous $q$-Legendre polynomials}
\index{q-Legendre polynomials@$q$-Legendre polynomials!Continuous}
\par

\subsection*{Basic hypergeometric representation} The continuous $q$-Legendre polynomials are continuous
$q$-Jacobi polynomials with $\alpha=\beta=0$:
\begin{equation}
\label{DefContqLegendre}
P_n(x|q)=\qhyp{4}{3}{q^{-n},q^{n+1},q^{\frac{1}{4}}\e^{i\theta},
q^{\frac{1}{4}}\e^{-i\theta}}{q,-q^{\frac{1}{2}},-q}{q},\quad x=\cos\theta.
\end{equation}

\subsection*{Orthogonality relation}
\begin{eqnarray}
\label{OrtContqLegendre}
& &\frac{1}{2\pi}\int_{-1}^1\frac{w(x;1|q)}{\sqrt{1-x^2}}P_m(x|q)P_n(x|q)\,dx\nonumber\\
& &{}=\frac{(q^{\frac{1}{2}};q)_{\infty}}{(q,q,-q^{\frac{1}{2}},-q;q)_{\infty}}
\frac{q^{\frac{1}{2}n}}{1-q^{n+\frac{1}{2}}}\,\delta_{mn},
\end{eqnarray}
where
\begin{eqnarray*}
w(x;a|q)&=&\left|\frac{(\e^{2i\theta};q)_{\infty}}{(aq^{\frac{1}{4}}\e^{i\theta},
aq^{\frac{3}{4}}\e^{i\theta},-aq^{\frac{1}{4}}\e^{i\theta},
-aq^{\frac{3}{4}}\e^{i\theta};q)_{\infty}}\right|^2\\
&=&\left|\frac{(\e^{i\theta},-\e^{i\theta};q^{\frac{1}{2}})_{\infty}}
{(aq^{\frac{1}{4}}\e^{i\theta},-aq^{\frac{1}{4}}\e^{i\theta};q^{\frac{1}{2}})_{\infty}}\right|^2
=\left|\frac{(\e^{2i\theta};q)_{\infty}}{(a^2q^{\frac{1}{2}}\e^{2i\theta};q)_{\infty}}\right|^2\\
&=&\frac{h(x,1)h(x,-1)h(x,q^{\frac{1}{2}})h(x,-q^{\frac{1}{2}})}
{h(x,aq^{\frac{1}{4}})h(x,aq^{\frac{3}{4}})h(x,-aq^{\frac{1}{4}})h(x,-aq^{\frac{3}{4}})},
\end{eqnarray*}
with
$$h(x,\alpha):=\prod_{k=0}^{\infty}\left(1-2\alpha xq^k+\alpha^2q^{2k}\right)
=\left(\alpha\e^{i\theta},\alpha\e^{-i\theta};q\right)_{\infty},\quad x=\cos\theta.$$

\subsection*{Recurrence relation}
\begin{equation}
\label{RecContqLegendre}
2(1-q^{n+\frac{1}{2}})xP_n(x|q)=q^{-\frac{1}{4}}(1-q^{n+1})P_{n+1}(x|q)
+q^{\frac{1}{4}}(1-q^n)P_{n-1}(x|q).
\end{equation}

\subsection*{Normalized recurrence relation}
\begin{equation}
\label{NormRecContqLegendre}
xp_n(x)=p_{n+1}(x)+\frac{(1-q^n)^2}{4(1-q^{n-\frac{1}{2}})(1-q^{n+\frac{1}{2}})}p_{n-1}(x),
\end{equation}
where
$$P_n(x|q)=\frac{2^nq^{\frac{1}{4}n}(q^{\frac{1}{2}};q)_n}{(q;q)_n}p_n(x).$$

\subsection*{$q$-Difference equation}
\begin{equation}
\label{dvContqLegendre}
(1-q)^2D_q\left[{\tilde w}(x;q^{\frac{1}{2}}|q)D_qy(x)\right]+
\lambda_n{\tilde w}(x;1|q)y(x)=0,\quad y(x)=P_n(x|q),
\end{equation}
where
$$\lambda_n=4q^{-n+1}(1-q^n)(1-q^{n+1})$$
and
$${\tilde w}(x;a|q):=\frac{w(x;a|q)}{\sqrt{1-x^2}}.$$

\subsection*{Rodrigues-type formula}
\begin{equation}
\label{RodContqLegendre}
{\tilde w}(x;1|q)P_n(x|q)=\left(\frac{q-1}{2}\right)^n
\frac{q^{\frac{1}{4}n^2}}{(q,-q^{\frac{1}{2}},-q;q)_n}
\left(D_q\right)^n\left[{\tilde w}(x;q^{\frac{1}{2}n}|q)\right].
\end{equation}

\subsection*{Generating functions}
\begin{equation}
\label{GenContqLegendre1}
\left|\frac{(q^{\frac{1}{2}}\e^{i\theta}t;q)_{\infty}}{(\e^{i\theta}t;q)_{\infty}}\right|^2
=\frac{(q^{\frac{1}{2}}\e^{i\theta}t,q^{\frac{1}{2}}\e^{-i\theta}t;q)_{\infty}}
{(\e^{i\theta}t,\e^{-i\theta}t;q)_{\infty}}
=\sum_{n=0}^{\infty}\frac{P_n(x|q)}{q^{\frac{1}{4}n}}t^n,\quad x=\cos\theta.
\end{equation}

\begin{equation}
\label{GenContqLegendre2}
\frac{1}{(\e^{i\theta}t;q)_{\infty}}\,
\qhyp{2}{1}{q^{\frac{1}{2}},q^{\frac{1}{2}}\e^{2i\theta}}{q}{\e^{-i\theta}t}
=\sum_{n=0}^{\infty}\frac{P_n(x|q)}{(q;q)_nq^{\frac{1}{4}n}}t^n,\quad x=\cos\theta.
\end{equation}

\begin{eqnarray}
\label{GenContqLegendre3}
& &(\e^{-i\theta}t;q)_{\infty}\cdot
\qhyp{2}{1}{q^{\frac{1}{2}},q^{\frac{1}{2}}\e^{2i\theta}}{q}{\e^{-i\theta}t}\nonumber\\
& &{}=\sum_{n=0}^{\infty}\frac{(-1)^nq^{\frac{1}{4}n+\binom{n}{2}}}{(q;q)_n}P_n(x|q)t^n,
\quad x=\cos\theta.
\end{eqnarray}

\begin{eqnarray}
\label{GenContqLegendre4}
& &\qhyp{2}{1}{q^{\frac{1}{4}}\e^{i\theta},q^{\frac{3}{4}}\e^{i\theta}}{q}{\e^{-i\theta}t}\,
\qhyp{2}{1}{-q^{\frac{1}{4}}\e^{-i\theta},-q^{\frac{3}{4}}\e^{-i\theta}}{q}{\e^{i\theta}t}\nonumber\\
& &{}=\sum_{n=0}^{\infty}\frac{(-q^{\frac{1}{2}},-q;q)_n}{(q,q;q)_n}
\frac{P_n(x|q)}{q^{\frac{1}{4}n}}t^n,\quad x=\cos\theta.
\end{eqnarray}

\begin{eqnarray}
\label{GenContqLegendre5}
& &\qhyp{2}{1}{q^{\frac{1}{4}}\e^{i\theta},-q^{\frac{1}{4}}\e^{i\theta}}{-q^{\frac{1}{2}}}{\e^{-i\theta}t}\,
\qhyp{2}{1}{q^{\frac{3}{4}}\e^{-i\theta},-q^{\frac{3}{4}}\e^{-i\theta}}{-q^{\frac{3}{2}}}{\e^{i\theta}t}\nonumber\\
& &{}=\sum_{n=0}^{\infty}\frac{(-q;q)_n}{(-q^{\frac{3}{2}};q)_n}
\frac{P_n(x|q)}{q^{\frac{1}{4}n}}t^n,\quad x=\cos\theta.
\end{eqnarray}

\begin{eqnarray}
\label{GenContqLegendre6}
& &\qhyp{2}{1}{q^{\frac{1}{4}}\e^{i\theta},-q^{\frac{3}{4}}\e^{i\theta}}{-q}{\e^{-i\theta}t}\,
\qhyp{2}{1}{q^{\frac{3}{4}}\e^{-i\theta},-q^{\frac{1}{4}}\e^{-i\theta}}{-q}{\e^{i\theta}t}\nonumber\\
& &{}=\sum_{n=0}^{\infty}\frac{(-q^{\frac{1}{2}};q)_n}{(-q;q)_n}
\frac{P_n(x|q)}{q^{\frac{1}{4}n}}t^n,\quad x=\cos\theta.
\end{eqnarray}

\begin{eqnarray}
\label{GenContqLegendre7}
& &\frac{(\gamma\e^{i\theta}t;q)_{\infty}}{(\e^{i\theta}t;q)_{\infty}}\,
\qhyp{3}{2}{\gamma,q^{\frac{1}{2}},q^{\frac{1}{2}}\e^{2i\theta}}{q,\gamma\e^{i\theta}t}{\e^{-i\theta}t}\nonumber\\
& &{}=\sum_{n=0}^{\infty}\frac{(\gamma;q)_n}{(q;q)_n}\frac{P_n(x|q)}{q^{\frac{1}{4}n}}t^n,
\quad x=\cos\theta,\quad\textrm{$\gamma$ arbitrary}.
\end{eqnarray}

\subsection*{Limit relations}

\subsubsection*{Continuous $q$-Legendre $\rightarrow$ Legendre / Spherical}
The Legendre (or spherical) polynomials given by (\ref{DefLegendre}) easily
follow from the continuous $q$-Legendre polynomials given by
(\ref{DefContqLegendre}) by taking the limit $q\rightarrow 1$:
\begin{equation}
\lim_{q\rightarrow 1}P_n(x|q)=P_n(x).
\end{equation}

\subsection*{Remarks} 
The continuous $q$-Legendre polynomials can also be written
as:
$$P_n(x|q)=q^{\frac{1}{4}n}\sum_{k=0}^n
\frac{(q^{\frac{1}{2}};q)_k(q^{\frac{1}{2}};q)_{n-k}}{(q;q)_k(q;q)_{n-k}}
\e^{i(n-2k)\theta},\quad x=\cos\theta.$$

\noindent
If we set $\alpha=\beta=0$ in (\ref{DefContqJacobi2}) we find
$$P_n(x;q)=\qhyp{4}{3}{q^{-n},q^{n+1},q^{\frac{1}{2}}\e^{i\theta},q^{\frac{1}{2}}\e^{-i\theta}}
{q,-q,-q}{q},\quad x=\cos\theta,$$
but these are not really different from those given by (\ref{DefContqLegendre})
in view of the quadratic transformation
$$P_n(x|q^2)=P_n(x;q).$$

\noindent
If we replace $q$ by $q^{-1}$ we find
$$P_n(x|q^{-1})=P_n(x|q).$$

\noindent
The continuous $q$-Legendre polynomials are related to the continuous
$q$-ultra\-spher\-i\-cal (or Rogers) polynomials given by (\ref{DefContqUltra}) in
the following way:
$$P_n(x|q)=q^{\frac{1}{4}n}C_n(x;q^{\frac{1}{2}}|q).$$

\subsection*{References}
\cite{KoelinkE}, \cite{Koelink94}, \cite{Koorn89I}, \cite{Koorn90II},
\cite{Koorn93}.


\section{Big $q$-Laguerre}
\index{Big q-Laguerre polynomials@Big $q$-Laguerre polynomials}
\index{q-Laguerre polynomials@$q$-Laguerre polynomials!Big}
\par\setcounter{equation}{0}

\subsection*{Basic hypergeometric representation}
\begin{eqnarray}
\label{DefBigqLaguerre}
P_n(x;a,b;q)&=&\qhyp{3}{2}{q^{-n},0,x}{aq,bq}{q}\\
&=&\frac{1}{(b^{-1}q^{-n};q)_n}\,\qhyp{2}{1}{q^{-n},aqx^{-1}}{aq}{\frac{x}{b}}.\nonumber
\end{eqnarray}

\subsection*{Orthogonality relation}
For $0<aq<1$ and $b<0$ we have
\begin{eqnarray}
\label{OrtBigqLaguerre}
& &\int_{bq}^{aq}\frac{(a^{-1}x,b^{-1}x;q)_{\infty}}{(x;q)_{\infty}}
P_m(x;a,b;q)P_n(x;a,b;q)\,d_qx\nonumber\\
& &{}=aq(1-q)\frac{(q,a^{-1}b,ab^{-1}q;q)_{\infty}}
{(aq,bq;q)_{\infty}}\frac{(q;q)_n}{(aq,bq;q)_n}(-abq^2)^nq^{\binom{n}{2}}\,\delta_{mn}.
\end{eqnarray}

\subsection*{Recurrence relation}
\begin{eqnarray}
\label{RecBigqLaguerre}
(x-1)P_n(x;a,b;q)&=&A_nP_{n+1}(x;a,b;q)-\left(A_n+C_n\right)P_n(x;a,b;q)\nonumber\\
& &{}\mathindent{}+C_nP_{n-1}(x;a,b;q),
\end{eqnarray}
where
$$\left\{\begin{array}{l}
\displaystyle A_n=(1-aq^{n+1})(1-bq^{n+1})\\
\\
\displaystyle C_n=-abq^{n+1}(1-q^n).\end{array}\right.$$

\subsection*{Normalized recurrence relation}
\begin{eqnarray}
\label{NormRecBigqLaguerre}
xp_n(x)&=&p_{n+1}(x)+\left[1-(A_n+C_n)\right]p_n(x)\nonumber\\
& &{}\mathindent{}-abq^{n+1}(1-q^n)(1-aq^n)(1-bq^n)p_{n-1}(x),
\end{eqnarray}
where
$$P_n(x;a,b;q)=\frac{1}{(aq,bq;q)_n}p_n(x).$$

\subsection*{$q$-Difference equation}
\begin{equation}
\label{dvBigqLaguerre}
q^{-n}(1-q^n)x^2y(x)=B(x)y(qx)-\left[B(x)+D(x)\right]y(x)+D(x)y(q^{-1}x),
\end{equation}
where
$$y(x)=P_n(x;a,b;q)$$
and
$$\left\{\begin{array}{l}\displaystyle B(x)=abq(1-x)\\
\\
\displaystyle D(x)=(x-aq)(x-bq).\end{array}\right.$$

\subsection*{Forward shift operator}
\begin{equation}
\label{shift1BigqLaguerreI}
P_n(x;a,b;q)-P_n(qx;a,b;q)=\frac{q^{-n+1}(1-q^n)}
{(1-aq)(1-bq)}xP_{n-1}(qx;aq,bq;q)
\end{equation}
or equivalently
\begin{equation}
\label{shift1BigqLaguerreII}
\mathcal{D}_qP_n(x;a,b;q)=\frac{q^{-n+1}(1-q^n)}
{(1-q)(1-aq)(1-bq)}P_{n-1}(qx;aq,bq;q).
\end{equation}

\subsection*{Backward shift operator}
\begin{eqnarray}
\label{shift2BigqLaguerreI}
& &(x-a)(x-b)P_n(x;a,b;q)-ab(1-x)P_n(qx;a,b;q)\nonumber\\
& &{}=(1-a)(1-b)xP_{n+1}(x;aq^{-1},bq^{-1};q)
\end{eqnarray}
or equivalently
\begin{eqnarray}
\label{shift2BigqLaguerreII}
& &\mathcal{D}_q\left[w(x;a,b;q)P_n(x;a,b;q)\right]\nonumber\\
& &{}=\frac{(1-a)(1-b)}{ab(1-q)}w(x;aq^{-1},bq^{-1};q)P_{n+1}(x;aq^{-1},bq^{-1};q),
\end{eqnarray}
where
$$w(x;a,b;q)=\frac{(a^{-1}x,b^{-1}x;q)_{\infty}}{(x;q)_{\infty}}.$$

\subsection*{Rodrigues-type formula}
\begin{eqnarray}
\label{RodBigqLaguerre}
& &w(x;a,b;q)P_n(x;a,b;q)\nonumber\\
& &{}=\frac{a^nb^nq^{n(n+1)}(1-q)^n}
{(aq,bq;q)_n}\left(\mathcal{D}_q\right)^n\left[w(x;aq^n,bq^n;q)\right].
\end{eqnarray}

\subsection*{Generating functions}
\begin{equation}
\label{GenBigqLaguerre1}
(bqt;q)_{\infty}\cdot\qhyp{2}{1}{aqx^{-1},0}{aq}{xt}
=\sum_{n=0}^{\infty}\frac{(bq;q)_n}{(q;q)_n}P_n(x;a,b;q)t^n.
\end{equation}

\begin{equation}
\label{GenBigqLaguerre2}
(aqt;q)_{\infty}\cdot\qhyp{2}{1}{bqx^{-1},0}{bq}{xt}
=\sum_{n=0}^{\infty}\frac{(aq;q)_n}{(q;q)_n}P_n(x;a,b;q)t^n.
\end{equation}

\begin{equation}
\label{GenBigqLaguerre3}
(t;q)_{\infty}\cdot\qhyp{3}{2}{0,0,x}{aq,bq}{t}=\sum_{n=0}^{\infty}
\frac{(-1)^nq^{\binom{n}{2}}}{(q;q)_n}P_n(x;a,b;q)t^n.
\end{equation}

\subsection*{Limit relations}

\subsubsection*{Big $q$-Jacobi $\rightarrow$ Big $q$-Laguerre}
If we set $b=0$ in the definition (\ref{DefBigqJacobi}) of
the big $q$-Jacobi polynomials we obtain the big $q$-Laguerre
polynomials given by (\ref{DefBigqLaguerre}):
$$P_n(x;a,0,c;q)=P_n(x;a,c;q).$$

\subsubsection*{Big $q$-Laguerre $\rightarrow$ Little $q$-Laguerre / Wall}
The little $q$-Laguerre (or Wall) polynomials given by (\ref{DefLittleqLaguerre})
can be obtained from the big $q$-Laguerre polynomials by taking $x\rightarrow bqx$
in (\ref{DefBigqLaguerre}) and letting $b\rightarrow -\infty$:
\begin{equation}
\lim_{b\rightarrow -\infty}P_n(bqx;a,b;q)=p_n(x;a|q).
\end{equation}

\subsubsection*{Big $q$-Laguerre $\rightarrow$ Al-Salam-Carlitz~I}
If we set $x\rightarrow aqx$ and $b\rightarrow ab$ in the definition
(\ref{DefBigqLaguerre}) of the big $q$-Laguerre polynomials and take the
limit $a\rightarrow 0$ we obtain the Al-Salam-Carlitz~I polynomials given by
(\ref{DefAlSalamCarlitzI}):
\begin{equation}
\lim_{a\rightarrow 0}\frac{P_n(aqx;a,ab;q)}{a^n}=q^nU_n^{(b)}(x;q).
\end{equation}

\subsubsection*{Big $q$-Laguerre $\rightarrow$ Laguerre}
The Laguerre polynomials given by (\ref{DefLaguerre}) can be obtained
from the big $q$-Laguerre polynomials by the substitution $a=q^{\alpha}$ and
$b=(1-q)^{-1}q^{\beta}$ in the definition (\ref{DefBigqLaguerre}) of the big
$q$-Laguerre polynomials and the limit $q\rightarrow 1$:
\begin{equation}
\lim_{q\rightarrow 1}P_n(x;q^{\alpha},(1-q)^{-1}q^{\beta};q)=
\frac{L_n^{(\alpha)}(x-1)}{L_n^{(\alpha)}(0)}.
\end{equation}

\subsection*{Remark}
The big $q$-Laguerre polynomials given by (\ref{DefBigqLaguerre}) and the
affine $q$-Krawtchouk polynomials given by (\ref{DefAffqKrawtchouk}) are
related in the following way:
$$K_n^{Aff}(q^{-x};p,N;q)=P_n(q^{-x};p,q^{-N-1};q).$$

\subsection*{References}
\cite{AlSalam90}, \cite{AlSalamVerma88}, \cite{AtakAtakKlimyk}, \cite{DattaGriffin},
\cite{IsmailLibis}.


\section{Little $q$-Jacobi}
\index{Little q-Jacobi polynomials@Little $q$-Jacobi polynomials}
\index{q-Jacobi polynomials@$q$-Jacobi polynomials!Little}
\par\setcounter{equation}{0}

\subsection*{Basic hypergeometric representation}
\begin{equation}
\label{DefLittleqJacobi}
p_n(x;a,b|q)=\qhyp{2}{1}{q^{-n},abq^{n+1}}{aq}{qx}.
\end{equation}

\subsection*{Orthogonality relation}
For $0<aq<1$ and $bq<1$ we have
\begin{eqnarray}
\label{OrtLittleqJacobi}
& &\sum_{k=0}^{\infty}\frac{(bq;q)_k}{(q;q)_k}(aq)^kp_m(q^k;a,b|q)p_n(q^k;a,b|q)\nonumber\\
& &{}=\frac{(abq^2;q)_{\infty}}{(aq;q)_{\infty}}\frac{(1-abq)(aq)^n}{(1-abq^{2n+1})}
\frac{(q,bq;q)_n}{(aq,abq;q)_n}\,\delta_{mn}.
\end{eqnarray}

\subsection*{Recurrence relation}
\begin{eqnarray}
\label{RecLittleqJacobi}
-xp_n(x;a,b|q)&=&A_np_{n+1}(x;a,b|q)-\left(A_n+C_n\right)p_n(x;a,b|q)\nonumber\\
& &{}\mathindent{}+C_np_{n-1}(x;a,b|q),
\end{eqnarray}
where
$$\left\{\begin{array}{l}
\displaystyle A_n=q^n\frac{(1-aq^{n+1})(1-abq^{n+1})}{(1-abq^{2n+1})(1-abq^{2n+2})}\\
\\
\displaystyle C_n=aq^n\frac{(1-q^n)(1-bq^n)}{(1-abq^{2n})(1-abq^{2n+1})}.
\end{array}\right.$$

\subsection*{Normalized recurrence relation}
\begin{equation}
\label{NormRecLittleqJacobi}
xp_n(x)=p_{n+1}(x)+(A_n+C_n)p_n(x)+A_{n-1}C_np_{n-1}(x),\end{equation}
where
$$p_n(x;a,b|q)=\frac{(-1)^nq^{-\binom{n}{2}}(abq^{n+1};q)_n}{(aq;q)_n}p_n(x).$$

\subsection*{$q$-Difference equation}
\begin{eqnarray}
\label{dvLittleqJacobi}
& &q^{-n}(1-q^n)(1-abq^{n+1})xy(x)\nonumber\\
& &{}=B(x)y(qx)-\left[B(x)+D(x)\right]y(x)+D(x)y(q^{-1}x),
\end{eqnarray}
where
$$y(x)=p_n(x;a,b|q)$$
and
$$\left\{\begin{array}{l}\displaystyle B(x)=a(bqx-1)\\
\\
\displaystyle D(x)=x-1.\end{array}\right.$$

\subsection*{Forward shift operator}
\begin{eqnarray}
\label{shift1LittleqJacobiI}
& &p_n(x;a,b|q)-p_n(qx;a,b|q)\nonumber\\
& &{}=-\frac{q^{-n+1}(1-q^n)(1-abq^{n+1})}{(1-aq)}xp_{n-1}(x;aq,bq|q)
\end{eqnarray}
or equivalently
\begin{equation}
\label{shift1LittleqJacobiII}
\mathcal{D}_qp_n(x;a,b|q)=-\frac{q^{-n+1}(1-q^n)(1-abq^{n+1})}
{(1-q)(1-aq)}p_{n-1}(x;aq,bq|q).
\end{equation}

\subsection*{Backward shift operator}
\begin{eqnarray}
\label{shift2LittleqJacobiI}
& &a(bx-1)p_n(x;a,b|q)-(x-1)p_n(q^{-1}x;a,b|q)\nonumber\\
& &{}=(1-a)p_{n+1}(x;aq^{-1},bq^{-1}|q)
\end{eqnarray}
or equivalently
\begin{eqnarray}
\label{shift2LittleqJacobiII}
& &\mathcal{D}_{q^{-1}}\left[w(x;\alpha,\beta|q)p_n(x;q^{\alpha},q^{\beta}|q)\right]\nonumber\\
& &{}=\frac{1-q^{\alpha}}{q^{\alpha-1}(1-q)}w(x;\alpha-1,\beta-1|q)p_{n+1}(x;q^{\alpha-1},q^{\beta-1}|q),
\end{eqnarray}
where
$$w(x;\alpha,\beta|q)=\frac{(qx;q)_{\infty}}{(q^{\beta+1}x;q)_{\infty}}x^{\alpha}.$$

\subsection*{Rodrigues-type formula}
\begin{eqnarray}
\label{RodLittleqJacobi}
& &w(x;\alpha,\beta|q)p_n(x;q^{\alpha},q^{\beta}|q)\nonumber\\
& &{}=\frac{q^{n\alpha+\binom{n}{2}}(1-q)^n}{(q^{\alpha+1};q)_n}
\left(\mathcal{D}_{q^{-1}}\right)^n\left[w(x;\alpha+n,\beta+n|q)\right].
\end{eqnarray}

\subsection*{Generating function}
\begin{equation}
\label{GenLittleqJacobi}
\qhyp{0}{1}{-}{aq}{aqxt}\,\qhyp{2}{1}{x^{-1},0}{bq}{xt}=\sum_{n=0}^{\infty}
\frac{(-1)^nq^{\binom{n}{2}}}{(bq,q;q)_n}p_n(x;a,b|q)t^n.
\end{equation}

\subsection*{Limit relations}

\subsubsection*{Big $q$-Jacobi $\rightarrow$ Little $q$-Jacobi}
The little $q$-Jacobi polynomials given by (\ref{DefLittleqJacobi}) can be obtained
from the big $q$-Jacobi polynomials by the substitution $x\rightarrow cqx$ in the definition
(\ref{DefBigqJacobi}) and then by the limit $c\rightarrow -\infty$:
$$\lim_{c\rightarrow -\infty}P_n(cqx;a,b,c;q)=p_n(x;a,b|q).$$

\subsubsection*{$q$-Hahn $\rightarrow$ Little $q$-Jacobi}
If we set $x\rightarrow N-x$ in the definition (\ref{DefqHahn}) of the $q$-Hahn
polynomials and take the limit $N\rightarrow\infty$ we find the little $q$-Jacobi polynomials:
$$\lim _{N\rightarrow\infty} Q_n(q^{x-N};\alpha,\beta,N|q)=p_n(q^x;\alpha,\beta|q),$$
where $p_n(q^x;\alpha,\beta|q)$ is given by (\ref{DefLittleqJacobi}).

\subsubsection*{Little $q$-Jacobi $\rightarrow$ Little $q$-Laguerre / Wall}
The little $q$-Laguerre (or Wall) polynomials given by (\ref{DefLittleqLaguerre})
are little $q$-Jacobi polynomials with $b=0$. So if we set $b=0$ in the
definition (\ref{DefLittleqJacobi}) of the little $q$-Jacobi polynomials we
obtain the little $q$-Laguerre (or Wall) polynomials:
\begin{equation}
p_n(x;a,0|q)=p_n(x;a|q).
\end{equation}

\subsubsection*{Little $q$-Jacobi $\rightarrow$ $q$-Laguerre}
If we substitute $a=q^{\alpha}$ and $x\rightarrow -b^{-1}q^{-1}x$ in the definition
(\ref{DefLittleqJacobi}) of the little $q$-Jacobi polynomials and then take the limit
$b\rightarrow -\infty$ we find the $q$-Laguerre polynomials given by (\ref{DefqLaguerre}):
\begin{equation}
\lim_{b\rightarrow -\infty}p_n(-b^{-1}q^{-1}x;q^{\alpha},b|q)=
\frac{(q;q)_n}{(q^{\alpha+1};q)_n}L_n^{(\alpha)}(x;q).
\end{equation}

\subsubsection*{Little $q$-Jacobi $\rightarrow$ $q$-Bessel}
If we set $b\rightarrow -a^{-1}q^{-1}b$ in the definition (\ref{DefLittleqJacobi}) of
the little $q$-Jacobi polynomials and then take the limit $a\rightarrow 0$ we obtain
the $q$-Bessel polynomials given by (\ref{DefqBessel}):
\begin{equation}
\lim_{a\rightarrow 0}p_n(x;a,-a^{-1}q^{-1}b|q)=y_n(x;b;q).
\end{equation}

\subsubsection*{Little $q$-Jacobi $\rightarrow$ Jacobi / Laguerre}
The Jacobi polynomials given by (\ref{DefJacobi}) simply follow from
the little $q$-Jacobi polynomials given by (\ref{DefLittleqJacobi}) in the following way:
\begin{equation}
\lim_{q\rightarrow 1}p_n(x;q^{\alpha},q^{\beta}|q)=\frac{P_n^{(\alpha,\beta)}(1-2x)}
{P_n^{(\alpha,\beta)}(1)}.
\end{equation}

If we take $a=q^{\alpha}$, $b=-q^{\beta}$ for arbitrary real $\beta$ and
$x\rightarrow\frac{1}{2}(1-q)x$ in the definition (\ref{DefLittleqJacobi})
of the little $q$-Jacobi polynomials and then take the limit $q\rightarrow 1$
we obtain the Laguerre polynomials given by (\ref{DefLaguerre}):
\begin{equation}
\lim_{q\rightarrow 1}p_n(\textstyle\frac{1}{2}(1-q)x;q^{\alpha},-q^{\beta}|q)
=\frac{L_n^{(\alpha)}(x)}{L_n^{(\alpha)}(0)}.
\end{equation}

\subsection*{Remarks} 
The little $q$-Jacobi polynomials given by
(\ref{DefLittleqJacobi}) and the big $q$-Jacobi polynomials given by
(\ref{DefBigqJacobi}) are related in the following way:
$$p_n(x;a,b|q)=\frac{(bq;q)_n}{(aq;q)_n}(-1)^nb^{-n}q^{-n-\binom{n}{2}}
P_n(bqx;b,a,0;q).$$

\noindent
The little $q$-Jacobi polynomials and the $q$-Meixner polynomials given by (\ref{DefqMeixner})
are related in the following way:
$$M_n(q^{-x};b,c;q)=p_n(-c^{-1}q^n;b,b^{-1}q^{-n-x-1}|q).$$

\subsection*{References}
\cite{NAlSalam89}, \cite{AlSalam90}, \cite{AlSalamIsmail77}, \cite{AlSalamIsmail83},
\cite{AndrewsAskey77}, \cite{AndrewsAskey85}, \cite{Askey89I}, \cite{AtakKlimyk2004},
\cite{AtakRahmanSuslov}, \cite{DattaGriffin}, \cite{Floris96}, \cite{Floris97},
\cite{FlorisKoelink}, \cite{GasperRahman84}, \cite{GasperRahman90}, \cite{GrunbaumHaine96},
\cite{Hahn}, \cite{Ismail86I}, \cite{IsmailMassonRahman}, \cite{IsmailWilson}, \cite{Kadell},
\cite{Koelink96I}, \cite{KoelinkKoorn}, \cite{Koorn89III}, \cite{Koorn90II},
\cite{Koorn91}, \cite{Koorn93}, \cite{Masuda+91}, \cite{Miller89}, \cite{Nikiforov+},
\cite{Rahman82}, \cite{Srivastava82}, \cite{SrivastavaJain90}, \cite{Stanton80I}.


\section*{Special case}

\subsection{Little $q$-Legendre}
\index{Little q-Legendre polynomials@Little $q$-Legendre polynomials}
\index{q-Legendre polynomials@$q$-Legendre polynomials!Little}
\par

\subsection*{Basic hypergeometric representation} The little $q$-Legendre polynomials are little $q$-Jacobi
polynomials with $a=b=1$:
\begin{equation}
\label{DefLittleqLegendre}
p_n(x|q)=\qhyp{2}{1}{q^{-n},q^{n+1}}{q}{qx}.
\end{equation}

\subsection*{Orthogonality relation}
\begin{eqnarray}
\label{OrtLittleqLegendre}
\int_0^1p_m(x|q)p_n(x|q)\,d_qx&=&(1-q)\sum_{k=0}^{\infty}q^kp_m(q^k|q)p_n(q^k|q)\nonumber\\
&=&\frac{(1-q)q^n}{(1-q^{2n+1})}\,\delta_{mn}.
\end{eqnarray}

\subsection*{Recurrence relation}
\begin{equation}
\label{RecLittleqLegendre}
-xp_n(x|q)=A_np_{n+1}(x|q)-\left(A_n+C_n\right)p_n(x|q)+C_np_{n-1}(x|q),
\end{equation}
where
$$\left\{\begin{array}{l}
\displaystyle A_n=q^n\frac{(1-q^{n+1})}{(1+q^{n+1})(1-q^{2n+1})}\\
\\
\displaystyle C_n=q^n\frac{(1-q^n)}{(1+q^n)(1-q^{2n+1})}.
\end{array}\right.$$

\subsection*{Normalized recurrence relation}
\begin{equation}
\label{NormRecLittleqLegendre}
xp_n(x)=p_{n+1}(x)+(A_n+C_n)p_n(x)+A_{n-1}C_np_{n-1}(x),
\end{equation}
where
$$p_n(x|q)=\frac{(-1)^nq^{-\binom{n}{2}}(q^{n+1};q)_n}{(q;q)_n}p_n(x).$$

\subsection*{$q$-Difference equation}
\begin{eqnarray}
\label{dvLittleqLegendre}
& &q^{-n}(1-q^n)(1-q^{n+1})xy(x)\nonumber\\
& &{}=B(x)y(qx)-\left[B(x)+D(x)\right]y(x)+D(x)y(q^{-1}x),
\end{eqnarray}
where
$$y(x)=p_n(x|q)$$
and
$$\left\{\begin{array}{l}\displaystyle B(x)=qx-1\\
\\
\displaystyle D(x)=x-1.\end{array}\right.$$

\subsection*{Rodrigues-type formula}
\begin{equation}
\label{RodLittleqLegendre}
p_n(x|q)=\frac{q^{\binom{n}{2}}(1-q)^n}{(q;q)_n}
\left(\mathcal{D}_{q^{-1}}\right)^n\left[(qx;q)_nx^n\right].
\end{equation}

\subsection*{Generating function}
\begin{equation}
\label{GenLittleqLegendre}
\qhyp{0}{1}{-}{q}{qxt}\,\qhyp{2}{1}{x^{-1},0}{q}{xt}=\sum_{n=0}^{\infty}
\frac{(-1)^nq^{\binom{n}{2}}}{(q,q;q)_n}p_n(x|q)t^n.
\end{equation}

\subsection*{Limit relation}

\subsubsection*{Little $q$-Legendre $\rightarrow$ Legendre / Spherical}
If we take the limit $q\rightarrow 1$ in the definition
(\ref{DefLittleqLegendre}) of the little $q$-Legendre polynomials we simply
find the Legendre (or spherical) polynomials given by (\ref{DefLegendre}):
\begin{equation}
\lim_{q\rightarrow 1}p_n(x|q)=P_n(1-2x).
\end{equation}

\subsection*{References}
\cite{Koorn90II}, \cite{Koorn91}, \cite{Rahman89}, \cite{VanAsscheKoorn}.


\section{$q$-Meixner}\index{q-Meixner polynomials@$q$-Meixner polynomials}
\par\setcounter{equation}{0}

\subsection*{Basic hypergeometric representation}
\begin{equation}
\label{DefqMeixner}
M_n(q^{-x};b,c;q)=\qhyp{2}{1}{q^{-n},q^{-x}}{bq}{-\frac{q^{n+1}}{c}}.
\end{equation}

\newpage

\subsection*{Orthogonality relation}
\begin{eqnarray}
\label{OrtqMeixner}
& &\sum_{x=0}^{\infty}\frac{(bq;q)_x}{(q,-bcq;q)_x}c^xq^{\binom{x}{2}}M_m(q^{-x};b,c;q)M_n(q^{-x};b,c;q)\nonumber\\
& &{}=\frac{(-c;q)_{\infty}}{(-bcq;q)_{\infty}}\frac{(q,-c^{-1}q;q)_n}{(bq;q)_n}q^{-n}\,\delta_{mn},
\quad 0\leq bq<1,\quad c>0.
\end{eqnarray}

\subsection*{Recurrence relation}
\begin{eqnarray}
\label{RecqMeixner}
& &q^{2n+1}(1-q^{-x})M_n(q^{-x})\nonumber\\
& &{}=c(1-bq^{n+1})M_{n+1}(q^{-x})\nonumber\\
& &{}\mathindent{}-\left[c(1-bq^{n+1})+q(1-q^n)(c+q^n)\right]M_n(q^{-x})\nonumber\\
& &{}\mathindent\mathindent{}+q(1-q^n)(c+q^n)M_{n-1}(q^{-x}),
\end{eqnarray}
where
$$M_n(q^{-x}):=M_n(q^{-x};b,c;q).$$

\subsection*{Normalized recurrence relation}
\begin{eqnarray}
\label{NormRecqMeixner}
xp_n(x)&=&p_{n+1}(x)+
\left[1+q^{-2n-1}\left\{c(1-bq^{n+1})+q(1-q^n)(c+q^n)\right\}\right]p_n(x)\nonumber\\
& &{}\mathindent{}+cq^{-4n+1}(1-q^n)(1-bq^n)(c+q^n)p_{n-1}(x),
\end{eqnarray}
where
$$M_n(q^{-x};b,c;q)=\frac{(-1)^nq^{n^2}}{(bq;q)_nc^n}p_n(q^{-x}).$$

\subsection*{$q$-Difference equation}
\begin{equation}
\label{dvqMeixner}
-(1-q^n)y(x)=B(x)y(x+1)-\left[B(x)+D(x)\right]y(x)+D(x)y(x-1),
\end{equation}
where
$$y(x)=M_n(q^{-x};b,c;q)$$
and
$$\left\{\begin{array}{l}\displaystyle B(x)=cq^x(1-bq^{x+1})\\
\\
\displaystyle D(x)=(1-q^x)(1+bcq^x).\end{array}\right.$$

\subsection*{Forward shift operator}
\begin{eqnarray}
\label{shift1qMeixnerI}
& &M_n(q^{-x-1};b,c;q)-M_n(q^{-x};b,c;q)\nonumber\\
& &{}=-\frac{q^{-x}(1-q^n)}{c(1-bq)}M_{n-1}(q^{-x};bq,cq^{-1};q)
\end{eqnarray}
or equivalently
\begin{equation}
\label{shift1qMeixnerII}
\frac{\Delta M_n(q^{-x};b,c;q)}{\Delta q^{-x}}
=-\frac{q(1-q^n)}{c(1-q)(1-bq)}M_{n-1}(q^{-x};bq,cq^{-1};q).
\end{equation}

\subsection*{Backward shift operator}
\begin{eqnarray}
\label{shift2qMeixnerI}
& &cq^x(1-bq^x)M_n(q^{-x};b,c;q)-(1-q^x)(1+bcq^x)M_n(q^{-x+1};b,c;q)\nonumber\\
& &{}=cq^x(1-b)M_{n+1}(q^{-x};bq^{-1},cq;q)
\end{eqnarray}
or equivalently
\begin{eqnarray}
\label{shift2qMeixnerII}
& &\frac{\nabla\left[w(x;b,c;q)M_n(q^{-x};b,c;q)\right]}{\nabla q^{-x}}\nonumber\\
& &{}=\frac{1}{1-q}w(x;bq^{-1},cq;q)M_{n+1}(q^{-x};bq^{-1},cq;q),
\end{eqnarray}
where
$$w(x;b,c;q)=\frac{(bq;q)_x}{(q,-bcq;q)_x}c^xq^{\binom{x+1}{2}}.$$

\subsection*{Rodrigues-type formula}
\begin{equation}
\label{RodqMeixner}
w(x;b,c;q)M_n(q^{-x};b,c;q)=(1-q)^n\left(\nabla_q\right)^n\left[w(x;bq^n,cq^{-n};q)\right],
\end{equation}
where
$$\nabla_q:=\frac{\nabla}{\nabla q^{-x}}.$$

\subsection*{Generating functions}
\begin{equation}
\label{GenqMeixner1}
\frac{1}{(t;q)_{\infty}}\,\qhyp{1}{1}{q^{-x}}{bq}{-c^{-1}qt}
=\sum_{n=0}^{\infty}\frac{M_n(q^{-x};b,c;q)}{(q;q)_n}t^n.
\end{equation}

\begin{eqnarray}
\label{GenqMeixner2}
& &\frac{1}{(t;q)_{\infty}}\,\qhyp{1}{1}{-b^{-1}c^{-1}q^{-x}}{-c^{-1}q}{bqt}\nonumber\\
& &{}=\sum_{n=0}^{\infty}\frac{(bq;q)_n}{(-c^{-1}q,q;q)_n}M_n(q^{-x};b,c;q)t^n.
\end{eqnarray}

\subsection*{Limit relations}

\subsubsection*{Big $q$-Jacobi $\rightarrow$ $q$-Meixner}
If we set $b=-a^{-1}cd^{-1}$ (with $d>0$) in the definition (\ref{DefBigqJacobi})
of the big $q$-Jacobi polynomials and take the limit $c\rightarrow -\infty$ we
obtain the $q$-Meixner polynomials given by (\ref{DefqMeixner}):
\begin{equation}
\lim_{c\rightarrow -\infty}P_n(q^{-x};a,-a^{-1}cd^{-1},c;q)=M_n(q^{-x};a,d;q).
\end{equation}

\subsubsection*{$q$-Hahn $\rightarrow$ $q$-Meixner}
The $q$-Meixner polynomials given by (\ref{DefqMeixner}) can be obtained from the $q$-Hahn polynomials
by setting $\alpha=b$ and $\beta=-b^{-1}c^{-1}q^{-N-1}$ in the definition (\ref{DefqHahn}) of the
$q$-Hahn polynomials and letting $N\rightarrow\infty$:
$$\lim_{N\rightarrow\infty}Q_n(q^{-x};b,-b^{-1}c^{-1}q^{-N-1},N|q)=M_n(q^{-x};b,c;q).$$

\subsubsection*{$q$-Meixner $\rightarrow$ $q$-Laguerre}
The $q$-Laguerre polynomials given by (\ref{DefqLaguerre}) can be obtained
from the $q$-Meixner polynomials given by (\ref{DefqMeixner}) by setting
$b=q^{\alpha}$ and $q^{-x}\rightarrow cq^{\alpha}x$ in the definition
(\ref{DefqMeixner}) of the $q$-Meixner polynomials and then taking the limit
$c\rightarrow\infty$:
\begin{equation}
\lim_{c\rightarrow\infty}M_n(cq^{\alpha}x;q^{\alpha},c;q)=
\frac{(q;q)_n}{(q^{\alpha+1};q)_n}L_n^{(\alpha)}(x;q).
\end{equation}

\subsubsection*{$q$-Meixner $\rightarrow$ $q$-Charlier}
The $q$-Charlier polynomials given by (\ref{DefqCharlier}) can easily be
obtained from the $q$-Meixner given by (\ref{DefqMeixner}) by setting $b=0$
in the definition (\ref{DefqMeixner}) of the $q$-Meixner polynomials:
\begin{equation}
M_n(x;0,c;q)=C_n(x;c;q).
\end{equation}

\subsubsection*{$q$-Meixner $\rightarrow$ Al-Salam-Carlitz~II}
The Al-Salam-Carlitz~II polynomials given by (\ref{DefAlSalamCarlitzII})
can be obtained from the $q$-Meixner polynomials given by (\ref{DefqMeixner})
by setting $b=-ac^{-1}$ in the definition (\ref{DefqMeixner}) of the
$q$-Meixner polynomials and then taking the limit $c\rightarrow 0$:
\begin{equation}
\lim_{c\rightarrow 0}M_n(x;-ac^{-1},c;q)=
\left(-\frac{1}{a}\right)^nq^{\binom{n}{2}}V_n^{(a)}(x;q).
\end{equation}

\subsubsection*{$q$-Meixner $\rightarrow$ Meixner}
To find the Meixner polynomials given by (\ref{DefMeixner}) from the $q$-Meixner polynomials given by
(\ref{DefqMeixner}) we set $b=q^{\beta-1}$ and $c\rightarrow (1-c)^{-1}c$ and let $q\rightarrow 1$:
\begin{equation}
\lim_{q\rightarrow 1}M_n(q^{-x};q^{\beta-1},(1-c)^{-1}c;q)=M_n(x;\beta,c).
\end{equation}

\subsection*{Remarks}
The $q$-Meixner polynomials given by (\ref{DefqMeixner}) and the little
$q$-Jacobi polynomials given by (\ref{DefLittleqJacobi}) are related in the
following way:
$$M_n(q^{-x};b,c;q)=p_n(-c^{-1}q^n;b,b^{-1}q^{-n-x-1}|q).$$

\noindent
The $q$-Meixner polynomials and the quantum $q$-Krawtchouk polynomials
given by (\ref{DefQuantumqKrawtchouk}) are related in the following way:
$$K_n^{qtm}(q^{-x};p,N;q)=M_n(q^{-x};q^{-N-1},-p^{-1};q).$$

\subsection*{References}
\cite{AlSalam90}, \cite{AlSalamVerma82II}, \cite{AlSalamVerma88}, \cite{AlvarezRonveaux},
\cite{AtakAtakKlimyk}, \cite{AtakRahmanSuslov}, \cite{Campigotto+}, \cite{GasperRahman90},
\cite{Hahn}, \cite{Ismail2005I}, \cite{Nikiforov+}, \cite{Smirnov}.


\section{Quantum $q$-Krawtchouk}
\index{Quantum q-Krawtchouk polynomials@Quantum $q$-Krawtchouk polynomials}
\index{q-Krawtchouk polynomials@$q$-Krawtchouk polynomials!Quantum}
\par\setcounter{equation}{0}

\subsection*{Basic hypergeometric representation}
\begin{equation}
\label{DefQuantumqKrawtchouk}
K_n^{qtm}(q^{-x};p,N;q)=
\qhyp{2}{1}{q^{-n},q^{-x}}{q^{-N}}{pq^{n+1}},\quad n=0,1,2,\ldots,N.
\end{equation}

\subsection*{Orthogonality relation}
\begin{eqnarray}
\label{OrtQuantumqKrawtchouk}
& &\sum_{x=0}^N\frac{(pq;q)_{N-x}}{(q;q)_x(q;q)_{N-x}}(-1)^{N-x}q^{\binom{x}{2}}
K_m^{qtm}(q^{-x};p,N;q)K_n^{qtm}(q^{-x};p,N;q)\nonumber\\
& &{}=\frac{(-1)^np^N(q;q)_{N-n}(q,pq;q)_n}{(q,q;q)_N}
q^{\binom{N+1}{2}-\binom{n+1}{2}+Nn}\,\delta_{mn},\quad p>q^{-N}.
\end{eqnarray}

\subsection*{Recurrence relation}
\begin{eqnarray}
\label{RecQuantumqKrawtchouk}
& &-pq^{2n+1}(1-q^{-x})K_n^{qtm}(q^{-x})\nonumber\\
& &{}=(1-q^{n-N})K_{n+1}^{qtm}(q^{-x})\nonumber\\
& &{}\mathindent{}-\left[(1-q^{n-N})+q(1-q^n)(1-pq^n)\right]K_n^{qtm}(q^{-x})\nonumber\\
& &{}\mathindent\mathindent{}+q(1-q^n)(1-pq^n)K_{n-1}^{qtm}(q^{-x}),
\end{eqnarray}
where
$$K_n^{qtm}(q^{-x}):=K_n^{qtm}(q^{-x};p,N;q).$$

\subsection*{Normalized recurrence relation}
\begin{eqnarray}
\label{NormRecQuantumqKrawtchouk}
xp_n(x)&=&p_{n+1}(x)+
\left[1-p^{-1}q^{-2n-1}\left\{(1-q^{n-N})+q(1-q^n)(1-pq^n)\right\}\right]p_n(x)\nonumber\\
& &{}\mathindent{}+p^{-2}q^{-4n+1}(1-q^n)(1-pq^n)(1-q^{n-N-1})p_{n-1}(x),
\end{eqnarray}
where
$$K_n^{qtm}(q^{-x};p,N;q)=\frac{p^nq^{n^2}}{(q^{-N};q)_n}p_n(q^{-x}).$$

\subsection*{$q$-Difference equation}
\begin{equation}
\label{dvQuantumqKrawtchouk}
-p(1-q^n)y(x)=B(x)y(x+1)-\left[B(x)+D(x)\right]y(x)+D(x)y(x-1),
\end{equation}
where
$$y(x)=K_n^{qtm}(q^{-x};p,N;q)$$
and
$$\left\{\begin{array}{l}\displaystyle B(x)=-q^x(1-q^{x-N})\\
\\
\displaystyle D(x)=(1-q^x)(p-q^{x-N-1}).\end{array}\right.$$

\subsection*{Forward shift operator}
\begin{eqnarray}
\label{shift1QuantumqKrawtchoukI}
& &K_n^{qtm}(q^{-x-1};p,N;q)-K_n^{qtm}(q^{-x};p,N;q)\nonumber\\
& &{}=\frac{pq^{-x}(1-q^n)}{1-q^{-N}}K_{n-1}^{qtm}(q^{-x};pq,N-1;q)
\end{eqnarray}
or equivalently
\begin{equation}
\label{shift1QuantumqKrawtchoukII}
\frac{\Delta K_n^{qtm}(q^{-x};p,N;q)}{\Delta q^{-x}}=
\frac{pq(1-q^n)}{(1-q)(1-q^{-N})}K_{n-1}^{qtm}(q^{-x};pq,N-1;q).
\end{equation}

\subsection*{Backward shift operator}
\begin{eqnarray}
\label{shift2QuantumqKrawtchoukI}
& &(1-q^{x-N-1})K_n^{qtm}(q^{-x};p,N;q)\nonumber\\
& &{}\mathindent{}+q^{-x}(1-q^x)(p-q^{x-N-1})K_n^{qtm}(q^{-x+1};p,N;q)\nonumber\\
& &{}=(1-q^{-N-1})K_{n+1}^{qtm}(q^{-x};pq^{-1},N+1;q)
\end{eqnarray}
or equivalently
\begin{eqnarray}
\label{shift2QuantumqKrawtchoukII}
& &\frac{\nabla\left[w(x;p,N;q)K_n^{qtm}(q^{-x};p,N;q)\right]}{\nabla q^{-x}}\nonumber\\
& &{}=\frac{1}{1-q}w(x;pq^{-1},N+1;q)K_{n+1}^{qtm}(q^{-x};pq^{-1},N+1;q),
\end{eqnarray}
where
$$w(x;p,N;q)=\frac{(q^{-N};q)_x}{(q,p^{-1}q^{-N};q)_x}(-p)^{-x}q^{\binom{x+1}{2}}.$$

\subsection*{Rodrigues-type formula}
\begin{equation}
\label{RodQuantumqKrawtchouk}
w(x;p,N;q)K_n^{qtm}(q^{-x};p,N;q)=(1-q)^n\left(\nabla_q\right)^n\left[w(x;pq^n,N-n;q)\right],
\end{equation}
where
$$\nabla_q:=\frac{\nabla}{\nabla q^{-x}}.$$

\subsection*{Generating functions} For $x=0,1,2,\ldots,N$ we have
\begin{eqnarray}
\label{GenQuantumqKrawtchouk1}
& &(q^{x-N}t;q)_{N-x}\cdot\qhyp{2}{1}{q^{-x},pq^{N+1-x}}{0}{q^{x-N}t}\nonumber\\
& &{}=\sum_{n=0}^N\frac{(q^{-N};q)_n}{(q;q)_n}K_n^{qtm}(q^{-x};p,N;q)t^n.
\end{eqnarray}

\begin{eqnarray}
\label{GenQuantumqKrawtchouk2}
& &(q^{-x}t;q)_x\cdot\qhyp{2}{1}{q^{x-N},0}{pq}{q^{-x}t}\nonumber\\
& &{}=\sum_{n=0}^N\frac{(q^{-N};q)_n}{(pq,q;q)_n}K_n^{qtm}(q^{-x};p,N;q)t^n.
\end{eqnarray}

\subsection*{Limit relations}

\subsubsection*{$q$-Hahn $\rightarrow$ Quantum $q$-Krawtchouk}
The quantum $q$-Krawtchouk polynomials given by (\ref{DefQuantumqKrawtchouk})
simply follow from the $q$-Hahn polynomials by setting $\beta=p$ in the definition (\ref{DefqHahn}) of
the $q$-Hahn polynomials and taking the limit $\alpha\rightarrow\infty$:
$$\lim_{\alpha\rightarrow\infty}Q_n(q^{-x};\alpha,p,N|q)=K_n^{qtm}(q^{-x};p,N;q).$$

\subsubsection*{Quantum $q$-Krawtchouk $\rightarrow$ Al-Salam-Carlitz~II}
If we set $p=a^{-1}q^{-N-1}$ in the definition (\ref{DefQuantumqKrawtchouk})
of the quantum $q$-Krawtchouk polynomials and let $N\rightarrow\infty$ we
obtain the Al-Salam-Carlitz~II polynomials given by
(\ref{DefAlSalamCarlitzII}). In fact we have
\begin{equation}
\lim_{N\rightarrow\infty}K_n^{qtm}(x;a^{-1}q^{-N-1},N;q)=
\left(-\frac{1}{a}\right)^nq^{\binom{n}{2}}V_n^{(a)}(x;q).
\end{equation}

\subsubsection*{Quantum $q$-Krawtchouk $\rightarrow$ Krawtchouk}
The Krawtchouk polynomials given by (\ref{DefKrawtchouk}) easily follow from
the quantum $q$-Krawtchouk polynomials given by (\ref{DefQuantumqKrawtchouk})
in the following way:
\begin{equation}
\lim_{q\rightarrow 1}K_n^{qtm}(q^{-x};p,N;q)=K_n(x;p^{-1},N).
\end{equation}

\subsection*{Remarks}
The quantum $q$-Krawtchouk polynomials given by
(\ref{DefQuantumqKrawtchouk}) and the $q$-Meixner polynomials given by
(\ref{DefqMeixner}) are related in the following way:
$$K_n^{qtm}(q^{-x};p,N;q)=M_n(q^{-x};q^{-N-1},-p^{-1};q).$$

\noindent
The quantum $q$-Krawtchouk polynomials are related to the affine
$q$-Krawtchouk polynomials given by (\ref{DefAffqKrawtchouk})
by the transformation $q\leftrightarrow q^{-1}$ in the following way:
$$K_n^{qtm}(q^x;p,N;q^{-1})=(p^{-1}q;q)_n\left(-\frac{p}{q}\right)^nq^{-\binom{n}{2}}
K_n^{Aff}(q^{x-N};p^{-1},N;q).$$

\subsection*{References}
\cite{GasperRahman90}, \cite{Koorn89III}, \cite{Koorn90II}, \cite{Smirnov}.


\section{$q$-Krawtchouk}\index{q-Krawtchouk polynomials@$q$-Krawtchouk polynomials}
\par\setcounter{equation}{0}

\subsection*{Basic hypergeometric representation} For $n=0,1,2,\ldots,N$ we have
\begin{eqnarray}
\label{DefqKrawtchouk}
K_n(q^{-x};p,N;q)&=&\qhyp{3}{2}{q^{-n},q^{-x},-pq^n}{q^{-N},0}{q}\\
&=&\frac{(q^{x-N};q)_n}{(q^{-N};q)_nq^{nx}}\,
\qhyp{2}{1}{q^{-n},q^{-x}}{q^{N-x-n+1}}{-pq^{n+N+1}}.\nonumber
\end{eqnarray}

\subsection*{Orthogonality relation}
\begin{eqnarray}
\label{OrtqKrawtchouk}
& &\sum_{x=0}^N\frac{(q^{-N};q)_x}{(q;q)_x}(-p)^{-x}K_m(q^{-x};p,N;q)K_n(q^{-x};p,N;q)\nonumber\\
& &{}=\frac{(q,-pq^{N+1};q)_n}{(-p,q^{-N};q)_n}\frac{(1+p)}{(1+pq^{2n})}\nonumber\\
& &{}\mathindent{}\times (-pq;q)_Np^{-N}q^{-\binom{N+1}{2}}
\left(-pq^{-N}\right)^nq^{n^2}\,\delta_{mn},\quad p>0.
\end{eqnarray}

\subsection*{Recurrence relation}
\begin{eqnarray}
\label{RecqKrawtchouk}
-\left(1-q^{-x}\right)K_n(q^{-x})&=&A_nK_{n+1}(q^{-x})-\left(A_n+C_n\right)K_n(q^{-x})\nonumber\\
& &{}\mathindent{}+C_nK_{n-1}(q^{-x}),
\end{eqnarray}
where
$$K_n(q^{-x}):=K_n(q^{-x};p,N;q)$$
and
$$\left\{\begin{array}{l}
\displaystyle A_n=\frac{(1-q^{n-N})(1+pq^n)}{(1+pq^{2n})(1+pq^{2n+1})}\\
\\
\displaystyle C_n=-pq^{2n-N-1}\frac{(1+pq^{n+N})(1-q^n)}{(1+pq^{2n-1})(1+pq^{2n})}.
\end{array}\right.$$

\subsection*{Normalized recurrence relation}
\begin{equation}
\label{NormRecqKrawtchouk}
xp_n(x)=p_{n+1}(x)+\left[1-(A_n+C_n)\right]p_n(x)+A_{n-1}C_np_{n-1}(x),
\end{equation}
where
$$K_n(q^{-x};p,N;q)=\frac{(-pq^n;q)_n}{(q^{-N};q)_n}p_n(q^{-x}).$$

\subsection*{$q$-Difference equation}
\begin{eqnarray}
\label{dvqKrawtchouk}
& &q^{-n}(1-q^n)(1+pq^n)y(x)\nonumber\\
& &{}=(1-q^{x-N})y(x+1)-\left[(1-q^{x-N})-p(1-q^x)\right]y(x)\nonumber\\
& &{}\mathindent{}-p(1-q^x)y(x-1),
\end{eqnarray}
where
$$y(x)=K_n(q^{-x};p,N;q).$$

\subsection*{Forward shift operator}
\begin{eqnarray}
\label{shift1qKrawtchoukI}
& &K_n(q^{-x-1};p,N;q)-K_n(q^{-x};p,N;q)\nonumber\\
& &{}=\frac{q^{-n-x}(1-q^n)(1+pq^n)}{1-q^{-N}}K_{n-1}(q^{-x};pq^2,N-1;q)
\end{eqnarray}
or equivalently
\begin{equation}
\label{shift1qKrawtchoukII}
\frac{\Delta K_n(q^{-x};p,N;q)}{\Delta q^{-x}}=
\frac{q^{-n+1}(1-q^n)(1+pq^n)}{(1-q)(1-q^{-N})}K_{n-1}(q^{-x};pq^2,N-1;q).
\end{equation}

\subsection*{Backward shift operator}
\begin{eqnarray}
\label{shift2qKrawtchoukI}
& &(1-q^{x-N-1})K_n(q^{-x};p,N;q)+pq^{-1}(1-q^x)K_n(q^{-x+1};p,N;q)\nonumber\\
& &{}=q^x(1-q^{-N-1})K_{n+1}(q^{-x};pq^{-2},N+1;q)
\end{eqnarray}
or equivalently
\begin{eqnarray}
\label{shift2qKrawtchoukII}
& &\frac{\nabla\left[w(x;p,N;q)K_n(q^{-x};p,N;q)\right]}{\nabla q^{-x}}\nonumber\\
& &{}=\frac{1}{1-q}w(x;pq^{-2},N+1;q)K_{n+1}(q^{-x};pq^{-2},N+1;q),
\end{eqnarray}
where
$$w(x;p,N;q)=\frac{(q^{-N};q)_x}{(q;q)_x}\left(-\frac{q}{p}\right)^x.$$

\subsection*{Rodrigues-type formula}
\begin{equation}
\label{RodqKrawtchouk}
w(x;p,N;q)K_n(q^{-x};p,N;q)=(1-q)^n\left(\nabla_q\right)^n\left[w(x;pq^{2n},N-n;q)\right],
\end{equation}
where
$$\nabla_q:=\frac{\nabla}{\nabla q^{-x}}.$$

\subsection*{Generating function} For $x=0,1,2,\ldots,N$ we have
\begin{eqnarray}
\label{GenqKrawtchouk}
& &\qhyp{1}{1}{q^{-x}}{0}{pqt}\,\qhyp{2}{0}{q^{x-N},0}{-}{-q^{-x}t}\nonumber\\
& &{}=\sum_{n=0}^N\frac{(q^{-N};q)_n}{(q;q)_n}q^{-\binom{n}{2}}K_n(q^{-x};p,N;q)t^n.
\end{eqnarray}

\subsection*{Limit relations}

\subsubsection*{$q$-Racah $\rightarrow$ $q$-Krawtchouk}
The $q$-Krawtchouk polynomials given by (\ref{DefqKrawtchouk}) can be obtained from
the $q$-Racah polynomials by setting $\alpha q=q^{-N}$, $\beta=-pq^N$ and
$\gamma=\delta=0$ in the definition (\ref{DefqRacah}) of the $q$-Racah polynomials:
$$R_n(q^{-x};q^{-N-1},-pq^N,0,0|q)=K_n(q^{-x};p,N;q).$$
Note that $\mu(x)=q^{-x}$ in this case.

\subsubsection*{$q$-Hahn $\rightarrow$ $q$-Krawtchouk}
If we set $\beta=-\alpha^{-1}q^{-1}p$ in the definition (\ref{DefqHahn}) of the $q$-Hahn polynomials
and then let $\alpha\rightarrow 0$ we obtain the $q$-Krawtchouk polynomials given by
(\ref{DefqKrawtchouk}):
$$\lim_{\alpha\rightarrow 0}
Q_n(q^{-x};\alpha,-\alpha^{-1}q^{-1}p,N|q)=K_n(q^{-x};p,N;q).$$

\subsubsection*{$q$-Krawtchouk $\rightarrow$ $q$-Bessel}
If we set $x\rightarrow N-x$ in the definition (\ref{DefqKrawtchouk}) of the
$q$-Krawtchouk polynomials and then take the limit $N\rightarrow\infty$ we
obtain the $q$-Bessel polynomials given by (\ref{DefqBessel}):
\begin{equation}
\lim_{N\rightarrow\infty}K_n(q^{x-N};p,N;q)=y_n(q^x;p;q).
\end{equation}

\subsubsection*{$q$-Krawtchouk $\rightarrow$ $q$-Charlier}
By setting $p=a^{-1}q^{-N}$ in the definition (\ref{DefqKrawtchouk}) of the
$q$-Krawtchouk polynomials and then taking the limit $N\rightarrow\infty$ we
obtain the $q$-Charlier polynomials given by (\ref{DefqCharlier}):
\begin{equation}
\lim_{N\rightarrow\infty}K_n(q^{-x};a^{-1}q^{-N},N;q)=C_n(q^{-x};a;q).
\end{equation}

\subsubsection*{$q$-Krawtchouk $\rightarrow$ Krawtchouk}
If we take the limit $q\rightarrow 1$ in the definition (\ref{DefqKrawtchouk}) of the $q$-Krawtchouk
polynomials we simply find the Krawtchouk polynomials given by (\ref{DefKrawtchouk}) in the
following way:
\begin{equation}
\lim_{q\rightarrow 1}K_n(q^{-x};p,N;q)=K_n(x;(p+1)^{-1},N).
\end{equation}

\subsection*{Remark}
The $q$-Krawtchouk polynomials given by (\ref{DefqKrawtchouk}) and the
dual $q$-Krawtchouk polynomials given by (\ref{DefDualqKrawtchouk}) are
related in the following way:
$$K_n(q^{-x};p,N;q)=K_x(\lambda(n);-pq^N,N|q)$$
with
$$\lambda(n)=q^{-n}-pq^n$$
or
$$K_n(\lambda(x);c,N|q)=K_x(q^{-n};-cq^{-N},N;q)$$
with
$$\lambda(x)=q^{-x}+cq^{x-N}.$$

\subsection*{References}
\cite{AlvarezRonveaux}, \cite{AskeyWilson79}, \cite{AtakRahmanSuslov},
\cite{Campigotto+}, \cite{GasperRahman90}, \cite{Nikiforov+},
\cite{NoumiMimachi91}, \cite{Stanton80III}, \cite{Stanton84}.


\newpage

\section{Affine $q$-Krawtchouk}
\index{Affine q-Krawtchouk polynomials@Affine $q$-Krawtchouk polynomials}
\index{q-Krawtchouk polynomials@$q$-Krawtchouk polynomials!Affine}
\par\setcounter{equation}{0}

\subsection*{Basic hypergeometric representation}
\begin{eqnarray}
\label{DefAffqKrawtchouk}
K_n^{Aff}(q^{-x};p,N;q)&=&\qhyp{3}{2}{q^{-n},0,q^{-x}}{pq,q^{-N}}{q}\\
&=&\frac{(-pq)^nq^{\binom{n}{2}}}{(pq;q)_n}\,
\qhyp{2}{1}{q^{-n},q^{x-N}}{q^{-N}}{\frac{q^{-x}}{p}},\quad n=0,1,2,\ldots,N.\nonumber
\end{eqnarray}

\subsection*{Orthogonality relation}
\begin{eqnarray}
\label{OrtAffqKrawtchouk}
& &\sum_{x=0}^N\frac{(pq;q)_x(q;q)_N}{(q;q)_x(q;q)_{N-x}}(pq)^{-x}K_m^{Aff}(q^{-x};p,N;q)K_n^{Aff}(q^{-x};p,N;q)\nonumber\\
& &{}=(pq)^{n-N}\frac{(q;q)_n(q;q)_{N-n}}{(pq;q)_n(q;q)_N}\,\delta_{mn},\quad 0<pq<1.
\end{eqnarray}

\subsection*{Recurrence relation}
\begin{eqnarray}
\label{RecAffqKrawtchouk}
& &-(1-q^{-x})K_n^{Aff}(q^{-x})\nonumber\\
& &{}=(1-q^{n-N})(1-pq^{n+1})K_{n+1}^{Aff}(q^{-x})\nonumber\\
& &{}\mathindent{}-\left[(1-q^{n-N})(1-pq^{n+1})-pq^{n-N}(1-q^n)\right]K_n^{Aff}(q^{-x})\nonumber\\
& &{}\mathindent\mathindent{}-pq^{n-N}(1-q^n)K_{n-1}^{Aff}(q^{-x}),
\end{eqnarray}
where
$$K_n^{Aff}(q^{-x}):=K_n^{Aff}(q^{-x};p,N;q).$$

\subsection*{Normalized recurrence relation}
\begin{eqnarray}
\label{NormRecAffqKrawtchouk}
xp_n(x)&=&p_{n+1}(x)+
\left[1-\left\{(1-q^{n-N})(1-pq^{n+1})-pq^{n-N}(1-q^n)\right\}\right]p_n(x)\nonumber\\
& &{}\mathindent{}-pq^{n-N}(1-q^n)(1-pq^n)(1-q^{n-N-1})p_{n-1}(x),
\end{eqnarray}
where
$$K_n^{Aff}(q^{-x};p,N;q)=\frac{1}{(pq,q^{-N};q)_n}p_n(q^{-x}).$$

\subsection*{$q$-Difference equation}
\begin{equation}
\label{dvAffqKrawtchouk}
q^{-n}(1-q^n)y(x)=B(x)y(x+1)-\left[B(x)+D(x)\right]y(x)+D(x)y(x-1),
\end{equation}
where
$$y(x)=K_n^{Aff}(q^{-x};p,N;q)$$
and
$$\left\{\begin{array}{l}\displaystyle B(x)=(1-q^{x-N})(1-pq^{x+1})\\
\\
\displaystyle D(x)=-p(1-q^x)q^{x-N}.\end{array}\right.$$

\subsection*{Forward shift operator}
\begin{eqnarray}
\label{shift1AffqKrawtchoukI}
& &K_n^{Aff}(q^{-x-1};p,N;q)-K_n^{Aff}(q^{-x};p,N;q)\nonumber\\
& &{}=\frac{q^{-n-x}(1-q^n)}{(1-pq)(1-q^{-N})}K_{n-1}^{Aff}(q^{-x};pq,N-1;q)
\end{eqnarray}
or equivalently
\begin{eqnarray}
\label{shift1AffqKrawtchoukII}
& &\frac{\Delta K_n^{Aff}(q^{-x};p,N;q)}{\Delta q^{-x}}\nonumber\\
& &{}=\frac{q^{-n+1}(1-q^n)}{(1-q)(1-pq)(1-q^{-N})}K_{n-1}^{Aff}(q^{-x};pq,N-1;q).
\end{eqnarray}

\subsection*{Backward shift operator}
\begin{eqnarray}
\label{shift2AffqKrawtchoukI}
& &(1-pq^x)(1-q^{-x+N+1})K_n^{Aff}(q^{-x};p,N;q)-p(1-q^x)K_n^{Aff}(q^{-x+1};p,N;q)\nonumber\\
& &{}=(1-p)(1-q^{N+1})K_{n+1}^{Aff}(q^{-x};pq^{-1},N+1;q)
\end{eqnarray}
or equivalently
\begin{eqnarray}
\label{shift2AffqKrawtchoukII}
& &\frac{\nabla\left[w(x;p,N;q)K_n^{Aff}(q^{-x};p,N;q)\right]}{\nabla q^{-x}}\nonumber\\
& &{}=\frac{1-q^{N+1}}{1-q}w(x;pq^{-1},N+1;q)K_{n+1}^{Aff}(q^{-x};pq^{-1},N+1;q),
\end{eqnarray}
where
$$w(x;p,N;q)=\frac{(pq;q)_x}{(q;q)_x(q;q)_{N-x}}p^{-x}.$$

\subsection*{Rodrigues-type formula}
\begin{eqnarray}
\label{RodAffqKrawtchouk}
& &w(x;p,N;q)K_n^{Aff}(q^{-x};p,N;q)\nonumber\\
& &{}=\frac{(-1)^nq^{-Nn+\binom{n}{2}}(1-q)^n}{(q^{-N};q)_n}\left(\nabla_q\right)^n\left[w(x;pq^n,N-n;q)\right],
\end{eqnarray}
where
$$\nabla_q:=\frac{\nabla}{\nabla q^{-x}}.$$

\subsection*{Generating functions} For $x=0,1,2,\ldots,N$ we have
\begin{equation}
\label{GenAffqKrawtchouk1}
(q^{-N}t;q)_{N-x}\cdot\qhyp{1}{1}{q^{-x}}{pq}{pqt}=\sum_{n=0}^N
\frac{(q^{-N};q)_n}{(q;q)_n}K_n^{Aff}(q^{-x};p,N;q)t^n.
\end{equation}

\begin{eqnarray}
\label{GenAffqKrawtchouk2}
& &(-pq^{-N+1}t;q)_x\cdot\qhyp{2}{0}{q^{x-N},pq^{x+1}}{-}{-q^{-x}t}\nonumber\\
& &{}=\sum_{n=0}^N\frac{(pq,q^{-N};q)_n}{(q;q)_n}q^{-\binom{n}{2}}K_n^{Aff}(q^{-x};p,N;q)t^n.
\end{eqnarray}

\subsection*{Limit relations}

\subsubsection*{$q$-Hahn $\rightarrow$ Affine $q$-Krawtchouk}
The affine $q$-Krawtchouk polynomials given by (\ref{DefAffqKrawtchouk})
can be obtained from the $q$-Hahn polynomials by the substitution $\alpha=p$ and
$\beta=0$ in (\ref{DefqHahn}):
$$Q_n(q^{-x};p,0,N|q)=K_n^{Aff}(q^{-x};p,N;q).$$

\subsubsection*{Dual $q$-Hahn $\rightarrow$ Affine $q$-Krawtchouk}
The affine $q$-Krawtchouk polynomials given by (\ref{DefAffqKrawtchouk})
can be obtained from the dual $q$-Hahn polynomials by the substitution $\gamma=p$
and $\delta=0$ in (\ref{DefDualqHahn}):
$$R_n(\mu(x);p,0,N|q)=K_n^{Aff}(q^{-x};p,N;q).$$
Note that $\mu(x)=q^{-x}$ in this case.

\subsubsection*{Affine $q$-Krawtchouk $\rightarrow$ Little $q$-Laguerre / Wall}
If we set $x\rightarrow N-x$ in the definition (\ref{DefAffqKrawtchouk})
of the affine $q$-Krawtchouk polynomials and take the limit
$N\rightarrow\infty$ we simply obtain the little $q$-Laguerre (or Wall)
polynomials given by (\ref{DefLittleqLaguerre}):
%HSC changed p_n(q^x;p;q) to p_n(q^x;p|q)
\begin{equation}
\lim_{N\rightarrow\infty}K_n^{Aff}(q^{x-N};p,N;q)=p_n(q^x;p|q).
\end{equation}

\subsubsection*{Affine $q$-Krawtchouk $\rightarrow$ Krawtchouk}
If we let $q\rightarrow 1$ in the definition (\ref{DefAffqKrawtchouk}) of the affine
$q$-Krawtchouk polynomials we obtain:
\begin{equation}
\lim_{q\rightarrow 1}K_n^{Aff}(q^{-x};p,N|q)=K_n(x;1-p,N),
\end{equation}
where $K_n(x;1-p,N)$ is the Krawtchouk polynomial given by (\ref{DefKrawtchouk}).

\subsection*{Remarks}
The affine $q$-Krawtchouk polynomials given by (\ref{DefAffqKrawtchouk})
and the big $q$-Laguerre polynomials given by (\ref{DefBigqLaguerre}) are
related in the following way:
$$K_n^{Aff}(q^{-x};p,N;q)=P_n(q^{-x};p,q^{-N-1};q).$$

\noindent
The affine $q$-Krawtchouk polynomials are related to the quantum
$q$-Krawtchouk polynomials given by (\ref{DefQuantumqKrawtchouk})
by the transformation $q\leftrightarrow q^{-1}$ in the following way:
$$K_n^{Aff}(q^x;p,N;q^{-1})=\frac{1}{(p^{-1}q;q)_n}
K_n^{qtm}(q^{x-N};p^{-1},N;q).$$

\subsection*{References}
\cite{AtakRahmanSuslov}, \cite{LChiharaStanton}, \cite{Delsarte},
\cite{DelsarteGoethals}, \cite{Dunkl78II}, \cite{FlorisKoelink},
\cite{GasperRahman90}, \cite{Stanton84}.


\section{Dual $q$-Krawtchouk}
\index{Dual q-Krawtchouk polynomials@Dual $q$-Krawtchouk polynomials}
\index{q-Krawtchouk polynomials@$q$-Krawtchouk polynomials!Dual}
\par\setcounter{equation}{0}

\subsection*{Basic hypergeometric representation}
\begin{eqnarray}
\label{DefDualqKrawtchouk}
K_n(\lambda(x);c,N|q)&=&\qhyp{3}{2}{q^{-n},q^{-x},cq^{x-N}}{q^{-N},0}{q}\\
&=&\frac{(q^{x-N};q)_n}{(q^{-N};q)_nq^{nx}}\,
\qhyp{2}{1}{q^{-n},q^{-x}}{q^{N-x-n+1}}{cq^{x+1}},\quad n=0,1,2,\ldots,N,\nonumber
\end{eqnarray}
where
$$\lambda(x):=q^{-x}+cq^{x-N}.$$

\subsection*{Orthogonality relation}
\begin{eqnarray}
\label{OrtDualqKrawtchouk}
& &\sum_{x=0}^N\frac{(cq^{-N},q^{-N};q)_x}{(q,cq;q)_x}
\frac{(1-cq^{2x-N})}{(1-cq^{-N})}c^{-x}q^{x(2N-x)}K_m(\lambda(x))K_n(\lambda(x))\nonumber\\
& &{}=(c^{-1};q)_N\frac{(q;q)_n}{(q^{-N};q)_n}(cq^{-N})^n\,\delta_{mn},\quad c<0,
\end{eqnarray}
where
$$K_n(\lambda(x)):=K_n(\lambda(x);c,N|q).$$

\newpage

\subsection*{Recurrence relation}
\begin{eqnarray}
\label{RecDualqKrawtchouk}
& &-(1-q^{-x})(1-cq^{x-N})K_n(\lambda(x))\nonumber\\
& &{}=(1-q^{n-N})K_{n+1}(\lambda(x))\nonumber\\
& &{}\mathindent{}-\left[(1-q^{n-N})+cq^{-N}(1-q^n)\right]K_n(\lambda(x))\nonumber\\
& &{}\mathindent\mathindent{}+cq^{-N}(1-q^n)K_{n-1}(\lambda(x)),
\end{eqnarray}
where
$$K_n(\lambda(x)):=K_n(\lambda(x);c,N|q).$$

\subsection*{Normalized recurrence relation}
\begin{eqnarray}
\label{NormRecDualqKrawtchouk}
xp_n(x)&=&p_{n+1}(x)+(1+c)q^{n-N}p_n(x)\nonumber\\
& &{}\mathindent{}+cq^{-N}(1-q^n)(1-q^{n-N-1})p_{n-1}(x),
\end{eqnarray}
where
$$K_n(\lambda(x);c,N|q)=\frac{1}{(q^{-N};q)_n}p_n(\lambda(x)).$$

\subsection*{$q$-Difference equation}
\begin{equation}
\label{dvDualqKrawtchouk}
q^{-n}(1-q^n)y(x)=B(x)y(x+1)-\left[B(x)+D(x)\right]y(x)+D(x)y(x-1),
\end{equation}
where
$$y(x)=K_n(\lambda(x);c,N|q)$$
and
$$\left\{\begin{array}{l}\displaystyle B(x)=\frac{(1-q^{x-N})(1-cq^{x-N})}{(1-cq^{2x-N})(1-cq^{2x-N+1})}\\
\\
\displaystyle D(x)=cq^{2x-2N-1}\frac{(1-q^x)(1-cq^x)}{(1-cq^{2x-N-1})(1-cq^{2x-N})}.\end{array}\right.$$

\newpage

\subsection*{Forward shift operator}
\begin{eqnarray}
\label{shift1DualqKrawtchoukI}
& &K_n(\lambda(x+1);c,N|q)-K_n(\lambda(x);c,N|q)\nonumber\\
& &{}=\frac{q^{-n-x}(1-q^n)(1-cq^{2x-N+1})}{1-q^{-N}}
K_{n-1}(\lambda(x);c,N-1|q)
\end{eqnarray}
or equivalently
\begin{equation}
\label{shift1DualqKrawtchoukII}
\frac{\Delta K_n(\lambda(x);c,N|q)}{\Delta\lambda(x)}=
\frac{q^{-n+1}(1-q^n)}{(1-q)(1-q^{-N})}K_{n-1}(\lambda(x);c,N-1|q).
\end{equation}

\subsection*{Backward shift operator}
\begin{eqnarray}
\label{shift2DualqKrawtchoukI}
& &(1-q^{x-N-1})(1-cq^{x-N-1})K_n(\lambda(x);c,N|q)\nonumber\\
& &{}\mathindent{}-cq^{2(x-N-1)}(1-q^x)(1-cq^x)K_n(\lambda(x-1);c,N|q)\nonumber\\
& &{}=q^x(1-q^{-N-1})(1-cq^{2x-N-1})K_{n+1}(\lambda(x);c,N+1|q)
\end{eqnarray}
or equivalently
\begin{eqnarray}
\label{shift2DualqKrawtchoukII}
& &\frac{\nabla\left[w(x;c,N|q)K_n(\lambda(x);c,N|q)\right]}{\nabla\lambda(x)}\nonumber\\
& &{}=\frac{1}{(1-q)(1-cq^{-N-1})}w(x;c,N+1|q)K_{n+1}(\lambda(x);c,N+1|q),
\end{eqnarray}
where
$$w(x;c,N|q)=\frac{(q^{-N},cq^{-N};q)_x}{(q,cq;q)_x}c^{-x}q^{2Nx-x(x-1)}.$$

\subsection*{Rodrigues-type formula}
\begin{eqnarray}
\label{RodDualqKrawtchouk}
& &w(x;c,N|q)K_n(\lambda(x);c,N|q)\nonumber\\
& &{}=(1-q)^n(cq^{-N};q)_n\left(\nabla_{\lambda}\right)^n\left[w(x;c,N-n|q)\right],
\end{eqnarray}
where
$$\nabla_{\lambda}:=\frac{\nabla}{\nabla\lambda(x)}.$$

\subsection*{Generating function} For $x=0,1,2,\ldots,N$ we have
\begin{equation}
\label{GenDualqKrawtchouk}
(cq^{-N}t;q)_x\cdot (q^{-N}t;q)_{N-x}=\sum_{n=0}^N
\frac{(q^{-N};q)_n}{(q;q)_n}K_n(\lambda(x);c,N|q)t^n.
\end{equation}

\subsection*{Limit relations}

\subsubsection*{$q$-Racah $\rightarrow$ Dual $q$-Krawtchouk}
The dual $q$-Krawtchouk polynomials given by (\ref{DefDualqKrawtchouk}) easily
follow from the $q$-Racah polynomials given by (\ref{DefqRacah}) by using the
substitutions $\alpha=\beta=0$, $\gamma q=q^{-N}$ and $\delta=c$:
$$R_n(\mu(x);0,0,q^{-N-1},c|q)=K_n(\lambda(x);c,N|q).$$
Note that
$$\mu(x)=\lambda(x)=q^{-x}+cq^{x-N}.$$

\subsubsection*{Dual $q$-Hahn $\rightarrow$ Dual $q$-Krawtchouk}
The dual $q$-Krawtchouk polynomials given by (\ref{DefDualqKrawtchouk}) can
be obtained from the dual $q$-Hahn polynomials by setting $\delta=c\gamma^{-1}q^{-N-1}$
in (\ref{DefDualqHahn}) and letting $\gamma\rightarrow 0$:
$$\lim_{\gamma\rightarrow 0}
R_n(\mu(x);\gamma,c\gamma^{-1}q^{-N-1},N|q)=K_n(\lambda(x);c,N|q).$$

\subsubsection*{Dual $q$-Krawtchouk $\rightarrow$ Al-Salam-Carlitz~I}
If we set $c=a^{-1}$ in the definition (\ref{DefDualqKrawtchouk})
of the dual $q$-Krawtchouk polynomials and take the limit
$N\rightarrow\infty$ we simply obtain the Al-Salam-Carlitz~I polynomials
given by (\ref{DefAlSalamCarlitzI}):
\begin{equation}
\lim_{N\rightarrow\infty}K_n(\lambda(x);a^{-1},N|q)=
\left(-\frac{1}{a}\right)^nq^{-\binom{n}{2}}U_n^{(a)}(q^x;q).
\end{equation}
Note that $\lambda(x)=q^{-x}+a^{-1}q^{x-N}$.

\subsubsection*{Dual $q$-Krawtchouk $\rightarrow$ Krawtchouk}
If we set $c=1-p^{-1}$ in the definition (\ref{DefDualqKrawtchouk}) of the
dual $q$-Krawtchouk polynomials and take the limit $q\rightarrow 1$ we simply find
the Krawtchouk polynomials given by (\ref{DefKrawtchouk}):
\begin{equation}
\lim_{q\rightarrow 1}K_n(\lambda(x);1-p^{-1},N|q)=K_n(x;p,N).
\end{equation}

\subsection*{Remark}
The dual $q$-Krawtchouk polynomials given by (\ref{DefDualqKrawtchouk})
and the $q$-Krawtchouk polynomials given by (\ref{DefqKrawtchouk}) are
related in the following way:
$$K_n(q^{-x};p,N;q)=K_x(\lambda(n);-pq^N,N|q)$$
with
$$\lambda(n)=q^{-n}-pq^n$$
or
$$K_n(\lambda(x);c,N|q)=K_x(q^{-n};-cq^{-N},N;q)$$
with
$$\lambda(x)=q^{-x}+cq^{x-N}.$$

\subsection*{References}
\cite{LChiharaStanton}, \cite{Koelink96I}, \cite{Koorn90II}, \cite{Koorn93}.


\section{Continuous big $q$-Hermite}
\index{Continuous big q-Hermite polynomials@Continuous big $q$-Hermite polynomials}
\index{Big q-Hermite polynomials@Big $q$-Hermite polynomials!Continuous}
\index{q-Hermite polynomials@$q$-Hermite polynomials!Continuous big}
\par\setcounter{equation}{0}

\subsection*{Basic hypergeometric representation}
\begin{eqnarray}
\label{DefContBigqHermite}
H_n(x;a|q)&=&a^{-n}\,\qhyp{3}{2}{q^{-n},a\e^{i\theta},a\e^{-i\theta}}{0,0}{q}\\
&=&\e^{in\theta}\,\qhyp{2}{0}{q^{-n},a\e^{i\theta}}{-}{q^n\e^{-2i\theta}},
\quad x=\cos\theta.\nonumber
\end{eqnarray}

\subsection*{Orthogonality relation}
If $a$ is real and $|a|<1$, then we have the following orthogonality relation
\begin{equation}
\label{OrtContBigqHermite1}
\frac{1}{2\pi}\int_{-1}^1\frac{w(x)}{\sqrt{1-x^2}}H_m(x;a|q)H_n(x;a|q)\,dx
=\frac{\,\delta_{mn}}{(q^{n+1};q)_{\infty}},
\end{equation}
where
$$w(x):=w(x;a|q)=\left|\frac{(\e^{2i\theta};q)_{\infty}}
{(a\e^{i\theta};q)_{\infty}}\right|^2=
\frac{h(x,1)h(x,-1)h(x,q^{\frac{1}{2}})h(x,-q^{\frac{1}{2}})}{h(x,a)},$$
with
$$h(x,\alpha):=\prod_{k=0}^{\infty}\left(1-2\alpha xq^k+\alpha^2q^{2k}\right)
=\left(\alpha\e^{i\theta},\alpha\e^{-i\theta};q\right)_{\infty},\quad x=\cos\theta.$$
If $a>1$, then we have another orthogonality relation given by:
\begin{eqnarray}
\label{OrtContBigqHermite2}
& &\frac{1}{2\pi}\int_{-1}^1\frac{w(x)}{\sqrt{1-x^2}}H_m(x;a|q)H_n(x;a|q)\,dx\nonumber\\
& &{}\mathindent{}+\sum_{\begin{array}{c}\scriptstyle k\\ \scriptstyle 1<aq^k\leq a\end{array}}
w_kH_m(x_k;a|q)H_n(x_k;a|q)=\frac{\,\delta_{mn}}{(q^{n+1};q)_{\infty}},
\end{eqnarray}
where $w(x)$ is as before,
$$x_k=\frac{aq^k+\left(aq^k\right)^{-1}}{2}$$
and
$$w_k=\frac{(a^{-2};q)_{\infty}}{(q;q)_{\infty}}
\frac{(1-a^2q^{2k})(a^2;q)_k}{(1-a^2)(q;q)_k}
q^{-\frac{3}{2}k^2-\frac{1}{2}k}\left(-\frac{1}{a^4}\right)^k.$$

\subsection*{Recurrence relation}
\begin{equation}
\label{RecContBigqHermite}
2xH_n(x;a|q)=H_{n+1}(x;a|q)+aq^nH_n(x;a|q)+(1-q^n)H_{n-1}(x;a|q).
\end{equation}

\subsection*{Normalized recurrence relation}
\begin{equation}
\label{NormRecContBigqHermite}
xp_n(x)=p_{n+1}(x)+\frac{1}{2}aq^np_n(x)+\frac{1}{4}(1-q^n)p_{n-1}(x),
\end{equation}
where
$$H_n(x;a|q)=2^np_n(x).$$

\subsection*{$q$-Difference equations}
\begin{equation}
\label{dvContBigqHermite1}
(1-q)^2D_q\left[{\tilde w}(x;aq^{\frac{1}{2}}|q)D_qy(x)\right]
+4q^{-n+1}(1-q^n){\tilde w}(x;a|q)y(x)=0,
\end{equation}
where
$$y(x)=H_n(x;a|q)$$
and
$${\tilde w}(x;a|q):=\frac{w(x;a|q)}{\sqrt{1-x^2}}.$$
If we define
$$P_n(z):=a^{-n}\,\qhyp{3}{2}{q^{-n},az,az^{-1}}{0,0}{q}$$
then the $q$-difference equation can also be written in the form
\begin{eqnarray}
\label{dvContBigqHermite2}
q^{-n}(1-q^n)P_n(z)&=&A(z)P_n(qz)-\left[A(z)+A(z^{-1})\right]P_n(z)\nonumber\\
& &{}\mathindent{}+A(z^{-1})P_n(q^{-1}z),
\end{eqnarray}
where
$$A(z)=\frac{(1-az)}{(1-z^2)(1-qz^2)}.$$

\subsection*{Forward shift operator}
\begin{equation}
\label{shift1ContBigqHermiteI}
\delta_qH_n(x;a|q)=-q^{-\frac{1}{2}n}(1-q^n)(\e^{i\theta}-\e^{-i\theta})
H_{n-1}(x;aq^{\frac{1}{2}}|q),\quad x=\cos\theta
\end{equation}
or equivalently
\begin{equation}
\label{shift1ContBigqHermiteII}
D_qH_n(x;a|q)=\frac{2q^{-\frac{1}{2}(n-1)}(1-q^n)}{1-q}
H_{n-1}(x;aq^{\frac{1}{2}}|q).
\end{equation}

\subsection*{Backward shift operator}
\begin{eqnarray}
\label{shift2ContBigqHermiteI}
& &\delta_q\left[{\tilde w}(x;a|q)H_n(x;a|q)\right]\nonumber\\
& &{}=q^{-\frac{1}{2}(n+1)}(\e^{i\theta}-\e^{-i\theta})\nonumber\\
& &{}\mathindent{}\times
{\tilde w}(x;aq^{-\frac{1}{2}}|q)H_{n+1}(x;aq^{-\frac{1}{2}}|q),\quad x=\cos\theta
\end{eqnarray}
or equivalently
\begin{equation}
\label{shift2ContBigqHermiteII}
D_q\left[{\tilde w}(x;a|q)H_n(x;a|q)\right]=
-\frac{2q^{-\frac{1}{2}n}}{1-q}{\tilde w}(x;aq^{-\frac{1}{2}}|q)H_{n+1}(x;aq^{-\frac{1}{2}}|q).
\end{equation}

\subsection*{Rodrigues-type formula}
\begin{equation}
\label{RodContBigqHermite}
w(x;a|q)H_n(x;a|q)=\left(\frac{q-1}{2}\right)^nq^{\frac{1}{4}n(n-1)}
\left(D_q\right)^n\left[w(x;aq^{\frac{1}{2}n}|q)\right].
\end{equation}

\subsection*{Generating functions}
\begin{equation}
\label{GenContBigqHermite1}
\frac{(at;q)_{\infty}}{(\e^{i\theta}t,\e^{-i\theta}t;q)_{\infty}}
=\sum_{n=0}^{\infty}\frac{H_n(x;a|q)}{(q;q)_n}t^n,\quad x=\cos\theta.
\end{equation}

\begin{eqnarray}
\label{GenContBigqHermite2}
& &(\e^{i\theta}t;q)_{\infty}\cdot\qhyp{1}{1}{a\e^{i\theta}}{\e^{i\theta}t}{\e^{-i\theta}t}\nonumber\\
& &{}=\sum_{n=0}^{\infty}\frac{(-1)^nq^{\binom{n}{2}}}{(q;q)_n}H_n(x;a|q)t^n,
\quad x=\cos\theta.
\end{eqnarray}

\begin{eqnarray}
\label{GenContBigqHermite3}
& &\frac{(\gamma\e^{i\theta}t;q)_{\infty}}{(\e^{i\theta}t;q)_{\infty}}\,
\qhyp{2}{1}{\gamma,a\e^{i\theta}}{\gamma\e^{i\theta}t}{\e^{-i\theta}t}\nonumber\\
& &{}=\sum_{n=0}^{\infty}\frac{(\gamma;q)_n}{(q;q)_n}H_n(x;a|q)t^n,
\quad x=\cos\theta,\quad\textrm{$\gamma$ arbitrary}.
\end{eqnarray}

\subsection*{Limit relations}

\subsubsection*{Al-Salam-Chihara $\rightarrow$ Continuous big $q$-Hermite}
If we take $b=0$ in the definition (\ref{DefAlSalamChihara}) of the Al-Salam-Chihara
polynomials we simply obtain the continuous big $q$-Hermite polynomials given by
(\ref{DefContBigqHermite}):
$$Q_n(x;a,0|q)=H_n(x;a|q).$$

\subsubsection*{Continuous big $q$-Hermite $\rightarrow$ Continuous $q$-Hermite}
The continuous $q$-Hermite polynomials given by (\ref{DefContqHermite})
can easily be obtained from the continuous big $q$-Hermite polynomials given by
(\ref{DefContBigqHermite}) by taking $a=0$:
\begin{equation}
H_n(x;0|q)=H_n(x|q).
\end{equation}

\subsubsection*{Continuous big $q$-Hermite $\rightarrow$ Hermite}
If we set $a=0$ and $x\rightarrow x\sqrt{\frac{1}{2}(1-q)}$ in the
definition (\ref{DefContBigqHermite}) of the continuous big $q$-Hermite
polynomials and let $q$ tend to $1$, we obtain the Hermite polynomials given
by (\ref{DefHermite}) in the following way:
\begin{equation}
\lim_{q\rightarrow 1}
\frac{H_n(x\sqrt{\frac{1}{2}(1-q)};0|q)}
{\left(\frac{1-q}{2}\right)^{\frac{n}{2}}}=H_n(x).
\end{equation}

If we take $a\rightarrow a\sqrt{2(1-q)}$ and
$x\rightarrow x\sqrt{\frac{1}{2}(1-q)}$ in the definition
(\ref{DefContBigqHermite}) of the continuous big $q$-Hermite polynomials and
take the limit $q\rightarrow 1$ we find the Hermite polynomials given by
(\ref{DefHermite}) with shifted argument:
\begin{equation}
\lim_{q\rightarrow 1}
\frac{H_n(x\sqrt{\frac{1}{2}(1-q)};a\sqrt{2(1-q)}|q)}
{\left(\frac{1-q}{2}\right)^{\frac{n}{2}}}=H_n(x-a).
\end{equation}

\subsection*{References}
\cite{AskeyRahmanSuslov}, \cite{AtakAtakII}, \cite{AtakAtakIII},
\cite{Floreanini+95}.


\newpage

\section{Continuous $q$-Laguerre}
\index{Continuous q-Laguerre polynomials@Continuous $q$-Laguerre polynomials}
\index{q-Laguerre polynomials@$q$-Laguerre polynomials!Continuous}
\par\setcounter{equation}{0}

\subsection*{Basic hypergeometric representation} The continuous $q$-Laguerre 
polynomials can be obtained from the continuous $q$-Jacobi polynomials given 
by (\ref{DefContqJacobi}) by taking the limit $\beta\rightarrow\infty$:
\begin{eqnarray}
\label{DefContqLaguerre}
P_n^{(\alpha)}(x|q)&=&\frac{(q^{\alpha+1};q)_n}{(q;q)_n}\,\qhyp{3}{2}
{q^{-n},q^{\frac{1}{2}\alpha+\frac{1}{4}}\e^{i\theta},q^{\frac{1}{2}\alpha+\frac{1}{4}}\e^{-i\theta}}
{q^{\alpha+1},0}{q}\\
&=&\frac{(q^{\frac{1}{2}\alpha+\frac{3}{4}}\e^{-i\theta};q)_n}{(q;q)_n}
q^{(\frac{1}{2}\alpha+\frac{1}{4})n}\e^{in\theta}\nonumber\\
& &{}\mathindent{}\times\qhyp{2}{1}{q^{-n},q^{\frac{1}{2}\alpha+\frac{1}{4}}\e^{i\theta}}
{q^{-\frac{1}{2}\alpha+\frac{1}{4}-n}\e^{i\theta}}
{q^{-\frac{1}{2}\alpha+\frac{1}{4}}\e^{-i\theta}},\quad x=\cos\theta.\nonumber
\end{eqnarray}

\subsection*{Orthogonality relation}
For $\alpha\geq -\frac{1}{2}$ we have
\begin{eqnarray}
\label{OrtContqLaguerre}
& &\frac{1}{2\pi}\int_{-1}^1\frac{w(x)}{\sqrt{1-x^2}}P_m^{(\alpha)}(x|q)P_n^{(\alpha)}(x|q)\,dx\nonumber\\
& &{}=\frac{1}{(q,q^{\alpha+1};q)_{\infty}}\frac{(q^{\alpha+1};q)_n}{(q;q)_n}
q^{(\alpha+\frac{1}{2})n}\,\delta_{mn},
\end{eqnarray}
where
\begin{eqnarray*} w(x):=w(x;q^{\alpha}|q)&=&\left|\frac{(\e^{2i\theta};q)_{\infty}}
{(q^{\frac{1}{2}\alpha+\frac{1}{4}}\e^{i\theta},
q^{\frac{1}{2}\alpha+\frac{3}{4}}\e^{i\theta};q)_{\infty}}\right|^2
=\left|\frac{(\e^{i\theta},-\e^{i\theta};q^{\frac{1}{2}})_{\infty}}
{(q^{\frac{1}{2}\alpha+\frac{1}{4}}\e^{i\theta};q^{\frac{1}{2}})_{\infty}}\right|^2\\
&=&\frac{h(x,1)h(x,-1)h(x,q^{\frac{1}{2}})h(x,-q^{\frac{1}{2}})}
{h(x,q^{\frac{1}{2}\alpha+\frac{1}{4}})h(x,q^{\frac{1}{2}\alpha+\frac{3}{4}})},
\end{eqnarray*}
with
$$h(x,\alpha):=\prod_{k=0}^{\infty}\left(1-2\alpha xq^k+\alpha^2q^{2k}\right)
=\left(\alpha\e^{i\theta},\alpha\e^{-i\theta};q\right)_{\infty},\quad x=\cos\theta.$$

\newpage

\subsection*{Recurrence relation}
\begin{eqnarray}
\label{RecContqLaguerre}
2xP_n^{(\alpha)}(x|q)&=&q^{-\frac{1}{2}\alpha-\frac{1}{4}}(1-q^{n+1})P_{n+1}^{(\alpha)}(x|q)\nonumber\\
& &{}\mathindent{}+q^{n+\frac{1}{2}\alpha+\frac{1}{4}}(1+q^{\frac{1}{2}})P_n^{(\alpha)}(x|q)\nonumber\\
& &{}\mathindent\mathindent{}+q^{\frac{1}{2}\alpha+\frac{1}{4}}(1-q^{n+\alpha})P_{n-1}^{(\alpha)}(x|q).
\end{eqnarray}

\subsection*{Normalized recurrence relation}
\begin{eqnarray}
\label{NormRecContqLaguerre}
xp_n(x)&=&p_{n+1}(x)+\frac{1}{2}q^{n+\frac{1}{2}\alpha+\frac{1}{4}}(1+q^{\frac{1}{2}})p_n(x)\nonumber\\
& &{}\mathindent{}+\frac{1}{4}(1-q^n)(1-q^{n+\alpha})p_{n-1}(x),
\end{eqnarray}
where
$$P_n^{(\alpha)}(x|q)=\frac{2^nq^{(\frac{1}{2}\alpha+\frac{1}{4})n}}{(q;q)_n}p_n(x).$$

\subsection*{$q$-Difference equations}
\begin{equation}
\label{dvContqLaguerre}
(1-q)^2D_q\left[{\tilde w}(x;q^{\alpha+1}|q)D_qy(x)\right]
+4q^{-n+1}(1-q^n){\tilde w}(x;q^{\alpha}|q)y(x)=0,
\end{equation}
where
$$y(x)=P_n^{(\alpha)}(x|q)$$
and
$${\tilde w}(x;q^{\alpha}|q):=\frac{w(x;q^{\alpha}|q)}{\sqrt{1-x^2}}.$$

\subsection*{Forward shift operator}
\begin{equation}
\label{shift1ContqLaguerreI}
\delta_qP_n^{(\alpha)}(x|q)=-q^{-n+\frac{1}{2}\alpha+\frac{3}{4}}
(\e^{i\theta}-\e^{-i\theta})P_{n-1}^{(\alpha+1)}(x|q),\quad x=\cos\theta
\end{equation}
or equivalently
\begin{equation}
\label{shift1ContqLaguerreII}
D_qP_n^{(\alpha)}(x|q)=\frac{2q^{-n+\frac{1}{2}\alpha+\frac{5}{4}}}{1-q}
P_{n-1}^{(\alpha+1)}(x|q).
\end{equation}

\subsection*{Backward shift operator}
\begin{eqnarray}
\label{shift2ContqLaguerreI}
& &\delta_q\left[{\tilde w}(x;q^{\alpha}|q)P_n^{(\alpha)}(x|q)\right]\nonumber\\
& &{}=q^{-\frac{1}{2}\alpha-\frac{1}{4}}(1-q^{n+1})(\e^{i\theta}-\e^{-i\theta})\nonumber\\
& &{}\mathindent{}\times
{\tilde w}(x;q^{\alpha-1}|q)P_{n+1}^{(\alpha-1)}(x|q),\quad x=\cos\theta
\end{eqnarray}
or equivalently
\begin{eqnarray}
\label{shift2ContqLaguerreII}
& &D_q\left[{\tilde w}(x;q^{\alpha}|q)P_n^{(\alpha)}(x|q)\right]\nonumber\\
& &{}=-2q^{-\frac{1}{2}\alpha+\frac{1}{4}}\frac{1-q^{n+1}}{1-q}
{\tilde w}(x;q^{\alpha-1}|q)P_{n+1}^{(\alpha-1)}(x|q).
\end{eqnarray}

\subsection*{Rodrigues-type formula}
\begin{equation}
\label{RodContqLaguerre}
{\tilde w}(x;q^{\alpha}|q)P_n^{(\alpha)}(x|q)=\left(\frac{q-1}{2}\right)^n
\frac{q^{\frac{1}{4}n^2+\frac{1}{2}n\alpha}}{(q;q)_n}
\left(D_q\right)^n\left[{\tilde w}(x;q^{\alpha+n}|q)\right].
\end{equation}

\subsection*{Generating functions}
\begin{equation}
\label{GenContqLaguerre1}
\frac{(q^{\alpha+\frac{1}{2}}t,q^{\alpha+1}t;q)_{\infty}}
{(q^{\frac{1}{2}\alpha+\frac{1}{4}}\e^{i\theta}t,q^{\frac{1}{2}\alpha+\frac{1}{4}}\e^{-i\theta}t;q)_{\infty}}
=\sum_{n=0}^{\infty}P_n^{(\alpha)}(x|q)t^n,\quad x=\cos\theta.
\end{equation}

\begin{eqnarray}
\label{GenContqLaguerre2}
& &\frac{1}{(\e^{i\theta}t;q)_{\infty}}\,\qhyp{2}{1}{q^{\frac{1}{2}\alpha+\frac{1}{4}}\e^{i\theta},
q^{\frac{1}{2}\alpha+\frac{3}{4}}\e^{i\theta}}{q^{\alpha+1}}{\e^{-i\theta}t}\nonumber\\
& &{}=\sum_{n=0}^{\infty}\frac{P_n^{(\alpha)}(x|q)t^n}{(q^{\alpha+1};q)_nq^{(\frac{1}{2}\alpha+\frac{1}{4})n}},
\quad x=\cos\theta.
\end{eqnarray}

\begin{eqnarray}
\label{GenContqLaguerre3}
& &(t;q)_{\infty}\cdot\qhyp{2}{1}{q^{\frac{1}{2}\alpha+\frac{1}{4}}\e^{i\theta},
q^{\frac{1}{2}\alpha+\frac{1}{4}}\e^{-i\theta}}{q^{\alpha+1}}{t}\nonumber\\
& &{}=\sum_{n=0}^{\infty}\frac{(-1)^nq^{\binom{n}{2}}}{(q^{\alpha+1};q)_n}
P_n^{(\alpha)}(x|q)t^n,\quad x=\cos\theta.
\end{eqnarray}

\begin{eqnarray}
\label{GenContqLaguerre4}
& &\frac{(\gamma\e^{i\theta}t;q)_{\infty}}{(\e^{i\theta}t;q)_{\infty}}\,
\qhyp{3}{2}{\gamma,q^{\frac{1}{2}\alpha+\frac{1}{4}}\e^{i\theta},q^{\frac{1}{2}\alpha+\frac{3}{4}}\e^{i\theta}}{q^{\alpha+1},\gamma\e^{i\theta}t}{\e^{-i\theta}t}\nonumber\\
& &{}=\sum_{n=0}^{\infty}\frac{(\gamma;q)_n}{(q^{\alpha+1};q)_n}\frac{P_n(x|q)}{q^{(\frac{1}{2}\alpha+\frac{1}{4})n}}t^n,
\quad x=\cos\theta,\quad\textrm{$\gamma$ arbitrary}.
\end{eqnarray}

\subsection*{Limit relations}

\subsubsection*{Al-Salam-Chihara $\rightarrow$ Continuous $q$-Laguerre}
The continuous $q$-Laguerre polynomials given by (\ref{DefContqLaguerre})
can be obtained from the Al-Salam-Chihara polynomials given by
(\ref{DefAlSalamChihara}) by taking $a=q^{\frac{1}{2}\alpha+\frac{1}{4}}$ and
$b=q^{\frac{1}{2}\alpha+\frac{3}{4}}$:
$$Q_n(x;q^{\frac{1}{2}\alpha+\frac{1}{4}},q^{\frac{1}{2}\alpha+\frac{3}{4}}|q)
=\frac{(q;q)_n}{q^{(\frac{1}{2}\alpha+\frac{1}{4})n}}P_n^{(\alpha)}(x|q).$$

\subsubsection*{$q$-Meixner-Pollaczek $\rightarrow$ Continuous $q$-Laguerre}
If we take $\e^{i\phi}=q^{-\frac{1}{4}}$, $a=q^{\frac{1}{2}\alpha+\frac{1}{2}}$
and $\e^{i\theta}\rightarrow q^{\frac{1}{4}}\e^{i\theta}$ in the definition
(\ref{DefqMP}) of the $q$-Meixner-Pollaczek polynomials we obtain the
continuous $q$-Laguerre polynomials given by (\ref{DefContqLaguerre}):
$$P_n(\cos(\theta+\phi);q^{\frac{1}{2}\alpha+\frac{1}{2}}|q)=
q^{-(\frac{1}{2}\alpha+\frac{1}{4})n}P_n^{(\alpha)}(\cos\theta|q).$$

\subsubsection*{Continuous $q$-Jacobi $\rightarrow$ Continuous $q$-Laguerre}
The continuous $q$-Laguerre polynomials given by (\ref{DefContqLaguerre})
follow simply from the continuous $q$-Jacobi polynomials given by (\ref{DefContqJacobi})
by taking the limit $\beta\rightarrow\infty$:
$$\lim _{\beta\rightarrow\infty} P_n^{(\alpha,\beta)}(x|q)=P_n^{(\alpha)}(x|q).$$

\subsubsection*{Continuous $q$-Laguerre $\rightarrow$ Continuous $q$-Hermite}
The continuous $q$-Hermite polynomials given by (\ref{DefContqHermite}) can
be obtained from the continuous $q$-Laguerre polynomials given by
(\ref{DefContqLaguerre}) by taking the limit $\alpha\rightarrow\infty$ in the
following way:
\begin{equation}
\lim_{\alpha\rightarrow\infty}
\frac{P_n^{(\alpha)}(x|q)}{q^{(\frac{1}{2}\alpha+\frac{1}{4})n}}
=\frac{H_n(x|q)}{(q;q)_n}.
\end{equation}

\subsubsection*{Continuous $q$-Laguerre $\rightarrow$ Laguerre}
If we set $x\rightarrow q^x$ in the definitions (\ref{DefContqLaguerre}) of the continuous
$q$-Laguerre polynomials and take the limit $q\rightarrow 1$ we find the Laguerre polynomials
given by (\ref{DefLaguerre}). In fact we have:
\begin{equation}
\lim_{q\rightarrow 1}P_n^{(\alpha)}(q^x|q)=L_n^{(\alpha)}(2x).
\end{equation}

\subsection*{Remark} If we let $\beta$ tend to infinity in (\ref{DefContqJacobi2})
and renormalize we obtain
\begin{equation}
\label{DefContqLaguerre2}
P_n^{(\alpha)}(x;q)=\frac{(q^{\alpha+1};q)_n}{(q;q)_n}\,\qhyp{3}{2}
{q^{-n},q^{\frac{1}{2}}\e^{i\theta},q^{\frac{1}{2}}\e^{-i\theta}}
{q^{\alpha+1},-q}{q},\quad x=\cos\theta.
\end{equation}
These two $q$-analogues of the Laguerre polynomials are connected by the
following quadratic transformation:
$$P_n^{(\alpha)}(x|q^2)=q^{n\alpha}P_n^{(\alpha)}(x;q).$$

\subsection*{Reference}
\cite{AskeyWilson85}.


\section{Little $q$-Laguerre / Wall}\index{Wall polynomials}
\index{Little q-Laguerre polynomials@Little $q$-Laguerre polynomials}
\index{q-Laguerre polynomials@$q$-Laguerre polynomials!Little}
\par\setcounter{equation}{0}

\subsection*{Basic hypergeometric representation}
\begin{eqnarray}
\label{DefLittleqLaguerre}
p_n(x;a|q)&=&\qhyp{2}{1}{q^{-n},0}{aq}{qx}\\
&=&\frac{1}{(a^{-1}q^{-n};q)_n}\,\qhyp{2}{0}{q^{-n},x^{-1}}{-}{\frac{x}{a}}.\nonumber
\end{eqnarray}

\newpage

\subsection*{Orthogonality relation}
\begin{equation}
\label{OrtLittleqLaguerre}
\sum_{k=0}^{\infty}\frac{(aq)^k}{(q;q)_k}p_m(q^k;a|q)p_n(q^k;a|q)=
\frac{(aq)^n}{(aq;q)_{\infty}}\frac{(q;q)_n}{(aq;q)_n}\,\delta_{mn},\quad 0<aq<1.
\end{equation}

\subsection*{Recurrence relation}
\begin{equation}
\label{RecLittleqLaguerre}
-xp_n(x;a|q)=A_np_{n+1}(x;a|q)-\left(A_n+C_n\right)p_n(x;a|q)+C_np_{n-1}(x;a|q),
\end{equation}
where
$$\left\{\begin{array}{l}A_n=q^n(1-aq^{n+1})\\
\\
C_n=aq^n(1-q^n).\end{array}\right.$$

\subsection*{Normalized recurrence relation}
\begin{equation}
\label{NormRecLittleqLaguerre}
xp_n(x)=p_{n+1}(x)+(A_n+C_n)p_n(x)+aq^{2n-1}(1-q^n)(1-aq^n)p_{n-1}(x),
\end{equation}
where
$$p_n(x;a|q)=\frac{(-1)^nq^{-\binom{n}{2}}}{(aq;q)_n}p_n(x).$$

\subsection*{$q$-Difference equation}
\begin{equation}
\label{dvLittleqLaguerre}
-q^{-n}(1-q^n)xy(x)=ay(qx)+(x-a-1)y(x)+(1-x)y(q^{-1}x),
\end{equation}
where
$$y(x)=p_n(x;a|q).$$

\subsection*{Forward shift operator}
\begin{equation}
\label{shift1LittleqLaguerreI}
p_n(x;a|q)-p_n(qx;a|q)=-\frac{q^{-n+1}(1-q^n)}{1-aq}xp_{n-1}(x;aq|q)
\end{equation}
or equivalently
\begin{equation}
\label{shift1LittleqLaguerreII}
\mathcal{D}_qp_n(x;a|q)=-\frac{q^{-n+1}(1-q^n)}{(1-q)(1-aq)}
p_{n-1}(x;aq|q).
\end{equation}

\subsection*{Backward shift operator}
\begin{equation}
\label{shift2LittleqLaguerreI}
ap_n(x;a|q)-(1-x)p_n(q^{-1}x;a|q)=(a-1)p_{n+1}(x;q^{-1}a|q)
\end{equation}
or equivalently
\begin{eqnarray}
\label{shift2LittleqLaguerreII}
& &\mathcal{D}_{q^{-1}}\left[w(x;\alpha|q)p_n(x;q^{\alpha}|q)\right]\nonumber\\
& &{}=\frac{1-q^{\alpha}}{q^{\alpha-1}(1-q)}w(x;\alpha-1|q)p_{n+1}(x;q^{\alpha-1}|q),
\end{eqnarray}
where
$$w(x;\alpha|q)=(qx;q)_{\infty}x^{\alpha}.$$

\subsection*{Rodrigues-type formula}
\begin{equation}
\label{RodLittleqLaguerre}
w(x;\alpha|q)p_n(x;q^{\alpha}|q)=\frac{q^{n\alpha+\binom{n}{2}}(1-q)^n}
{(q^{\alpha+1};q)_n}\left(\mathcal{D}_{q^{-1}}\right)^n\left[w(x;\alpha+n|q)\right].
\end{equation}

\subsection*{Generating function}
\begin{equation}
\label{GenLittleqLaguerre}
\frac{(t;q)_{\infty}}{(xt;q)_{\infty}}\,\qhyp{0}{1}{-}{aq}{aqxt}=\sum_{n=0}^{\infty}
\frac{(-1)^nq^{\binom{n}{2}}}{(q;q)_n}p_n(x;a|q)t^n.
\end{equation}

\subsection*{Limit relations}

\subsubsection*{Big $q$-Laguerre $\rightarrow$ Little $q$-Laguerre / Wall}
The little $q$-Laguerre (or Wall) polynomials given by (\ref{DefLittleqLaguerre})
can be obtained from the big $q$-Laguerre polynomials by taking $x\rightarrow bqx$
in (\ref{DefBigqLaguerre}) and letting $b\rightarrow -\infty$:
$$\lim_{b\rightarrow -\infty}P_n(bqx;a,b;q)=p_n(x;a|q).$$

\subsubsection*{Little $q$-Jacobi $\rightarrow$ Little $q$-Laguerre / Wall}
The little $q$-Laguerre (or Wall) polynomials given by (\ref{DefLittleqLaguerre})
are little $q$-Jacobi polynomials with $b=0$. So if we set $b=0$ in the
definition (\ref{DefLittleqJacobi}) of the little $q$-Jacobi polynomials we
obtain the little $q$-Laguerre (or Wall) polynomials:
$$p_n(x;a,0|q)=p_n(x;a|q).$$

\subsubsection*{Affine $q$-Krawtchouk $\rightarrow$ Little $q$-Laguerre / Wall}
If we set $x\rightarrow N-x$ in the definition (\ref{DefAffqKrawtchouk})
of the affine $q$-Krawtchouk polynomials and take the limit
$N\rightarrow\infty$ we simply obtain the little $q$-Laguerre (or Wall)
polynomials given by (\ref{DefLittleqLaguerre}):
%HSC changed p_n(q^x;p;q) to p_n(q^x;p|q)
$$\lim_{N\rightarrow\infty}K_n^{Aff}(q^{x-N};p,N;q)=p_n(q^x;p|q).$$

\subsubsection*{Little $q$-Laguerre / Wall $\rightarrow$ Laguerre / Charlier}
If we set $a=q^{\alpha}$ and $x\rightarrow (1-q)x$ in the definition (\ref{DefLittleqLaguerre})
of the little $q$-Laguerre (or Wall) polynomials and let $q$ tend to $1$, we obtain
the Laguerre polynomials given by (\ref{DefLaguerre}):
\begin{equation}
\lim_{q\rightarrow 1}p_n((1-q)x;q^{\alpha}|q)=\frac{L_n^{(\alpha)}(x)}{L_n^{(\alpha)}(0)}.
\end{equation}

If we set $a\rightarrow (q-1)a$ and $x\rightarrow q^x$ in the definition (\ref{DefLittleqLaguerre})
of the little $q$-Laguerre (or Wall) polynomials and take the limit $q\rightarrow 1$ we obtain
the Charlier polynomials given by (\ref{DefCharlier}) in the following way:
\begin{equation}
\lim_{q\rightarrow 1}\frac{p_n(q^x;(q-1)a|q)}{(1-q)^n}=
\frac{C_n(x;a)}{a^n}.
\end{equation}

\subsection*{Remark} If we set $a=q^{\alpha}$ and replace $q$ by $q^{-1}$ we find the
$q$-Laguerre polynomials given by (\ref{DefqLaguerre}) in the following
way:
$$p_n(x;q^{-\alpha}|q^{-1})=\frac{(q;q)_n}{(q^{\alpha+1};q)_n}L_n^{(\alpha)}(-x;q).$$

\subsection*{References}
\cite{AlSalam90}, \cite{AlSalamVerma82I}, \cite{AtakAtakI},
\cite{Chihara68II}, \cite{Chihara78}, \cite{Chihara79}, \cite{DattaGriffin},
\cite{FlorisKoelink}, \cite{GasperRahman90}, \cite{Koorn90II}, \cite{Koorn91},
\cite{Nikiforov+}, \cite{VanAsscheKoorn}, \cite{Wall}.


\section{$q$-Laguerre}\index{q-Laguerre polynomials@$q$-Laguerre polynomials}
\par\setcounter{equation}{0}

\subsection*{Basic hypergeometric representation}
\begin{eqnarray}
\label{DefqLaguerre}
L_n^{(\alpha)}(x;q)&=&\frac{(q^{\alpha+1};q)_n}{(q;q)_n}\,
\qhyp{1}{1}{q^{-n}}{q^{\alpha+1}}{-q^{n+\alpha+1}x}\\
&=&\frac{1}{(q;q)_n}\,\qhyp{2}{1}{q^{-n},-x}{0}{q^{n+\alpha+1}}.\nonumber
\end{eqnarray}

\subsection*{Orthogonality relation}
The $q$-Laguerre polynomials satisfy two kinds of orthogonality relations, an absolutely
continuous one and a discrete one. These orthogonality relations are given by, respectively:
\begin{eqnarray}
\label{OrtqLaguerre1}
& &\int_{0}^{\infty}\frac{x^{\alpha}}{(-x;q)_{\infty}}L_m^{(\alpha)}(x;q)L_n^{(\alpha)}(x;q)\,dx\nonumber\\
& &{}=\frac{(q^{-\alpha};q)_{\infty}}{(q;q)_{\infty}}
\frac{(q^{\alpha+1};q)_n}{(q;q)_nq^n}\Gamma(-\alpha)\Gamma(\alpha+1)\,\delta_{mn},\quad\alpha>-1
\end{eqnarray}
and
\begin{eqnarray}
\label{OrtqLaguerre2}
& &\sum_{k=-\infty}^{\infty}\frac{q^{k\alpha+k}}{(-cq^k;q)_{\infty}}L_m^{(\alpha)}(cq^k;q)L_n^{(\alpha)}(cq^k;q)\nonumber\\
& &{}=\frac{(q,-cq^{\alpha+1},-c^{-1}q^{-\alpha};q)_{\infty}}
{(q^{\alpha+1},-c,-c^{-1}q;q)_{\infty}}\frac{(q^{\alpha+1};q)_n}{(q;q)_nq^n}\,\delta_{mn},
\quad\alpha>-1,\quad c>0.
\end{eqnarray}
For $c=1$ the latter orthogonality relation can also be written as
\begin{eqnarray}
\label{OrtqLaguerre3}
& &\int_0^{\infty}\frac{x^{\alpha}}{(-x;q)_{\infty}}L_m^{(\alpha)}(x;q)L_n^{(\alpha)}(x;q)\,d_qx\nonumber\\
& &{}=\frac{1-q}{2}\,\frac{(q,-q^{\alpha+1},-q^{-\alpha};q)_{\infty}}
{(q^{\alpha+1},-q,-q;q)_{\infty}}\frac{(q^{\alpha+1};q)_n}{(q;q)_nq^n}\,\delta_{mn},\quad\alpha>-1.
\end{eqnarray}

\newpage

\subsection*{Recurrence relation}
\begin{eqnarray}
\label{RecqLaguerre}
-q^{2n+\alpha+1}xL_n^{(\alpha)}(x;q)&=&(1-q^{n+1})L_{n+1}^{(\alpha)}(x;q)\nonumber\\
& &{}\mathindent{}-\left[(1-q^{n+1})+q(1-q^{n+\alpha})\right]L_n^{(\alpha)}(x;q)\nonumber\\
& &{}\mathindent\mathindent{}+q(1-q^{n+\alpha})L_{n-1}^{(\alpha)}(x;q).
\end{eqnarray}

\subsection*{Normalized recurrence relation}
\begin{eqnarray}
\label{NormRecqLaguerre}
xp_n(x)&=&p_{n+1}(x)+q^{-2n-\alpha-1}\left[(1-q^{n+1})+q(1-q^{n+\alpha})\right]p_n(x)\nonumber\\
& &{}\mathindent{}+q^{-4n-2\alpha+1}(1-q^n)(1-q^{n+\alpha})p_{n-1}(x),
\end{eqnarray}
where
$$L_n^{(\alpha)}(x;q)=\frac{(-1)^nq^{n(n+\alpha)}}{(q;q)_n}p_n(x).$$

\subsection*{$q$-Difference equation}
\begin{equation}
\label{dvqLaguerre}
-q^{\alpha}(1-q^n)xy(x)=q^{\alpha}(1+x)y(qx)-\left[1+q^{\alpha}(1+x)\right]y(x)+y(q^{-1}x),
\end{equation}
where
$$y(x)=L_n^{(\alpha)}(x;q).$$

\subsection*{Forward shift operator}
\begin{equation}
\label{shift1qLaguerreI}
L_n^{(\alpha)}(x;q)-L_n^{(\alpha)}(qx;q)=-q^{\alpha+1}xL_{n-1}^{(\alpha+1)}(qx;q)
\end{equation}
or equivalently
\begin{equation}
\label{shift1qLaguerreII}
\mathcal{D}_qL_n^{(\alpha)}(x;q)=-\frac{q^{\alpha+1}}{1-q}L_{n-1}^{(\alpha+1)}(qx;q).
\end{equation}

\subsection*{Backward shift operator}
\begin{equation}
\label{shift2qLaguerreI}
L_n^{(\alpha)}(x;q)-q^{\alpha}(1+x)L_n^{(\alpha)}(qx;q)=(1-q^{n+1})L_{n+1}^{(\alpha-1)}(x;q)
\end{equation}
or equivalently
\begin{equation}
\label{shift2qLaguerreII}
\mathcal{D}_q\left[w(x;\alpha;q)L_n^{(\alpha)}(x;q)\right]=
\frac{1-q^{n+1}}{1-q}w(x;\alpha-1;q)L_{n+1}^{(\alpha-1)}(x;q),
\end{equation}
where
$$w(x;\alpha;q)=\frac{x^{\alpha}}{(-x;q)_{\infty}}.$$

\subsection*{Rodrigues-type formula}
\begin{equation}
\label{RodqLaguerre}
w(x;\alpha;q)L_n^{(\alpha)}(x;q)=
\frac{(1-q)^n}{(q;q)_n}\left(\mathcal{D}_q\right)^n\left[w(x;\alpha+n;q)\right].
\end{equation}

\subsection*{Generating functions}
\begin{equation}
\label{GenqLaguerre1}
\frac{1}{(t;q)_{\infty}}\,\qhyp{1}{1}{-x}{0}{q^{\alpha+1}t}
=\sum_{n=0}^{\infty}L_n^{(\alpha)}(x;q)t^n.
\end{equation}

\begin{equation}
\label{GenqLaguerre2}
\frac{1}{(t;q)_{\infty}}\,\qhyp{0}{1}{-}{q^{\alpha+1}}{-q^{\alpha+1}xt}
=\sum_{n=0}^{\infty}\frac{L_n^{(\alpha)}(x;q)}{(q^{\alpha+1};q)_n}t^n.
\end{equation}

\begin{equation}
\label{GenqLaguerre3}
(t;q)_{\infty}\cdot\qhyp{0}{2}{-}{q^{\alpha+1},t}{-q^{\alpha+1}xt}
=\sum_{n=0}^{\infty}\frac{(-1)^nq^{\binom{n}{2}}}{(q^{\alpha+1};q)_n}L_n^{(\alpha)}(x;q)t^n.
\end{equation}

\begin{eqnarray}
\label{GenqLaguerre4}
& &\frac{(\gamma t;q)_{\infty}}{(t;q)_{\infty}}\,\qhyp{1}{2}{\gamma}{q^{\alpha+1},\gamma t}{-q^{\alpha+1}xt}\nonumber\\
& &{}=\sum_{n=0}^{\infty}\frac{(\gamma;q)_n}{(q^{\alpha+1};q)_n}L_n^{(\alpha)}(x;q)t^n,
\quad\textrm{$\gamma$ arbitrary}.
\end{eqnarray}

\subsection*{Limit relations}

\subsubsection*{Little $q$-Jacobi $\rightarrow$ $q$-Laguerre}
If we substitute $a=q^{\alpha}$ and $x\rightarrow -b^{-1}q^{-1}x$ in the definition
(\ref{DefLittleqJacobi}) of the little $q$-Jacobi polynomials and then take the limit
$b\rightarrow -\infty$ we find the $q$-Laguerre polynomials given by (\ref{DefqLaguerre}):
$$\lim_{b\rightarrow -\infty}p_n(-b^{-1}q^{-1}x;q^{\alpha},b|q)=
\frac{(q;q)_n}{(q^{\alpha+1};q)_n}L_n^{(\alpha)}(x;q).$$

\subsubsection*{$q$-Meixner $\rightarrow$ $q$-Laguerre}
The $q$-Laguerre polynomials given by (\ref{DefqLaguerre}) can be obtained
from the $q$-Meixner polynomials given by (\ref{DefqMeixner}) by setting
$b=q^{\alpha}$ and $q^{-x}\rightarrow cq^{\alpha}x$ in the definition
(\ref{DefqMeixner}) of the $q$-Meixner polynomials and then taking the limit
$c\rightarrow\infty$:
$$\lim_{c\rightarrow\infty}M_n(cq^{\alpha}x;q^{\alpha},c;q)=
\frac{(q;q)_n}{(q^{\alpha+1};q)_n}L_n^{(\alpha)}(x;q).$$

\subsubsection*{$q$-Laguerre $\rightarrow$ Stieltjes-Wigert}
If we set $x\rightarrow xq^{-\alpha}$ in the definition
(\ref{DefqLaguerre}) of the $q$-Laguerre polynomials and take the limit
$\alpha\rightarrow\infty$ we simply obtain the Stieltjes-Wigert polynomials given
by (\ref{DefStieltjesWigert}):
\begin{equation}
\lim_{\alpha\rightarrow\infty}L_n^{(\alpha)}\left(xq^{-\alpha};q\right)=S_n(x;q).
\end{equation}

\subsubsection*{$q$-Laguerre $\rightarrow$ Laguerre / Charlier}
If we set $x\rightarrow (1-q)x$ in the definition (\ref{DefqLaguerre})
of the $q$-Laguerre polynomials and take the limit $q\rightarrow 1$ we obtain
the Laguerre polynomials given by (\ref{DefLaguerre}):
\begin{equation}
\lim_{q\rightarrow 1}L_n^{(\alpha)}((1-q)x;q)=L_n^{(\alpha)}(x).
\end{equation}

If we set $x\rightarrow -q^{-x}$ and $q^{\alpha}=a^{-1}(q-1)^{-1}$ (or
$\alpha=-(\ln q)^{-1}\ln (q-1)a$) in the definition (\ref{DefqLaguerre}) of the
$q$-Laguerre polynomials, multiply by $(q;q)_n$, and take the limit
$q\rightarrow 1$ we obtain the Charlier polynomials given by (\ref{DefCharlier}):
\begin{equation}
\lim_{q\rightarrow 1}\,(q;q)_nL_n^{(\alpha)}(-q^{-x};q)=C_n(x;a),
\end{equation}
where
$$q^{\alpha}=\frac{1}{a(q-1)}\quad\textrm{or}\quad\alpha=-\frac{\ln (q-1)a}{\ln q}.$$

\subsection*{Remarks} 
The $q$-Laguerre polynomials are sometimes called the
generalized Stieltjes-Wigert polynomials.

If we replace $q$ by $q^{-1}$ we obtain the little $q$-Laguerre
(or Wall) polynomials given by (\ref{DefLittleqLaguerre}) in the following
way:
$$L_n^{(\alpha)}(x;q^{-1})=\frac{(q^{\alpha+1};q)_n}{(q;q)_nq^{n\alpha}}p_n(-x;q^{\alpha}|q).$$

\noindent
The $q$-Laguerre polynomials given by (\ref{DefqLaguerre}) and the $q$-Bessel polynomials
given by (\ref{DefqBessel}) are related in the following way:
$$\frac{y_n(q^x;a;q)}{(q;q)_n}=L_n^{(x-n)}(aq^n;q).$$

\noindent
The $q$-Laguerre polynomials given by (\ref{DefqLaguerre}) and the
$q$-Charlier polynomials given by (\ref{DefqCharlier}) are related in the
following way:
$$\frac{C_n(-x;-q^{-\alpha};q)}{(q;q)_n}=L_n^{(\alpha)}(x;q).$$

\noindent
Since the Stieltjes and Hamburger moment problems corresponding to the
$q$-Laguerre polynomials are indeterminate there exist many different weight
functions.

\subsection*{References}
\cite{NAlSalam89}, \cite{AlSalam90}, \cite{Askey86}, \cite{Askey89I}, \cite{AskeyWilson85},
\cite{AtakAtakI}, \cite{ChenIsmailMuttalib}, \cite{Chihara68II}, \cite{Chihara78},
\cite{Chihara79}, \cite{Chris}, \cite{Exton77}, \cite{GasperRahman90}, \cite{GrunbaumHaine96},
\cite{Ismail2005I}, \cite{IsmailRahman98}, \cite{Jain95}, \cite{Moak}.


\section{$q$-Bessel}
\index{q-Bessel polynomials@$q$-Bessel polynomials}
\par\setcounter{equation}{0}

\subsection*{Basic hypergeometric representation}
\begin{eqnarray}
\label{DefqBessel}
y_n(x;a;q)&=&\qhyp{2}{1}{q^{-n},-aq^n}{0}{qx}\\
&=&(q^{-n+1}x;q)_n\cdot\qhyp{1}{1}{q^{-n}}{q^{-n+1}x}{-aq^{n+1}x}\nonumber\\
&=&\left(-aq^nx\right)^n\cdot\qhyp{2}{1}{q^{-n},x^{-1}}{0}{-\frac{q^{-n+1}}{a}}.\nonumber
\end{eqnarray}

\newpage

\subsection*{Orthogonality relation}
\begin{eqnarray}
\label{OrtqBessel}
& &\sum_{k=0}^{\infty}\frac{a^k}{(q;q)_k}q^{\binom{k+1}{2}}y_m(q^k;a;q)y_n(q^k;a;q)\nonumber\\
& &{}=(q;q)_n(-aq^n;q)_{\infty}\frac{a^nq^{\binom{n+1}{2}}}{(1+aq^{2n})}\,\delta_{mn},\quad a>0.
\end{eqnarray}

\subsection*{Recurrence relation}
\begin{equation}
\label{RecqBessel}
-xy_n(x;a;q)=A_ny_{n+1}(x;a;q)-(A_n+C_n)y_n(x;a;q)+C_ny_{n-1}(x;a;q),
\end{equation}
where
$$\left\{\begin{array}{l}\displaystyle A_n=q^n\frac{(1+aq^n)}{(1+aq^{2n})(1+aq^{2n+1})}\\
\\
\displaystyle C_n=aq^{2n-1}\frac{(1-q^n)}{(1+aq^{2n-1})(1+aq^{2n})}.\end{array}\right.$$

\subsection*{Normalized recurrence relation}
\begin{equation}
\label{NormRecqBessel}
xp_n(x)=p_{n+1}(x)+(A_n+C_n)p_n(x)+A_{n-1}C_np_{n-1}(x),
\end{equation}
where
$$y_n(x;a;q)=(-1)^nq^{-\binom{n}{2}}(-aq^n;q)_np_n(x).$$

\subsection*{$q$-Difference equation}
\begin{eqnarray}
\label{dvqBessel}
& &-q^{-n}(1-q^n)(1+aq^n)xy(x)\nonumber\\
& &{}=axy(qx)-(ax+1-x)y(x)+(1-x)y(q^{-1}x),
\end{eqnarray}
where
$$y(x)=y_n(x;a;q).$$

\newpage

\subsection*{Forward shift operator}
\begin{equation}
\label{shift1qBesselI}
y_n(x;a;q)-y_n(qx;a;q)=-q^{-n+1}(1-q^n)(1+aq^n)xy_{n-1}(x;aq^2;q)
\end{equation}
or equivalently
\begin{equation}
\label{shift1qBesselII}
\mathcal{D}_qy_n(x;a;q)=-\frac{q^{-n+1}(1-q^n)(1+aq^n)}{1-q}y_{n-1}(x;aq^2;q).
\end{equation}

\subsection*{Backward shift operator}
\begin{equation}
\label{shift2qBesselI}
aq^{x-1}y_n(q^x;a;q)-(1-q^x)y_n(q^{x-1}x;a;q)=-y_{n+1}(q^x;aq^{-2};q)
\end{equation}
or equivalently
\begin{equation}
\label{shift2qBesselII}
\frac{\nabla\left[w(x;a;q)y_n(q^x;a;q)\right]}{\nabla q^x}=
\frac{q^2}{a(1-q)}w(x;aq^{-2};q)y_{n+1}(q^x;aq^{-2};q),
\end{equation}
where
$$w(x;a;q)=\frac{a^xq^{\binom{x}{2}}}{(q;q)_x}.$$

\subsection*{Rodrigues-type formula}
\begin{equation}
\label{RodqBessel}
w(x;a;q)y_n(q^x;a;q)=a^n(1-q)^nq^{n(n-1)}\left(\nabla_q\right)^n\left[w(x;aq^{2n};q)\right],
\end{equation}
where
$$\nabla_q:=\frac{\nabla}{\nabla q^x}.$$

\subsection*{Generating functions}
\begin{eqnarray}
\label{GenqBessel1}
& &\qhyp{0}{1}{-}{0}{-aq^{x+1}t}\,\qhyp{2}{0}{q^{-x},0}{-}{q^xt}\nonumber\\
& &{}=\sum_{n=0}^{\infty}\frac{y_n(q^x;a;q)}{(q;q)_n}t^n,\quad x=0,1,2,\ldots.
\end{eqnarray}

\begin{equation}
\label{GenqBessel2}
\frac{(t;q)_{\infty}}{(xt;q)_{\infty}}\,\qhyp{1}{3}{xt}{0,0,t}{-aqxt}=
\sum_{n=0}^{\infty}\frac{(-1)^nq^{\binom{n}{2}}}{(q;q)_n}y_n(x;a;q)t^n.
\end{equation}

\subsection*{Limit relations}

\subsubsection*{Little $q$-Jacobi $\rightarrow$ $q$-Bessel}
If we set $b\rightarrow -a^{-1}q^{-1}b$ in the definition (\ref{DefLittleqJacobi}) of
the little $q$-Jacobi polynomials and then take the limit $a\rightarrow 0$ we obtain
the $q$-Bessel polynomials given by (\ref{DefqBessel}):
$$\lim_{a\rightarrow 0}p_n(x;a,-a^{-1}q^{-1}b|q)=y_n(x;b;q).$$

\subsubsection*{$q$-Krawtchouk $\rightarrow$ $q$-Bessel}
If we set $x\rightarrow N-x$ in the definition (\ref{DefqKrawtchouk}) of the
$q$-Krawtchouk polynomials and then take the limit $N\rightarrow\infty$ we
obtain the $q$-Bessel polynomials given by (\ref{DefqBessel}):
$$\lim_{N\rightarrow\infty}K_n(q^{x-N};p,N;q)=y_n(q^x;p;q).$$

\subsubsection*{$q$-Bessel $\rightarrow$ Stieltjes-Wigert}
The Stieltjes-Wigert polynomials given by (\ref{DefStieltjesWigert}) can be obtained
from the $q$-Bessel polynomials by setting $x\rightarrow a^{-1}x$ in the definition
(\ref{DefqBessel}) of the $q$-Bessel polynomials and then taking the limit
$a\rightarrow\infty$. In fact we have
\begin{equation}
\lim_{a\rightarrow\infty}y_n(a^{-1}x;a;q)=(q;q)_nS_n(x;q).
\end{equation}

\subsubsection*{$q$-Bessel $\rightarrow$ Bessel}
If we set $x\rightarrow -\frac{1}{2}(1-q)^{-1}x$ and $a\rightarrow -q^{a+1}$ in the definition
(\ref{DefqBessel}) of the $q$-Bessel polynomials and take the limit $q\rightarrow 1$
we find the Bessel polynomials given by (\ref{DefBessel}):
\begin{equation}
\lim_{q\rightarrow 1}y_n(-\textstyle\frac{1}{2}(1-q)^{-1}x;-q^{a+1};q)=y_n(x;a).
\end{equation}

\subsubsection*{$q$-Bessel $\rightarrow$ Charlier}
If we set $x\rightarrow q^x$ and $a\rightarrow a(1-q)$ in the definition (\ref{DefqBessel})
of the $q$-Bessel polynomials and take the limit $q\rightarrow 1$ we find the Charlier
polynomials given by (\ref{DefCharlier}):
\begin{equation}
\lim_{q\rightarrow 1}\frac{y_n(q^x;a(1-q);q)}{(q-1)^n}=a^nC_n(x;a).
\end{equation}

\subsection*{Remark}
In \cite{Koekoek94} and \cite{Koekoek98} these $q$-Bessel polynomials were called
\emph{alternative $q$-Charlier polynomials}.

\noindent
The $q$-Bessel polynomials given by (\ref{DefqBessel}) and the $q$-Laguerre
polynomials given by (\ref{DefqLaguerre}) are related in the following way:
$$\frac{y_n(q^x;a;q)}{(q;q)_n}=L_n^{(x-n)}(aq^n;q).$$

\subsection*{Reference}
\cite{DattaGriffin}.


\section{$q$-Charlier}\index{q-Charlier polynomials@$q$-Charlier polynomials}
\par\setcounter{equation}{0}

\subsection*{Basic hypergeometric representation}
\begin{eqnarray}
\label{DefqCharlier}
C_n(q^{-x};a;q)&=&\qhyp{2}{1}{q^{-n},q^{-x}}{0}{-\frac{q^{n+1}}{a}}\\
&=&(-a^{-1}q;q)_n\cdot\qhyp{1}{1}{q^{-n}}{-a^{-1}q}{-\frac{q^{n+1-x}}{a}}.\nonumber
\end{eqnarray}

\subsection*{Orthogonality relation}
\begin{eqnarray}
\label{OrtqCharlier}
& &\sum_{x=0}^{\infty}\frac{a^x}{(q;q)_x}q^{\binom{x}{2}}C_m(q^{-x};a;q)C_n(q^{-x};a;q)\nonumber\\
& &{}=q^{-n}(-a;q)_{\infty}(-a^{-1}q,q;q)_n\,\delta_{mn},\quad a>0.
\end{eqnarray}

\newpage

\subsection*{Recurrence relation}
\begin{eqnarray}
\label{RecqCharlier}
& &q^{2n+1}(1-q^{-x})C_n(q^{-x})\nonumber\\
& &{}=aC_{n+1}(q^{-x})-\left[a+q(1-q^n)(a+q^n)\right]C_n(q^{-x})\nonumber\\
& &{}\mathindent{}+q(1-q^n)(a+q^n)C_{n-1}(q^{-x}),
\end{eqnarray}
where
$$C_n(q^{-x}):=C_n(q^{-x};a;q).$$

\subsection*{Normalized recurrence relation}
\begin{eqnarray}
\label{NormRecqCharlier}
xp_n(x)&=&p_{n+1}(x)+\left[1+q^{-2n-1}\left\{a+q(1-q^n)(a+q^n)\right\}\right]p_n(x)\nonumber\\
& &{}\mathindent{}+aq^{-4n+1}(1-q^n)(a+q^n)p_{n-1}(x),
\end{eqnarray}
where
$$C_n(q^{-x};a;q)=\frac{(-1)^nq^{n^2}}{a^n}p_n(q^{-x}).$$

\subsection*{$q$-Difference equation}
\begin{equation}
\label{dvqCharlier}
q^ny(x)=aq^xy(x+1)-q^x(a-1)y(x)+(1-q^x)y(x-1),
\end{equation}
where
$$y(x)=C_n(q^{-x};a;q).$$

\subsection*{Forward shift operator}
\begin{equation}
\label{shift1qCharlierI}
C_n(q^{-x-1};a;q)-C_n(q^{-x};a;q)=-a^{-1}q^{-x}(1-q^n)C_{n-1}(q^{-x};aq^{-1};q)
\end{equation}
or equivalently
\begin{equation}
\label{shift1qCharlierII}
\frac{\Delta C_n(q^{-x};a;q)}{\Delta q^{-x}}=-\frac{q(1-q^n)}{a(1-q)}C_{n-1}(q^{-x};aq^{-1};q).
\end{equation}

\newpage

\subsection*{Backward shift operator}
\begin{equation}
\label{shift2qCharlierI}
C_n(q^{-x};a;q)-a^{-1}q^{-x}(1-q^x)C_n(q^{-x+1};a;q)=C_{n+1}(q^{-x};aq;q)
\end{equation}
or equivalently
\begin{equation}
\label{shift2qCharlierII}
\frac{\nabla\left[w(x;a;q)C_n(q^{-x};a;q)\right]}{\nabla q^{-x}}
=\frac{1}{1-q}w(x;aq;q)C_{n+1}(q^{-x};aq;q),
\end{equation}
where
$$w(x;a;q)=\frac{a^xq^{\binom{x+1}{2}}}{(q;q)_x}.$$

\subsection*{Rodrigues-type formula}
\begin{equation}
\label{RodqCharlier}
w(x;a;q)C_n(q^{-x};a;q)=(1-q)^n\left(\nabla_q\right)^n\left[w(x;aq^{-n};q)\right],
\end{equation}
where
$$\nabla_q:=\frac{\nabla}{\nabla q^{-x}}.$$

\subsection*{Generating functions}
\begin{equation}
\label{GenqCharlier1}
\frac{1}{(t;q)_{\infty}}\,\qhyp{1}{1}{q^{-x}}{0}{-a^{-1}qt}
=\sum_{n=0}^{\infty}\frac{C_n(q^{-x};a;q)}{(q;q)_n}t^n.
\end{equation}

\begin{equation}
\label{GenqCharlier2}
\frac{1}{(t;q)_{\infty}}\,\qhyp{0}{1}{-}{-a^{-1}q}{-a^{-1}q^{-x+1}t}
=\sum_{n=0}^{\infty}\frac{C_n(q^{-x};a;q)}{(-a^{-1}q,q;q)_n}t^n.
\end{equation}

\subsection*{Limit relations}

\subsubsection*{$q$-Meixner $\rightarrow$ $q$-Charlier}
The $q$-Charlier polynomials given by (\ref{DefqCharlier}) can easily be
obtained from the $q$-Meixner given by (\ref{DefqMeixner}) by setting $b=0$
in the definition (\ref{DefqMeixner}) of the $q$-Meixner polynomials:
\begin{equation}
M_n(x;0,c;q)=C_n(x;c;q).
\end{equation}

\subsubsection*{$q$-Krawtchouk $\rightarrow$ $q$-Charlier}
By setting $p=a^{-1}q^{-N}$ in the definition (\ref{DefqKrawtchouk}) of the
$q$-Krawtchouk polynomials and then taking the limit $N\rightarrow\infty$ we
obtain the $q$-Charlier polynomials given by (\ref{DefqCharlier}):
$$\lim_{N\rightarrow\infty}K_n(q^{-x};a^{-1}q^{-N},N;q)=C_n(q^{-x};a;q).$$

\subsubsection*{$q$-Charlier $\rightarrow$ Stieltjes-Wigert}
If we set $q^{-x}\rightarrow ax$ in the definition (\ref{DefqCharlier}) of the
$q$-Charlier polynomials and take the limit $a\rightarrow\infty$ we obtain
the Stieltjes-Wigert polynomials given by (\ref{DefStieltjesWigert}) in the
following way:
\begin{equation}
\lim_{a\rightarrow\infty}C_n(ax;a;q)=(q;q)_nS_n(x;q).
\end{equation}

\subsubsection*{$q$-Charlier $\rightarrow$ Charlier}
If we set $a\rightarrow a(1-q)$ in the definition (\ref{DefqCharlier})
of the $q$-Charlier polynomials and take the limit $q\rightarrow 1$ we obtain the
Charlier polynomials given by (\ref{DefCharlier}):
\begin{equation}
\lim_{q\rightarrow 1}C_n(q^{-x};a(1-q);q)=C_n(x;a).
\end{equation}

\subsection*{Remark}
The $q$-Charlier polynomials given by (\ref{DefqCharlier}) and the
$q$-Laguerre polynomials given by (\ref{DefqLaguerre}) are related in the
following way:
$$\frac{C_n(-x;-q^{-\alpha};q)}{(q;q)_n}=L_n^{(\alpha)}(x;q).$$

\subsection*{References}
\cite{AlvarezRonveaux}, \cite{AtakRahmanSuslov}, \cite{GasperRahman90},
\cite{Hahn}, \cite{Koelink96III}, \cite{Nikiforov+}, \cite{Zeng95}.


\section{Al-Salam-Carlitz~I}\index{Al-Salam-Carlitz~I polynomials}
\par\setcounter{equation}{0}

\subsection*{Basic hypergeometric representation}
\begin{equation}
\label{DefAlSalamCarlitzI}
U_n^{(a)}(x;q)=(-a)^nq^{\binom{n}{2}}\,\qhyp{2}{1}{q^{-n},x^{-1}}{0}{\frac{qx}{a}}.
\end{equation}

\subsection*{Orthogonality relation}
\begin{eqnarray}
\label{OrtAlSalamCarlitzI}
& &\int_a^1(qx,a^{-1}qx;q)_{\infty}U_m^{(a)}(x;q)U_n^{(a)}(x;q)\,d_qx\nonumber\\
& &{}=(-a)^n(1-q)(q;q)_n(q,a,a^{-1}q;q)_{\infty}q^{\binom{n}{2}}\,\delta_{mn},\quad a<0.
\end{eqnarray}

\subsection*{Recurrence relation}
\begin{equation}
\label{RecAlSalamCarlitzI}
xU_n^{(a)}(x;q)=U_{n+1}^{(a)}(x;q)+(a+1)q^nU_n^{(a)}(x;q)
-aq^{n-1}(1-q^n)U_{n-1}^{(a)}(x;q).
\end{equation}

\subsection*{Normalized recurrence relation}
\begin{equation}
\label{NormRecAlSalamCarlitzI}
xp_n(x)=p_{n+1}(x)+(a+1)q^np_n(x)-aq^{n-1}(1-q^n)p_{n-1}(x),
\end{equation}
where
$$U_n^{(a)}(x;q)=p_n(x).$$

\subsection*{$q$-Difference equation}
\begin{eqnarray}
\label{dvAlSalamCarlitzI}
(1-q^n)x^2y(x)&=&aq^{n-1}y(qx)-\left[aq^{n-1}+q^n(1-x)(a-x)\right]y(x)\nonumber\\
& &{}\mathindent{}+q^n(1-x)(a-x)y(q^{-1}x),
\end{eqnarray}
where
$$y(x)=U_n^{(a)}(x;q).$$

\subsection*{Forward shift operator}
\begin{equation}
\label{shift1AlSalamCarlitzI-I}
U_n^{(a)}(x;q)-U_n^{(a)}(qx;q)=(1-q^n)xU_{n-1}^{(a)}(x;q)
\end{equation}
or equivalently
\begin{equation}
\label{shift1AlSalamCarlitzI-II}
\mathcal{D}_qU_n^{(a)}(x;q)=\frac{1-q^n}{1-q}U_{n-1}^{(a)}(x;q).
\end{equation}

\subsection*{Backward shift operator}
\begin{equation}
\label{shift2AlSalamCarlitzI-I}
aU_n^{(a)}(x;q)-(1-x)(a-x)U_n^{(a)}(q^{-1}x;q)=-q^{-n}xU_{n+1}^{(a)}(x;q)
\end{equation}
or equivalently
\begin{equation}
\label{shift2AlSalamCarlitzI-II}
\mathcal{D}_{q^{-1}}\left[w(x;a;q)U_n^{(a)}(x;q)\right]=
\frac{q^{-n+1}}{a(1-q)}w(x;a;q)U_{n+1}^{(a)}(x;q),
\end{equation}
where
$$w(x;a;q)=(qx,a^{-1}qx;q)_{\infty}.$$

\subsection*{Rodrigues-type formula}
\begin{equation}
\label{RodAlSalamCarlitzI}
w(x;a;q)U_n^{(a)}(x;q)=a^nq^{\frac{1}{2}n(n-3)}(1-q)^n
\left(\mathcal{D}_{q^{-1}}\right)^n\left[w(x;a;q)\right].
\end{equation}

\subsection*{Generating function}
\begin{equation}
\label{GenAlSalamCarlitzI}
\frac{(t,at;q)_{\infty}}{(xt;q)_{\infty}}=
\sum_{n=0}^{\infty}\frac{U_n^{(a)}(x;q)}{(q;q)_n}t^n.
\end{equation}

\subsection*{Limit relations}

\subsubsection*{Big $q$-Laguerre $\rightarrow$ Al-Salam-Carlitz~I}
If we set $x\rightarrow aqx$ and $b\rightarrow ab$ in the definition
(\ref{DefBigqLaguerre}) of the big $q$-Laguerre polynomials and take the
limit $a\rightarrow 0$ we obtain the Al-Salam-Carlitz~I polynomials given by
(\ref{DefAlSalamCarlitzI}):
$$\lim_{a\rightarrow 0}\frac{P_n(aqx;a,ab;q)}{a^n}=q^nU_n^{(b)}(x;q).$$

\subsubsection*{Dual $q$-Krawtchouk $\rightarrow$ Al-Salam-Carlitz~I}
If we set $c=a^{-1}$ in the definition (\ref{DefDualqKrawtchouk})
of the dual $q$-Krawtchouk polynomials and take the limit
$N\rightarrow\infty$ we simply obtain the Al-Salam-Carlitz~I polynomials
given by (\ref{DefAlSalamCarlitzI}):
$$\lim_{N\rightarrow\infty}K_n(\lambda(x);a^{-1},N|q)=
\left(-\frac{1}{a}\right)^nq^{-\binom{n}{2}}U_n^{(a)}(q^x;q).$$
Note that $\lambda(x)=q^{-x}+a^{-1}q^{x-N}$.

\subsubsection*{Al-Salam-Carlitz~I $\rightarrow$ Discrete $q$-Hermite~I}
The discrete $q$-Hermite~I polynomials given by
(\ref{DefDiscreteqHermiteI}) can easily be obtained from the
Al-Salam-Carlitz~I polynomials given by (\ref{DefAlSalamCarlitzI}) by the
substitution $a=-1$:
\begin{equation}
U_n^{(-1)}(x;q)=h_n(x;q).
\end{equation}

\subsubsection*{Al-Salam-Carlitz~I $\rightarrow$ Charlier / Hermite}
If we set $a\rightarrow a(q-1)$ and $x\rightarrow q^x$ in the definition
(\ref{DefAlSalamCarlitzI}) of the Al-Salam-Carlitz~I polynomials and take the
limit $q\rightarrow 1$ after dividing by $a^n(1-q)^n$ we obtain the Charlier
polynomials given by (\ref{DefCharlier}):
\begin{equation}
\lim_{q\rightarrow 1}\frac{U_n^{(a(q-1))}(q^x;q)}{(1-q)^n}=a^nC_n(x;a).
\end{equation}

If we set $x\rightarrow x\sqrt{1-q^2}$ and $a\rightarrow a\sqrt{1-q^2}-1$
in the definition (\ref{DefAlSalamCarlitzI}) of the Al-Salam-Carlitz~I
polynomials, divide by $(1-q^2)^{\frac{n}{2}}$, and let $q$ tend to $1$ we
obtain the Hermite polynomials given by (\ref{DefHermite}) with shifted
argument. In fact we have
\begin{equation}
\lim_{q\rightarrow 1}\frac{U_n^{(a\sqrt{1-q^2}-1)}(x\sqrt{1-q^2};q)}
{(1-q^2)^{\frac{n}{2}}}=\frac{H_n(x-a)}{2^n}.
\end{equation}

\subsection*{Remark}
The Al-Salam-Carlitz~I polynomials are related to the Al-Salam-Carlitz~II
polynomials given by (\ref{DefAlSalamCarlitzII}) in the following way:
$$U_n^{(a)}(x;q^{-1})=V_n^{(a)}(x;q).$$

\subsection*{References}
\cite{AlSalam90}, \cite{AlSalamCarlitz}, \cite{AlSalamChihara76}, \cite{AskeySuslovII},
\cite{AtakRahmanSuslov}, \cite{Chihara68II}, \cite{Chihara78}, \cite{DattaGriffin},
\cite{Dehesa}, \cite{DohaAhmed2005}, \cite{GasperRahman90}, \cite{Ismail85I},
\cite{IsmailMuldoon}, \cite{Kim}, \cite{Zeng95}.


\section{Al-Salam-Carlitz~II}\index{Al-Salam-Carlitz~II polynomials}
\par\setcounter{equation}{0}

\subsection*{Basic hypergeometric representation}
\begin{equation}
\label{DefAlSalamCarlitzII}
V_n^{(a)}(x;q)=
(-a)^nq^{-\binom{n}{2}}\,\qhyp{2}{0}{q^{-n},x}{-}{\frac{q^n}{a}}.
\end{equation}

\subsection*{Orthogonality relation}
\begin{eqnarray}
\label{OrtAlSalamCarlitzII}
& &\sum_{k=0}^{\infty}\frac{q^{k^2}a^k}{(q;q)_k(aq;q)_k}
V_m^{(a)}(q^{-k};q)V_n^{(a)}(q^{-k};q)\nonumber\\
& &{}=\frac{(q;q)_na^n}{(aq;q)_{\infty}q^{n^2}}\,\delta_{mn},\quad 0<aq<1.
\end{eqnarray}

\subsection*{Recurrence relation}
\begin{eqnarray}
\label{RecAlSalamCarlitzII}
xV_n^{(a)}(x;q)&=&V_{n+1}^{(a)}(x;q)+(a+1)q^{-n}V_n^{(a)}(x;q)\nonumber\\
& &{}\mathindent{}+aq^{-2n+1}(1-q^n)V_{n-1}^{(a)}(x;q).
\end{eqnarray}

\subsection*{Normalized recurrence relation}
\begin{equation}
\label{NormRecAlSalamCarlitzII}
xp_n(x)=p_{n+1}(x)+(a+1)q^{-n}p_n(x)+aq^{-2n+1}(1-q^n)p_{n-1}(x),
\end{equation}
where
$$V_n^{(a)}(x;q)=p_n(x).$$

\subsection*{$q$-Difference equation}
\begin{eqnarray}
\label{dvAlSalamCarlitzII}
& &-(1-q^n)x^2y(x)\nonumber\\
& &{}=(1-x)(a-x)y(qx)-\left[(1-x)(a-x)+aq\right]y(x)+aqy(q^{-1}x),
\end{eqnarray}
where
$$y(x)=V_n^{(a)}(x;q).$$

\subsection*{Forward shift operator}
\begin{equation}
\label{shift1AlSalamCarlitzII-I}
V_n^{(a)}(x;q)-V_n^{(a)}(qx;q)=q^{-n+1}(1-q^n)xV_{n-1}^{(a)}(qx;q)
\end{equation}
or equivalently
\begin{equation}
\label{shift1AlSalamCarlitzII-II}
\mathcal{D}_qV_n^{(a)}(x;q)=\frac{q^{-n+1}(1-q^n)}{1-q}
V_{n-1}^{(a)}(qx;q).
\end{equation}

\subsection*{Backward shift operator}
\begin{equation}
\label{shift2AlSalamCarlitzII-I}
aV_n^{(a)}(x;q)-(1-x)(a-x)V_n^{(a)}(qx;q)=-q^nxV_{n+1}^{(a)}(x;q)
\end{equation}
or equivalently
\begin{equation}
\label{shift2AlSalamCarlitzII-II}
\mathcal{D}_q\left[w(x;a;q)V_n^{(a)}(x;q)\right]
=-\frac{q^n}{a(1-q)}w(x;a;q)V_{n+1}^{(a)}(x;q),
\end{equation}
where
$$w(x;a;q)=\frac{1}{(x,a^{-1}x;q)_{\infty}}.$$

\newpage

\subsection*{Rodrigues-type formula}
\begin{equation}
\label{RodAlSalamCarlitzII}
w(x;a;q)V_n^{(a)}(x;q)=a^n(q-1)^nq^{-\binom{n}{2}}
\left(\mathcal{D}_q\right)^n\left[w(x;a;q)\right].
\end{equation}

\subsection*{Generating functions}
\begin{equation}
\label{GenAlSalmCarlitzII1}
\frac{(xt;q)_{\infty}}{(t,at;q)_{\infty}}=\sum_{n=0}^{\infty}
\frac{(-1)^nq^{\binom{n}{2}}}{(q;q)_n}V_n^{(a)}(x;q)t^n.
\end{equation}

\begin{equation}
\label{GenAlSalamCarlitzII2}
(at;q)_{\infty}\cdot\qhyp{1}{1}{x}{at}{t}=\sum_{n=0}^{\infty}
\frac{q^{n(n-1)}}{(q;q)_n}V_n^{(a)}(x;q)t^n.
\end{equation}

\subsection*{Limit relations}

\subsubsection*{$q$-Meixner $\rightarrow$ Al-Salam-Carlitz~II}
The Al-Salam-Carlitz~II polynomials given by (\ref{DefAlSalamCarlitzII})
can be obtained from the $q$-Meixner polynomials given by (\ref{DefqMeixner})
by setting $b=-ac^{-1}$ in the definition (\ref{DefqMeixner}) of the
$q$-Meixner polynomials and then taking the limit $c\rightarrow 0$:
$$\lim_{c\rightarrow 0}M_n(x;-ac^{-1},c;q)=
\left(-\frac{1}{a}\right)^nq^{\binom{n}{2}}V_n^{(a)}(x;q).$$

\subsubsection*{Quantum $q$-Krawtchouk $\rightarrow$ Al-Salam-Carlitz~II}
If we set $p=a^{-1}q^{-N-1}$ in the definition (\ref{DefQuantumqKrawtchouk})
of the quantum $q$-Krawtchouk polynomials and let $N\rightarrow\infty$ we
obtain the Al-Salam-Carlitz~II polynomials given by
(\ref{DefAlSalamCarlitzII}). In fact we have
$$\lim_{N\rightarrow\infty}K_n^{qtm}(x;a^{-1}q^{-N-1},N;q)=
\left(-\frac{1}{a}\right)^nq^{\binom{n}{2}}V_n^{(a)}(x;q).$$

\subsubsection*{Al-Salam-Carlitz~II $\rightarrow$ Discrete $q$-Hermite~II}
The discrete $q$-Hermite~II polynomials given by
(\ref{DefDiscreteqHermiteII}) follow from the Al-Salam-Carlitz~II
polynomials given by (\ref{DefAlSalamCarlitzII}) by the substitution $a=-1$
in the following way:
\begin{equation}
i^{-n}V_n^{(-1)}(ix;q)={\tilde h}_n(x;q).
\end{equation}

\subsubsection*{Al-Salam-Carlitz~II $\rightarrow$ Charlier / Hermite}
If we set $a\rightarrow a(1-q)$ and $x\rightarrow q^{-x}$ in the
definition (\ref{DefAlSalamCarlitzII}) of the Al-Salam-Carlitz~II polynomials
and taking the limit $q\rightarrow 1$ we find
\begin{equation}
\lim_{q\rightarrow 1}\frac{V_n^{(a(1-q))}(q^{-x};q)}{(q-1)^n}=
a^nC_n(x;a).
\end{equation}

If we set $x\rightarrow ix\sqrt{1-q^2}$ and $a\rightarrow ia\sqrt{1-q^2}-1$
in the definition (\ref{DefAlSalamCarlitzII}) of the Al-Salam-Carlitz~II
polynomials, divide by $i^n(1-q^2)^{\frac{n}{2}}$, and let $q$ tend to $1$ we
obtain the Hermite polynomials given by (\ref{DefHermite}) with shifted
argument. In fact we have
\begin{equation}
\lim_{q\rightarrow 1}\frac{V_n^{(ia\sqrt{1-q^2}-1)}(ix\sqrt{1-q^2};q)}
{i^n(1-q^2)^{\frac{n}{2}}}=\frac{H_n(x-a)}{2^n}.
\end{equation}

\subsection*{Remark}
The Al-Salam-Carlitz~II polynomials are related to the Al-Salam-Carlitz~I
polynomials given by (\ref{DefAlSalamCarlitzI}) in the following way:
$$V_n^{(a)}(x;q^{-1})=U_n^{(a)}(x;q).$$

\subsection*{References}
\cite{AlSalam90}, \cite{AlSalamCarlitz}, \cite{AlSalamChihara76},
\cite{AskeySuslovI}, \cite{BergValent}, \cite{Chihara68I},
\cite{Chihara68II}, \cite{Chihara78}, \cite{Dehesa}, \cite{FlorisKoelink},
\cite{Ismail85I}, \cite{Ismail2005I}.


\section{Continuous $q$-Hermite}
\index{Continuous q-Hermite polynomials@Continuous $q$-Hermite polynomials}
\index{q-Hermite polynomials@$q$-Hermite polynomials!Continuous}
\par\setcounter{equation}{0}

\subsection*{Basic hypergeometric representation}
\begin{equation}
\label{DefContqHermite}
H_n(x|q)=\e^{in\theta}\,\qhyp{2}{0}{q^{-n},0}{-}{q^n\e^{-2i\theta}},
\quad x=\cos\theta.
\end{equation}

\newpage

\subsection*{Orthogonality relation}
\begin{equation}
\label{OrtContqHermite}
\frac{1}{2\pi}\int_{-1}^1\frac{w(x|q)}{\sqrt{1-x^2}}H_m(x|q)H_n(x|q)\,dx=
\frac{\,\delta_{mn}}{(q^{n+1};q)_{\infty}},
\end{equation}
where
$$w(x|q)=\left|\left(\e^{2i\theta};q\right)_{\infty}\right|^2=h(x,1)h(x,-1)
h(x,q^{\frac{1}{2}})h(x,-q^{\frac{1}{2}}),$$
with
$$h(x,\alpha):=\prod_{k=0}^{\infty}\left(1-2\alpha xq^k+\alpha^2q^{2k}\right)
=\left(\alpha\e^{i\theta},\alpha\e^{-i\theta};q\right)_{\infty},\quad x=\cos\theta.$$

\subsection*{Recurrence relation}
\begin{equation}
\label{RecContqHermite}
2xH_n(x|q)=H_{n+1}(x|q)+(1-q^n)H_{n-1}(x|q).
\end{equation}

\subsection*{Normalized recurrence relation}
\begin{equation}
\label{NormRecContqHermite}
xp_n(x)=p_{n+1}(x)+\frac{1}{4}(1-q^n)p_{n-1}(x),
\end{equation}
where
$$H_n(x|q)=2^np_n(x).$$

\subsection*{$q$-Difference equation}
\begin{equation}
\label{dvContqHermite}
(1-q)^2D_q\left[{\tilde w}(x|q)D_qy(x)\right]+4q^{-n+1}(1-q^n){\tilde w}(x|q)y(x)=0,
\end{equation}
where
$$y(x)=H_n(x|q)$$
and
$${\tilde w}(x|q):=\frac{w(x|q)}{\sqrt{1-x^2}}.$$

\subsection*{Forward shift operator}
\begin{equation}
\label{shift1ContqHermiteI}
\delta_qH_n(x|q)=-q^{-\frac{1}{2}n}(1-q^n)(\e^{i\theta}-\e^{-i\theta})H_{n-1}(x|q),
\quad x=\cos\theta
\end{equation}
or equivalently
\begin{equation}
\label{shift1ContqHermiteII}
D_qH_n(x|q)=\frac{2q^{-\frac{1}{2}(n-1)}(1-q^n)}{1-q}H_{n-1}(x|q).
\end{equation}

\subsection*{Backward shift operator}
\begin{equation}
\label{shift2ContqHermiteI}
\delta_q\left[{\tilde w}(x|q)H_n(x|q)\right]=
q^{-\frac{1}{2}(n+1)}(\e^{i\theta}-\e^{-i\theta})
{\tilde w}(x|q)H_{n+1}(x|q),\quad x=\cos\theta
\end{equation}
or equivalently
\begin{equation}
\label{shift2ContqHermiteII}
D_q\left[{\tilde w}(x|q)H_n(x|q)\right]=
-\frac{2q^{-\frac{1}{2}n}}{1-q}{\tilde w}(x|q)H_{n+1}(x|q).
\end{equation}

\subsection*{Rodrigues-type formula}
\begin{equation}
\label{RodContqHermite}
{\tilde w}(x|q)H_n(x|q)=\left(\frac{q-1}{2}\right)^nq^{\frac{1}{4}n(n-1)}
\left(D_q\right)^n\left[{\tilde w}(x|q)\right].
\end{equation}

\subsection*{Generating functions}
\begin{equation}
\label{GenContqHermite1}
\frac{1}{\left|(\e^{i\theta}t;q)_{\infty}\right|^2}=
\frac{1}{(\e^{i\theta}t,\e^{-i\theta}t;q)_{\infty}}=
\sum_{n=0}^{\infty}\frac{H_n(x|q)}{(q;q)_n}t^n,\quad x=\cos\theta.
\end{equation}

\begin{eqnarray}
\label{GenContqHermite2}
& &(\e^{i\theta}t;q)_{\infty}\cdot\qhyp{1}{1}{0}{\e^{i\theta}t}{\e^{-i\theta}t}\nonumber\\
& &{}=\sum_{n=0}^{\infty}\frac{(-1)^nq^{\binom{n}{2}}}{(q;q)_n}H_n(x|q)t^n,
\quad x=\cos\theta.
\end{eqnarray}

\begin{eqnarray}
\label{GenContqHermite3}
& &\frac{(\gamma\e^{i\theta}t;q)_{\infty}}{(\e^{i\theta}t;q)_{\infty}}\,
\qhyp{2}{1}{\gamma,0}{\gamma\e^{i\theta}t}{\e^{-i\theta}t}\nonumber\\
& &{}=\sum_{n=0}^{\infty}\frac{(\gamma;q)_n}{(q;q)_n}H_n(x|q)t^n,
\quad x=\cos\theta,\quad\textrm{$\gamma$ arbitrary}.
\end{eqnarray}

\subsection*{Limit relations}

\subsubsection*{Continuous big $q$-Hermite $\rightarrow$ Continuous $q$-Hermite}
The continuous $q$-Hermite polynomials given by (\ref{DefContqHermite})
can easily be obtained from the continuous big $q$-Hermite polynomials given by
(\ref{DefContBigqHermite}) by taking $a=0$:
$$H_n(x;0|q)=H_n(x|q).$$

\subsubsection*{Continuous $q$-Laguerre $\rightarrow$ Continuous $q$-Hermite}
The continuous $q$-Hermite polynomials given by (\ref{DefContqHermite}) can
be obtained from the continuous $q$-Laguerre polynomials given by
(\ref{DefContqLaguerre}) by taking the limit $\alpha\rightarrow\infty$ in the
following way:
$$\lim_{\alpha\rightarrow\infty}
\frac{P_n^{(\alpha)}(x|q)}{q^{(\frac{1}{2}\alpha+\frac{1}{4})n}}
=\frac{H_n(x|q)}{(q;q)_n}.$$

\subsubsection*{Continuous $q$-Hermite $\rightarrow$ Hermite}
The Hermite polynomials given by (\ref{DefHermite}) can be obtained
from the continuous $q$-Hermite polynomials given by (\ref{DefContqHermite}) by
setting $x\rightarrow x\sqrt{\frac{1}{2}(1-q)}$. In fact we have
\begin{equation}
\lim_{q\rightarrow 1}
\frac{H_n(x\sqrt{\frac{1}{2}(1-q)}|q)}
{\left(\frac{1-q}{2}\right)^{\frac{n}{2}}}=H_n(x).
\end{equation}

\subsection*{Remark} The continuous $q$-Hermite polynomials can also be written
as:
$$H_n(x|q)=\sum_{k=0}^n\frac{(q;q)_n}{(q;q)_k(q;q)_{n-k}}\e^{i(n-2k)\theta},
\quad x=\cos\theta.$$

\subsection*{References}
\cite{Allaway80}, \cite{AlSalam90}, \cite{AlSalam95}, \cite{AlSalamIsmail88},
\cite{AndrewsAskey85}, \cite{Askey89I}, \cite{Askey89II}, \cite{AskeyIsmail80},
\cite{AskeyIsmail83}, \cite{AskeyWilson85}, \cite{AskeyRahmanSuslov}, \cite{AtakKlimyk2007},
\cite{Atak96}, \cite{AtakFeinsilver}, \cite{AtakRahmanSuslov}, \cite{AtakAtakIII},
\cite{AtakAtakIV}, \cite{BergIsmail}, \cite{Bressoud80}, \cite{BustozIsmail82},
\cite{BustozIsmail97}, \cite{FloreaniniVinetI}, \cite{GasperRahman83II}, \cite{GasperRahman90},
\cite{Ismail86II}, \cite{Ismail2005I}, \cite{IsmailMasson94}, \cite{IsmailStanton88},
\cite{IsmailStantonViennot}, \cite{Nikiforov+}, \cite{Rogers93}, \cite{Rogers94},
\cite{Rogers95}, \cite{SrivastavaJain89}.


\section{Stieltjes-Wigert}\index{Stieltjes-Wigert polynomials}
\par\setcounter{equation}{0}

\subsection*{Basic hypergeometric representation}
\begin{equation}
\label{DefStieltjesWigert}
S_n(x;q)=\frac{1}{(q;q)_n}\,\qhyp{1}{1}{q^{-n}}{0}{-q^{n+1}x}.
\end{equation}

\subsection*{Orthogonality relation}
\begin{equation}
\label{OrtStieltjesWigert}
\int_{0}^{\infty}\frac{S_m(x;q)S_n(x;q)}{(-x,-qx^{-1};q)_{\infty}}\,dx
=-\frac{\ln q}{q^n}\frac{(q;q)_{\infty}}{(q;q)_n}\,\delta_{mn}.
\end{equation}

\subsection*{Recurrence relation}
\begin{eqnarray}
\label{RecStieltjesWigert}
& &-q^{2n+1}xS_n(x;q)\nonumber\\
& &{}=(1-q^{n+1})S_{n+1}(x;q)-[1+q-q^{n+1}]S_n(x;q)+qS_{n-1}(x;q).
\end{eqnarray}

\subsection*{Normalized recurrence relation}
\begin{eqnarray}
\label{NormRecStieltjesWigert}
xp_n(x)&=&p_{n+1}(x)+q^{-2n-1}\left[1+q-q^{n+1}\right]p_n(x)\nonumber\\
& &{}\mathindent{}+q^{-4n+1}(1-q^n)p_{n-1}(x),
\end{eqnarray}
where
$$S_n(x;q)=\frac{(-1)^nq^{n^2}}{(q;q)_n}p_n(x).$$

\subsection*{$q$-Difference equation}
\begin{equation}
\label{dvStieltjesWigert}
-x(1-q^n)y(x)=xy(qx)-(x+1)y(x)+y(q^{-1}x),\quad y(x)=S_n(x;q).
\end{equation}

\newpage

\subsection*{Forward shift operator}
\begin{equation}
\label{shift1StieltjesWigertI}
S_n(x;q)-S_n(qx;q)=-qxS_{n-1}(q^2x;q)
\end{equation}
or equivalently
\begin{equation}
\label{shift1StieltjesWigertII}
\mathcal{D}_qS_n(x;q)=-\frac{q}{1-q}S_{n-1}(q^2x;q).
\end{equation}

\subsection*{Backward shift operator}
\begin{equation}
\label{shift2StieltjesWigertI}
S_n(x;q)-xS_n(qx;q)=(1-q^{n+1})S_{n+1}(q^{-1}x;q),
\end{equation}
or equivalently
\begin{equation}
\label{shift2StieltjesWigertII}
\mathcal{D}_q\left[w(x;q)S_n(x;q)\right]=\frac{1-q^{n+1}}{1-q}q^{-1}w(q^{-1}x;q)S_{n+1}(q^{-1}x;q),
\end{equation}
where
$$w(x;q)=\frac{1}{(-x,-qx^{-1};q)_{\infty}}.$$

\subsection*{Rodrigues-type formula}
\begin{equation}
\label{RodStieltjesWigert}
w(x;q)S_n(x;q)=\frac{q^n(1-q)^n}{(q;q)_n}\left(\left(\mathcal{D}_q\right)^n w\right)(q^nx;q).
\end{equation}

\subsection*{Generating functions}
\begin{equation}
\label{GenStieltjesWigert1}
\frac{1}{(t;q)_{\infty}}\,\qhyp{0}{1}{-}{0}{-qxt}=
\sum_{n=0}^{\infty}S_n(x;q)t^n.
\end{equation}

\begin{equation}
\label{GenStieltjesWigert2}
(t;q)_{\infty}\cdot\qhyp{0}{2}{-}{0,t}{-qxt}=
\sum_{n=0}^{\infty}(-1)^nq^{\binom{n}{2}}S_n(x;q)t^n.
\end{equation}

\begin{equation}
\label{GenStieltjesWigert3}
\frac{(\gamma t;q)_{\infty}}{(t;q)_{\infty}}\,\qhyp{1}{2}{\gamma}{0,\gamma t}{-qxt}
=\sum_{n=0}^{\infty}(\gamma;q)_nS_n(x;q)t^n,\quad\textrm{$\gamma$ arbitrary}.
\end{equation}

\subsection*{Limit relations}

\subsubsection*{$q$-Laguerre $\rightarrow$ Stieltjes-Wigert}
If we set $x\rightarrow xq^{-\alpha}$ in the definition
(\ref{DefqLaguerre}) of the $q$-Laguerre polynomials and take the limit
$\alpha\rightarrow\infty$ we simply obtain the Stieltjes-Wigert polynomials given
by (\ref{DefStieltjesWigert}):
$$\lim_{\alpha\rightarrow\infty}L_n^{(\alpha)}\left(xq^{-\alpha};q\right)=S_n(x;q).$$

\subsubsection*{$q$-Bessel $\rightarrow$ Stieltjes-Wigert}
The Stieltjes-Wigert polynomials given by (\ref{DefStieltjesWigert}) can be obtained
from the $q$-Bessel polynomials by setting $x\rightarrow a^{-1}x$ in the definition
(\ref{DefqBessel}) of the $q$-Bessel polynomials and then taking the limit
$a\rightarrow\infty$. In fact we have
$$\lim_{a\rightarrow\infty}y_n(a^{-1}x;a;q)=(q;q)_nS_n(x;q).$$

\subsubsection*{$q$-Charlier $\rightarrow$ Stieltjes-Wigert}
If we set $q^{-x}\rightarrow ax$ in the definition (\ref{DefqCharlier}) of the
$q$-Charlier polynomials and take the limit $a\rightarrow\infty$ we obtain
the Stieltjes-Wigert polynomials given by (\ref{DefStieltjesWigert}) in the
following way:
$$\lim_{a\rightarrow\infty}C_n(ax;a;q)=(q;q)_nS_n(x;q).$$

\subsubsection*{Stieltjes-Wigert $\rightarrow$ Hermite}
The Hermite polynomials given by (\ref{DefHermite}) can be obtained from
the Stieltjes-Wigert polynomials given by (\ref{DefStieltjesWigert}) by
setting $x\rightarrow q^{-1}x\sqrt{2(1-q)}+1$ and taking the limit
$q\rightarrow 1$ in the following way:
\begin{equation}
\lim_{q\rightarrow 1}\frac{(q;q)_nS_n(q^{-1}x\sqrt{2(1-q)}+1;q)}
{\left(\frac{1-q}{2}\right)^{\frac{n}{2}}}=(-1)^nH_n(x).
\end{equation}

\subsection*{Remark} Since the Stieltjes and Hamburger moment problems corresponding
to the Stieltjes-Wigert polynomials are indeterminate there exist many
different weight functions. For instance, they are also orthogonal with
respect to the weight function
$$w(x)=\frac{\gamma}{\sqrt{\pi}}\exp\left(-\gamma^2\ln^2x\right),\quad x>0,\quad\textrm{with}\quad
\gamma^2=-\frac{1}{2\ln q}.$$

\subsection*{References}
\cite{Askey86}, \cite{Askey89I}, \cite{AtakAtakIII}, \cite{Chihara70}, \cite{Chihara78},
\cite{Dehesa}, \cite{Ismail2005I}, \cite{Nikiforov+}, \cite{Stieltjes}, \cite{Szego75},
\cite{ValentAssche}, \cite{Wigert}.


\section{Discrete $q$-Hermite~I}
\index{Discrete q-Hermite~I polynomials@Discrete $q$-Hermite~I polynomials}
\index{q-Hermite~I polynomials@$q$-Hermite~I polynomials!Discrete}
\par\setcounter{equation}{0}

\subsection*{Basic hypergeometric representation}
The discrete $q$-Hermite~I polynomials are Al-Salam-Carlitz~I polynomials
with $a=-1$:
\begin{eqnarray}
\label{DefDiscreteqHermiteI}
h_n(x;q)=U_n^{(-1)}(x;q)&=&q^{\binom{n}{2}}\,\qhyp{2}{1}{q^{-n},x^{-1}}{0}{-qx}\\
&=&x^n\qhypK{2}{0}{q^{-n},q^{-n+1}}{0}{q^2,\frac{q^{2n-1}}{x^2}}\nonumber
\end{eqnarray}

\subsection*{Orthogonality relation}
\begin{eqnarray}
\label{OrtDiscreteqHermiteI}
& &\int_{-1}^1(qx,-qx;q)_{\infty}h_m(x;q)h_n(x;q)\,d_qx\nonumber\\
& &{}=(1-q)(q;q)_n(q,-1,-q;q)_{\infty}q^{\binom{n}{2}}\,\delta_{mn}.
\end{eqnarray}

\subsection*{Recurrence relation}
\begin{equation}
\label{RecDiscreteqHermiteI}
xh_n(x;q)=h_{n+1}(x;q)+q^{n-1}(1-q^n)h_{n-1}(x;q).
\end{equation}

\subsection*{Normalized recurrence relation}
\begin{equation}
\label{NormRecDiscreteqHermiteI}
xp_n(x)=p_{n+1}(x)+q^{n-1}(1-q^n)p_{n-1}(x),
\end{equation}
where
$$h_n(x;q)=p_n(x).$$

\subsection*{$q$-Difference equation}
\begin{equation}
\label{dvDiscreteqHermiteI}
-q^{-n+1}x^2y(x)=y(qx)-(1+q)y(x)+q(1-x^2)y(q^{-1}x),
\end{equation}
where
$$y(x)=h_n(x;q).$$

\subsection*{Forward shift operator}
\begin{equation}
\label{shift1DiscreteqHermiteI-I}
h_n(x;q)-h_n(qx;q)=(1-q^n)xh_{n-1}(x;q)
\end{equation}
or equivalently
\begin{equation}
\label{shift1DiscreteqHermiteI-II}
\mathcal{D}_qh_n(x;q)=\frac{1-q^n}{1-q}h_{n-1}(x;q).
\end{equation}

\subsection*{Backward shift operator}
\begin{equation}
\label{shift2DiscreteqHermiteI-I}
h_n(x;q)-(1-x^2)h_n(q^{-1}x;q)=q^{-n}xh_{n+1}(x;q)
\end{equation}
or equivalently
\begin{equation}
\label{shift2DiscreteqHermiteI-II}
\mathcal{D}_{q^{-1}}\left[w(x;q)h_n(x;q)\right]
=-\frac{q^{-n+1}}{1-q}w(x;q)h_{n+1}(x;q),
\end{equation}
where
$$w(x;q)=(qx,-qx;q)_{\infty}.$$

\subsection*{Rodrigues-type formula}
\begin{equation}
\label{RodDiscreteqHermiteI}
w(x;q)h_n(x;q)=(q-1)^nq^{\frac{1}{2}n(n-3)}
\left(\mathcal{D}_{q^{-1}}\right)^n\left[w(x;q)\right].
\end{equation}

\subsection*{Generating function}
\begin{equation}
\label{GenDiscreteqHermiteI}
\frac{(t^2;q^2)_{\infty}}{(xt;q)_{\infty}}=
\sum_{n=0}^{\infty}\frac{h_n(x;q)}{(q;q)_n}t^n.
\end{equation}

\subsection*{Limit relations}

\subsubsection*{Al-Salam-Carlitz~I $\rightarrow$ Discrete $q$-Hermite~I}
The discrete $q$-Hermite~I polynomials given by
(\ref{DefDiscreteqHermiteI}) can easily be obtained from the
Al-Salam-Carlitz~I polynomials given by (\ref{DefAlSalamCarlitzI}) by the
substitution $a=-1$:
$$U_n^{(-1)}(x;q)=h_n(x;q).$$

\subsubsection*{Discrete $q$-Hermite~I $\rightarrow$ Hermite}
The Hermite polynomials given by (\ref{DefHermite}) can be found
from the discrete $q$-Hermite~I polynomials given by
(\ref{DefDiscreteqHermiteI}) in the following way:
\begin{equation}
\lim_{q\rightarrow 1}\frac{h_n(x\sqrt{1-q^2};q)}
{(1-q^2)^{\frac{n}{2}}}=\frac{H_n(x)}{2^n}.
\end{equation}

\subsection*{Remark} The discrete $q$-Hermite~I polynomials are related to the
discrete $q$-Hermite~II polynomials given by (\ref{DefDiscreteqHermiteII})
in the following way:
$$h_n(ix;q^{-1})=i^n{\tilde h}_n(x;q).$$

\subsection*{References}
\cite{AlSalam90}, \cite{AlSalamCarlitz}, \cite{AtakRahmanSuslov},
\cite{BergIsmail}, \cite{BustozIsmail82}, \cite{GasperRahman90}, \cite{Hahn},
\cite{Koorn97}.


\newpage

\section{Discrete $q$-Hermite~II}
\index{Discrete q-Hermite~II polynomials@Discrete $q$-Hermite~II polynomials}
\index{q-Hermite~II polynomials@$q$-Hermite~II polynomials!Discrete}
\par\setcounter{equation}{0}

\subsection*{Basic hypergeometric representation}
The discrete $q$-Hermite~II polynomials are Al-Salam-Carlitz~II polynomials
with $a=-1$:
\begin{eqnarray}
\label{DefDiscreteqHermiteII}
{\tilde h}_n(x;q)=i^{-n}V_n^{(-1)}(ix;q)
&=&i^{-n}q^{-\binom{n}{2}}\,\qhyp{2}{0}{q^{-n},ix}{-}{-q^n}\\
&=&x^n
\qhypK{2}{1}{q^{-n},q^{-n+1}}{0}{q^2,-\frac{q^2}{x^2}}\nonumber
\end{eqnarray}

\subsection*{Orthogonality relation}
\begin{eqnarray}
\label{OrtDiscreteqHermiteIIa}
& &\sum_{k=-\infty}^{\infty}\left[{\tilde h}_m(cq^k;q)
{\tilde h}_n(cq^k;q)+{\tilde h}_m(-cq^k;q){\tilde h}_n(-cq^k;q)\right]
w(cq^k;q)q^k\nonumber\\
& &{}=2\frac{(q^2,-c^2q,-c^{-2}q;q^2)_{\infty}}{(q,-c^2,-c^{-2}q^2;q^2)_{\infty}}
\frac{(q;q)_n}{q^{n^2}}\,\delta_{mn},\quad c>0,
\end{eqnarray}
where
$$w(x;q)=\frac{1}{(ix,-ix;q)_{\infty}}=\frac{1}{(-x^2;q^2)_{\infty}}.$$
For $c=1$ this orthogonality relation can also be written as
\begin{equation}
\label{OrtDiscreteqHermiteIIb}
\int_{-\infty}^{\infty}\frac{{\tilde h}_m(x;q){\tilde h}_n(x;q)}{(-x^2;q^2)_{\infty}}\,d_qx
=\frac{\left(q^2,-q,-q;q^2\right)_{\infty}}{\left(q^3,-q^2,-q^2;q^2\right)_{\infty}}
\frac{(q;q)_n}{q^{n^2}}\,\delta_{mn}.
\end{equation}

\subsection*{Recurrence relation}
\begin{equation}
\label{RecDiscreteqHermiteII}
x{\tilde h}_n(x;q)={\tilde h}_{n+1}(x;q)+q^{-2n+1}(1-q^n){\tilde h}_{n-1}(x;q).
\end{equation}

\subsection*{Normalized recurrence relation}
\begin{equation}
\label{NormRecDiscreteqHermiteII}
xp_n(x)=p_{n+1}(x)+q^{-2n+1}(1-q^n)p_{n-1}(x),
\end{equation}
where
$${\tilde h}_n(x;q)=p_n(x).$$

\subsection*{$q$-Difference equation}
\begin{eqnarray}
\label{dvDiscreteqHermiteII}
& &-(1-q^n)x^2{\tilde h}_n(x;q)\nonumber\\
& &{}=(1+x^2){\tilde h}_n(qx;q)-(1+x^2+q){\tilde h}_n(x;q)+q{\tilde h}_n(q^{-1}x;q).
\end{eqnarray}

\subsection*{Forward shift operator}
\begin{equation}
\label{shift1DiscreteqHermiteII-I}
\tilde{h}_n(x;q)-\tilde{h}_n(qx;q)=q^{-n+1}(1-q^n)x\tilde{h}_{n-1}(qx;q)
\end{equation}
or equivalently
\begin{equation}
\label{shift1DiscreteqHermiteII-II}
\mathcal{D}_q\tilde{h}_n(x;q)=\frac{q^{-n+1}(1-q^n)}{1-q}\tilde{h}_{n-1}(qx;q).
\end{equation}

\subsection*{Backward shift operator}
\begin{equation}
\label{shift2DiscreteqHermiteII-I}
\tilde{h}_n(x;q)-(1+x^2)\tilde{h}_n(qx;q)=-q^nx\tilde{h}_{n+1}(x;q)
\end{equation}
or equivalently
\begin{equation}
\label{shift2DiscreteqHermiteII-II}
\mathcal{D}_q\left[w(x;q)\tilde{h}_n(x;q)\right]
=-\frac{q^n}{1-q}w(x;q)\tilde{h}_{n+1}(x;q).
\end{equation}

\subsection*{Rodrigues-type formula}
\begin{equation}
\label{RodDiscreteqHermiteII}
w(x;q)\tilde{h}_n(x;q)=(q-1)^nq^{-\binom{n}{2}}
\left(\mathcal{D}_q\right)^n\left[w(x;q)\right].
\end{equation}

\subsection*{Generating functions}
\begin{equation}
\label{GenDiscreteqHermiteII1}
\frac{(-xt;q)_{\infty}}{(-t^2;q^2)_{\infty}}=\sum_{n=0}^{\infty}
\frac{q^{\binom{n}{2}}}{(q;q)_n}{\tilde h}_n(x;q)t^n.
\end{equation}

\begin{equation}
\label{GenDiscreteqHermiteII2}
(-it;q)_{\infty}\cdot\qhyp{1}{1}{ix}{-it}{it}=\sum_{n=0}^{\infty}
\frac{(-1)^nq^{n(n-1)}}{(q;q)_n}{\tilde h}_n(x;q)t^n.
\end{equation}

\subsection*{Limit relations}

\subsubsection*{Al-Salam-Carlitz~II $\rightarrow$ Discrete $q$-Hermite~II}
The discrete $q$-Hermite~II polynomials given by
(\ref{DefDiscreteqHermiteII}) follow from the Al-Salam-Carlitz~II
polynomials given by (\ref{DefAlSalamCarlitzII}) by the substitution $a=-1$
in the following way:
$$i^{-n}V_n^{(-1)}(ix;q)={\tilde h}_n(x;q).$$

\subsubsection*{Discrete $q$-Hermite~II $\rightarrow$ Hermite}
The Hermite polynomials given by (\ref{DefHermite}) can be found
from the discrete $q$-Hermite~II polynomials given by
(\ref{DefDiscreteqHermiteII}) in the following way:
\begin{equation}
\lim_{q\rightarrow 1}\frac{{\tilde h}_n(x\sqrt{1-q^2};q)}
{(1-q^2)^{\frac{n}{2}}}=\frac{H_n(x)}{2^n}.
\end{equation}

\subsection*{Remark} The discrete $q$-Hermite~II polynomials are related to the
discrete $q$-Hermite~I polynomials given by (\ref{DefDiscreteqHermiteI})
in the following way:
$${\tilde h}_n(x;q^{-1})=i^{-n}h_n(ix;q).$$

\subsection*{References}
\cite{BergIsmail}, \cite{Koorn97}.

\end{document}
